

%============================================================
%  Justin M Selfridge, PhD
%  Gradient Consulting, LLC
%  NSF SBIR 2018 Phase I Proposal
%  Achieving Persistently Enduring Flight With A
%  Tethered Uni-Rotor Network (TURN) System
%============================================================


\documentclass[11pt]{article}


% Define usepackages
\usepackage{amsmath,amsfonts,amssymb}
\usepackage{caption}
\usepackage{enumitem}
\usepackage{fancyhdr}
\usepackage{floatrow}
\usepackage{graphicx}
\usepackage{lipsum}
\usepackage{textcomp,gensymb}
\usepackage{times}
\usepackage[table]{xcolor}


% Document margins
\usepackage[
  top    = 1.0in,
  bottom = 1.0in,
  left   = 1.0in,
  right  = 1.0in ]
  {geometry}


% Footer spacing
\setlength{\footskip}{20pt}


% Section title formatting
\makeatletter
\renewcommand\section{
\@startsection{section}{1}{\z@}%
{4.0ex \@plus 0.1ex \@minus 0.1ex}%
{1.0ex \@plus 0.1ex}%
{\normalfont\LARGE\bfseries}}
\makeatother


% Subsection title formatting
\makeatletter
\renewcommand\subsection{
\@startsection{subsection}{2}{\z@}%
{1.0ex \@plus 0.1ex \@minus 0.1ex}%
{0.01ex \@plus 0.01ex}%
{\normalfont\Large\bfseries}}
\makeatother


% Subsubsection title formatting
\makeatletter
\renewcommand\subsubsection{
\@startsection{subsubsection}{3}{\z@}%
{1.0ex \@plus 0.1ex \@minus 0.1ex}%
{0.0ex \@plus 0.1ex}%
{\normalfont\normalsize\bfseries}}
\makeatother


% Paragraph formatting
\setlength{\parindent}{0pt}
\setlength{\parskip}{6pt plus1pt minus1pt}
\setlist{nolistsep}


% Adjust TOC Indents
\makeatletter
\renewcommand*\l@section{\@dottedtocline{1}{1.5em}{2.3em}}
\renewcommand*\l@subsection{\@dottedtocline{2}{3.8em}{2.3em}}
\renewcommand*\l@subsubsection{\@dottedtocline{3}{6.0em}{2.9em}}
\makeatother


% TOC formatting
\makeatletter
\renewcommand\tableofcontents{%
\null\hfill
\textbf{\normalsize TABLE OF CONTENTS}
\hfill\null
\par
\hspace{0.55cm}
\textbf{\normalsize Section}
\hfill
\textbf{\normalsize Page}
\par
\@starttoc{toc}%
\newpage}
\makeatother


% Caption formatting
\floatsetup[table]{capposition=top}
\captionsetup{labelsep=period}
\captionsetup{labelfont=bf}
\captionsetup{font={small,bf}}


% Custom spacing commands
\newcommand{\PubSpace}{\vspace{-0.12cm}}  % Publication padding
\newcommand{\SecSpace}{\vspace{0.5cm}}  % Add padding to sections


% Common figure dimensions
\newcommand{\Wwhole}   {6.30in}
\newcommand{\Whalf}    {3.00in}
\newcommand{\Wthird}   {2.10in}
\newcommand{\Hwhole}   {8.60in}
\newcommand{\Hhalf}    {3.80in}


% Custom lineweights
\makeatletter
\newcommand{\thickhline}{%
\noalign {\ifnum 0=`}\fi \hrule height 1pt
\futurelet \reserved@a \@xhline }
\newcolumntype{[}{@{\vrule width 1pt\hspace{6pt}}} \newcolumntype{]}{@{\hspace{6pt}\vrule width 1pt}} \newcolumntype{!}{@{\hskip\tabcolsep\vrule width 1pt\hskip\tabcolsep}}
\makeatother


% Color specification
\definecolor{darkblue}{rgb}{0.2,0.2,0.4}


% Table cell background colors
\newcommand{\p}{\cellcolor{red!10}}
\newcommand{\y}{\cellcolor{yellow!10}}
\newcommand{\g}{\cellcolor{green!10}}


% Link customization
\PassOptionsToPackage{hyphens}{url}\usepackage{hyperref}
\hypersetup{colorlinks=true}
\hypersetup{citecolor=darkblue}
\hypersetup{linkcolor=darkblue}
\hypersetup{urlcolor=darkblue}


\usepackage{nameref}
\makeatletter
\newcommand*{\currentname}{\@currentlabelname}
\makeatother


% Header information
\pagestyle{fancy}
\fancyhf{}
\lhead{
  NSF Phase I SBIR 18-550 / Other Topics \\
  Gradient Consulting, LLC }
\rhead{
  Persistently Enduring Flight Utilizing A \\
  Tethered Uni-Rotor Network (TURN) System}
\cfoot{ }


%% Custom hyperlinked subtasks
%\makeatletter
%\let\oldhypertarget\hypertarget
%\renewcommand{\hypertarget}[2]{%
%\oldhypertarget{#1}{#2}%
%\protected@write\@mainaux{}{%
%\string\expandafter\string\gdef
%\string\csname\string\detokenize{#1}\string\endcsname{#2}%
%}%
%}
%\newcommand{\task}[1]{%
%\hyperlink{#1}{\textit{\csname #1\endcsname}}%
%}
%\makeatother








%------------------------------------------------------------
% PROPOSAL INSTRUCTIONS
%------------------------------------------------------------

% Phase I proposals may be submitted for up to $225,000 in R&D funding intended to support projects from 6-12 months in duration. Typically, small businesses will be notified of the award decision four to six months after the submission deadline.

% Phase I Proposal and Program Objectives. An SBIR Phase I proposal must describe the research effort needed to establish the feasibility of the proposed scientific or technical innovation. The primary objectives of the Phase I effort are to (i) determine whether the innovation has sufficient intellectual/technical and broader/commercial impact merit for proceeding into a Phase II project and (ii) to assess commercial feasibility of the proposed innovation. The deliverable of an SBIR Phase I grant is a report describing the technical accomplishments and outcomes of the Phase I project.

% The aim of a Phase I project should be to demonstrate technical feasibility of the proposed innovation and thereby bring the innovation closer to commercialization. Proposals should describe the development of an innovation that demonstrates the following characteristics:
% - Involves a high degree of technical risk – for example: Has never been attempted and/or successfully done before; Is still facing technical hurdles (that the NSF-funded R&D work is intended to overcome).
% - Has the potential for significant commercial impact and/or societal benefit, as evidenced by: Having the potential to disrupt the targeted market segment; Having good product-market fit (as validated by customers); Presenting barriers to entry for competition; Offering potential for societal benefit (through commercialization under a sustainable business model).

% Mandatory Sections:
% - Project Summary (1 pg max)
% - Project Description (15 pg max)
% - References Cited
% - Biographical Sketches
% - Budget and Subaward Budgets
% - Budget Justification
% - Current and Pending Support
% - Collaborators and Other Affiliations
% - Facilities, Equipment and Other Resources
% - Supplementary Documents (all that are applicable)








%------------------------------------------------------------
\begin{document}
%------------------------------------------------------------

\begin{titlepage}

\begin{center}
{$ $}  \\
\LARGE {\bf
Persistently Enduring Flight Utilizing A  \\
Tethered Uni-Rotor Network (TURN) System
}  \\
\vspace{0.5in}
\LARGE \emph{
NSF Phase I SBIR 18-550  \\
Other Topics  {$\quad$}  Proposal: ???
}  \\
\vspace{0.5in}
\Large {
Gradient Consulting, LLC  \\
Justin M Selfridge, PhD
}  \\
\vspace{0.75in}
\end{center}

\begin{figure} [h!]
\includegraphics
[ width=\Wwhole, height=3.2in, clip, trim={150 350 150 0} ]
{../Figures/TURN.jpg}
\end{figure}

\end{titlepage}








%------------------------------------------------------------
\newpage {$ $} \\ 
{\color{red} \bf \Huge Project Summary \\}  % 1 page max
%------------------------------------------------------------
%Information MUST be entered into all three text boxes, or the proposal will not be accepted. Do not upload your Project Summary as a PDF file.


\subsection*{\color{red} Overview, Key Words, and Subtopic Name}
%Describe the potential outcome(s) of the proposed activity in terms of a product, process, or service. Provide a list of key words or phrases that identify the areas of technical expertise to be invoked in reviewing the proposal; and the areas of application that are the initial target of the technology. Provide the subtopic name.

%Developing a unique unmanned aerial system (UAS), which is seeking a persistently enduring flight capability.  Such a vehicle collects enough solar energy during the day to remain aloft throughout the night, thereby eliminating the need to land for refueling.  Accomplishing this feat would enable a new breed of aircraft, called atmospheric satellites, which can provide the same services and functionality as our existing space network, but with some decided advantages.  Low signal latency can achieve broadband speed data transmission rates, station-keeping maintains continuous coverage over a target area, takeoff and landing permits scheduled maintenance and payload upgrades, and eliminating a launch platform vastly reduces program costs.  Conventional aircraft predominantly utilize a common tube-and-wing design approach, but this configuration has been optimized to its aerodynamic and structural limits.  This research considers a novel concept architecture which takes the best features of both glider and helicopter design methodologies, and minimizes their respective shortcomings.  The vehicle, named the Tethered Uni-Rotor Network (TURN), is a low-power system which utilizes centrifugal stiffening as a design element.  This permits high-camber thin-thickness airfoil profiles, that can attain lift-to-drag ratios three times greater than standard practice; and unlike traditional fixed-wing aircraft, the TURN system is capable of vertical takeoff and landing (VTOL).  Persistently enduring flight paired with a VTOL capability offers numerous advantages to both the DoD and the commercial sector.  Previous research, from an Air Force Phase I SBIR, built the first two prototypes within a spiral development program, which delivered custom avionics hardware and software, and helped validate early dynamic simulation models.  This proposed DARPA research will continue that effort with two more prototypes, which will validate the aerodynamic and structural models needed to design subsequent TURN embodiments with confidence.  The culmination of this project will conclude with a demonstrator seeking a world record flight endurance, and continue moving toward the aviation milestone of persistently enduring flight.

%This project presents a novel persistently enduring unmanned aerial system (UAS).  Conventional aircraft predominantly utilize a common tube-and-wing design approach, but this configuration has been optimized to its aerodynamic and structural limits.  Conversely, this research considers a unique concept architecture which takes the best features of both glider and helicopter design methodologies, and minimizes their respective shortcomings.  The vehicle, named the Tethered Uni-Rotor Network (TURN), is a low-power system which utilizes centrifugal stiffening as a design element.  This permits high-camber thin-thickness profiles, that can attain lift-to-drag ratios three times greater than standard airfoils; and unlike traditional fixed-wing aircraft, the TURN system is capable of vertical takeoff and landing (VTOL).  Greatly extended flight endurance paired with a VTOL capability offers numerous advantages to both the DoD and the commercial industry.  Air Force sponsored Phase I SBIR research built the first two prototypes within a spiral development program, which delivered custom avionics hardware and software, and helped validate early simulation models.  Proposed DARPA research will fund the development of the next two prototype installments, which address aerodynamic model validation and structural considerations within a custom designed airfoil.  The project will culminate with a demonstration that is anticipated to set a new world record for flight endurance, and further progress toward persistently enduring flight.

% Centrifugal stiffening within the TURN concept helps minimize vehicle weight while maintaining a robust structural configuration, and allows for much more aerodynamically efficient airfoils to further reduce power consumption.  This vastly increases flight endurance over comparable fixed-wing aircraft, but still offers a VTOL capability.  Various scales of the TURN system would satisfy different mission requirements for both the DoD and the commercial sector.  First, consider a small-scale TURN system with electric propulsion.  While existing fixed-wing aircraft achieve a 90-minute flight endurance with a five pound payload, a TURN system of comparable size and weight could remain airborne for over six hours.  This would serve as a mobile remote ISR platform to aid ground troops, or could be used for precision agriculture and construction applications.  Second, a medium-scale vehicle utilizing a combustion engine, which is the focus of the Air Force SBIR research, supporting a 250 pound payload while drawing 2000 watts of power, would remain airborne for several weeks at a time.  This medium-altitude embodiment would reduce costs and improve operational logistics for ISR pattern-of-life missions, and could augment border patrol monitoring.  Third, the largest-scale TURN system, and the subject matter of this proposed DARPA research, is designed to attain a persistently enduring flight capability.  While operating within the stratosphere, the vehicle collects enough solar energy during the day to remain aloft throughout the night, thereby eliminating the need to land for refueling.  This capability would enable atmospheric satellites, which can augment/replace our existing space satellite network, but at a greatly reduced cost.  Furthermore, closer proximity to the earth allows for atmospheric weather monitoring, and reduces signal latency permitting broadband speed data transmission rates.  This could serve the DoD as an alternative position, navigation and timing (APNT) solution, or as a network relay station.  Meanwhile, Google and Facebook are both striving for this aviation milestone to bring the Internet to the most remote parts of the planet.

%Many unique benefits arise from the TURN system.  Advantageous centrifugal stiffening is similar to traditional helicopters, except thin tether filaments replace wasted material with detrimental aerodynamic properties.  Outboard sections are free to pivot, so there is no adverse bending moment typically found at a wing root.  This permits much higher aspect ratio, increased lift-to-drag ratio, and lower thickness-to-chord ratio, than can be attained by a tube-and-wing aircraft.  This new approach dramatically lowers structural weight, greatly reduces drag, and provides a far more robust structure than existing designs.  Consider some of the system benefits in more detail.

%The goal is to develop the set of tools and models to design a TURN system with confidence.  Spiral development is an iterative approach, where prototypes address the largest risk, simulation models are refined with flight data, and enhanced analytic tools aid with subsequent vehicle designs.

%Modeled nonlinear dynamics within a numerical multibody simulation.  Presented the initial controls analysis, outlined the controller methodology, and introduced inner-loop and outer-loop control systems, which handle stabilization and navigation.  Multivariable adaptive control reduced cross-coupling between inputs and outputs and mitigated plant model uncertainties.

%Developed aerodynamic and structural models of the TURN system, through an Air Force Phase I SBIR.  Analysis shows persistently enduring flight is feasible for a wide range of latitudes and day-of-year.  Also, considered a combustion engine TURN embodiment, which operates at 15k feet and provides 2000W of power to a 250 lbf payload.  Additional SBIR funding will pursue a 30-day flight endurance utilizing a combustion engine TURN system.

%Atmospheric satellites are a new breed of aircraft designed to operate at altitudes around 60,000 feet, which offers air density and winds aloft that require the lowest power consumption while minimizing weather concerns.  A single system can offer continuous coverage because it operates over relatively fixed locations; and it is fifty times closer than LEO satellites, so the signal latency is extremely low.  Cost of deployment and recovery is inexpensive because the aircraft is able to takeoff and land, so the total system cost is expected to be several orders of magnitude less than other satellite solutions, with the added benefit of upgrading and refurbishing satellite payloads as equipment ages.  At this time, existing HALE aircraft can only maintain flight for several weeks at a time, and lack the highly reliable and persistent capabilities that satellites offer.  To successfully compete with existing satellite technology, an operational atmospheric satellite product must demonstrate a minimum level of proficiency.  It must be capable of controlled flight, remain airborne indefinitely throughout the year, and only need to land for routine maintenance.  It needs a sufficiently large margin of error to operate at high latitudes, overcome seasonal wind, and accommodate solar flux variations.  It cannot utilize altitude energy storing, because weather effects hamper station-keeping, and line of sight is reduced at lower altitudes.  For these reasons, a new concept approach is needed to tackle the persistently enduring flight problem.

%------------------------------
%{\bf Airfoil Selection:}
%An analysis investigated the influence of two primary airfoil parameters: thickness and camber.  For each group, five sample profiles were compiled from an airfoil database.  Camber has a diminished influence on thick airfoils, and the $L/D_\text{max}$ ratio falls within a 70-110 range.  Camber has a much more pronounced effect on thin airfoils, where the $L/D_\text{max}$ averaged over 200, with a maximum of 237.  Furthermore, they demonstrate greater separation between cruise and stall, which helps improve wind robustness at slow airspeeds.  Thick airfoils are commonly used to reinforce flimsy wing structures, but with with centrifugal stiffening within the TURN concept, slender airfoils with high camber are feasible, and can achieve extraordinarily high $L/D$ ratios, as seen in Figure \ref{Fig:Eppler}.  Finally, the proposed research budgets for a custom designed airfoil, which may attain values beyond what is readily available in the existing airfoil database.
%
%\begin{figure}
%\begin{floatrow}
%\includegraphics
%[ width=\Wthird, height=1.7in, clip, trim={0 0 0 0} ]
%{../Figures/Eppler_LDAlpha.jpg}
%\hspace{0.1cm}
%\includegraphics
%[ width=\Wthird, height=1.7in, clip, trim={0 0 0 0} ]
%{../Figures/Eppler_ClAlpha.jpg}
%\hspace{0.1cm}
%\includegraphics
%[ width=\Wthird, height=1.7in, clip, trim={0 0 0 0} ]
%{../Figures/Eppler_CdAlpha.jpg}
%\end{floatrow}
%\caption{Eppler 58 Airfoil Used for TURN Sizing Analysis}
%\label{Fig:Eppler}
%\end{figure}




\subsection*{\color{red} Intellectual Merit}
%This section MUST begin with "This Small Business Innovation Research Phase I project..." Address the intellectual merits of the proposed activity. Do not include proprietary information in the summary. Briefly describe the technical hurdle(s) that will be addressed by the proposed R&D (which should be crucial to successful commercialization of the innovation), the goals of the proposed R&D, and a high-level summary of the plan to reach those goals.

Insert.


\subsection*{\color{red} Broader/Commercial Impact}
%In this field, discuss the expected outcomes in terms of how the proposed project will bring the innovation closer to commercialization under a sustainable business model. In this box, also describe the potential commercial and market impacts that such a commercialization effort would have, if successful.  As appropriate, also discuss potential broader societal impacts of the innovation (e.g. educational, environmental, scientific, societal, or other impacts on the nation and the world).

Insert.








%------------------------------------------------------------
\newpage
\setcounter{page}{1}
\cfoot{\thepage}
{\bf \Huge Project Description}  % 15 page max
%------------------------------------------------------------

Flight has always captured man's imagination, which is evidenced by the great variety of aerial vehicles that exist today.  Everything from fixed-wing to rotorcraft; satellites to spaceships; mono-wing to quadrotor.  However, despite the wide variety of existing aircraft, none have attained persistently enduring flight.  Accomplishing this aviation milestone is one of the great challenges still facing the aerospace community.




%------------------------------------------------------------
\section*{\color{red} Elevator Pitch}  % (no more than one page)


%------------------------------
\subsection*{\color{red} The Customer}
%------------------------------
% The Customer. Describe the expected customer for the innovation. What customer needs or market pain points are you addressing?


%------------------------------
\subsection*{\color{red} The Value Proposition}
%------------------------------
% The Value Proposition. What are the benefits to the customer of your proposed innovation? What is the key differentiator of your company or technology? What is the potential societal value of your innovation?


%------------------------------
\subsection*{\color{red} The Innovation}
%------------------------------
% The Innovation: Succinctly describe your innovation. This section can contain proprietary information that could not be discussed in the Project Summary. What aspects are original, unusual, novel, disruptive, or transformative compared to the current state of the art?








%------------------------------------------------------------
\newpage
\section*{The Commercial Opportunity}

Insert.




%------------------------------
\subsection*{Market Assessment}
%------------------------------

Unmanned aerial systems are the fastest growing segment within the aerospace industry.  Even so, the number of potential applications is still limited by some common technical challenges.


%------------------------------
{\bf Unmanned System Market Growth:}
According to the Teal Group's 2017 World Civil UAS Market Profile and Forecast, non-military UAS production will total \$73.5 billion in the next decade, soaring from \$2.8 billion worldwide in 2017 to \$11.8 billion in 2026, which is a 15.5\% compound annual growth rate in constant dollars.  Commercial use will be the fastest growing civil segment, rising more than twelve-fold from \$512 million in 2017 to \$6.5 billion.  As indicated in Figure \ref{Fig:Comm_Invest}, technology companies and venture capitalists have invested more than \$3 billion into the UAS sector since 2014.  Acquisition patterns follow investment patterns, and large companies are seeking UAS technologies.  Intel, Verizon, Facebook and Google have all made significant investments or acquisitions within their UAS portfolios.  Other technology leaders, including Amazon and General Electric, have focused on building up their own internal capabilities.

\begin{figure} [h!]
\includegraphics 
[ scale=0.92, clip, trim={0 0 0 0} ]
{../Figures/Comm_Invest.png}
\caption{Investments in UAS Companies Since 2014}
\label{Fig:Comm_Invest}
\end{figure}


%------------------------------
{\bf Technical Challenges:}
While the UAS market is expected to grow rapidly, their usefulness is largely predicated on flight endurance.  Flight endurance is the key parameter that dictates: range, time on station, size and weight of the vehicle and payload, coverage area, downtime for refueling, and number of vehicles required for a specified mission; just to name a few.  For comparable size/weight/cost, low-power and aerodynamically efficient UAVs are more desirable because they offer extended flight endurance.  Furthermore, for many potential applications, a VTOL capability is an absolute necessity.  Under these scenarios, multirotors are the only readily available option, but they can only operate for a small fraction of the time as a comparable fixed-wing system.  The TURN system addresses each of these challenges, offers a VTOL capability, and holds the promise of true ultra-endurance flight.


%------------------------------
{\bf Competitive Landscape:}
The UAV market is populated with many competitive equipment manufacturers.  As previously stated, they have been successful in attracting venture capital, so investment patterns are a reasonable way to identify potential competitors.  Figure \ref{Fig:Comm_Comp} presents the most significant competitors, and the total third-party investment made within each company.  While the space is densely populated, a TURN architecture offers significant advantages over existing solutions.  For small systems, multirotors typically offer 20-30 minute flight operations, while fixed-wing aircraft average around 90 minutes.  Medium size systems, primarily for ISR missions, typically support payloads of a few hundred pounds for several days at a time.  Finally, low-cost, high-altitude, ultra-endurance UAS for internet and communications, promise to create an entirely new segment of the market.  Airbus has already begun low rate production of solar-powered UAS, and both Google and Facebook have been developing their own technologies.  This analysis identified several important elements of a commercialization strategy, including: a broad range of possible customers in several different markets, some potential ``hot spot'' sectors that would benefit most from the unique attributes of a TURN system, and a focus on verticals that provide a large addressable market.

\begin{figure} [h!]
\includegraphics 
[ scale=0.8, clip, trim={0 0 0 0} ]
{../Figures/Comm_Comp.png}
\caption{Venture Capital Investments in Specific Drone Companies}
\label{Fig:Comm_Comp}
\end{figure}




%------------------------------
\subsection*{Target Customers}
%------------------------------

UAV platforms, and their potential use in commercial and military applications, have become nearly ubiquitous in the United States. Applications for UAV technologies are wide-ranging, from precision agriculture, inspection, infrastructure monitoring, crowd control, first-responder incident evaluation and management, to name a few.  As previously noted, the UAV market is both large and growing rapidly.


%------------------------------
{\bf Target Customers:}
Based on this analysis, the ideal target customers during first market entry are within the \emph{Infrastructure} and \emph{Agriculture} market segments.  Combined, those markets have potential for more than \$77 billion of product and service potential, as indicated in Figure \ref{Fig:Comm_Industry}.

\begin{figure}
\includegraphics 
[ scale=0.85, clip, trim={0 0 0 0} ]
{../Figures/Comm_Industry.png}
\caption{Market Opportunity by UAS Industry Sector}
\label{Fig:Comm_Industry}
\end{figure}


%------------------------------
{\bf Infrastructure:}
Infrastructure (including the construction sector) will lead the commercial market.  All ten of the largest worldwide construction firms are experimenting with unmanned systems and are working to quickly deploy fleets worldwide.  Three of the largest construction equipment suppliers have agreements with drone manufacturers to provide everything from off-the-shelf systems to full end-to-end services.  This addressable market segment, and its related services, is estimated at more than \$45 billion dollars by 2021.


%------------------------------
{\bf Agriculture:}
Agriculture, which will adopt UAVs more slowly, will still rank second worldwide over the next decade.  The ability to provide imagery that can detect when and where to apply fertilizer, pesticides and water offers tremendous potential.  While the agricultural UAV market will grow more slowly than construction or communication sectors, gradual adoption by skeptical farmers and improved technology will allow it to grow considerably and reach its full potential beyond the forecast period.  This addressable market segment is estimated to be more than \$32 billion dollars by 2021.




%------------------------------
\subsection*{Commercialization Implementation}
%------------------------------

Transitioning the TURN technology from basic research into a commercially viable product will require additional outside capital and specialized expertise from key individuals.


%------------------------------
{\bf Financing Plan:}
To arrive at an estimate of required funds to commercialize the TURN system, a preliminary top-level financial plan is presented.  It includes the technology/product development effort described for the current Phase II SBIR, and then considers potential spin-offs of other TURN vehicles at various scales serving different demographics.  From that foundation, initial required funding is comprised of: (i) \$3.0M of additional research development and testing, including subsequent prototype building and allocation for two engineers in the second year; (ii) \$50K for preparation and filing of utility patent applications; (iii) \$150K for business development including: fostering of strategic partnerships, travel funds, attorney fees, and other expenses; and (iv) \$300K for initial commercialization and marketing/advertising/promotional endeavors.  Thus, approximately \$3.5M is needed to fund the remaining research and initial commercialization activities contemplated above.  After developing the thirty-day endurance TURN system for the Air Force, ancillary markets will be pursued, including the small scale remote deployment system and the larger persistently enduring flight embodiment.  Growing the enterprise at this level will require executive leadership recruitment, additional business and marketing managers, and an expanded engineering staff.  These activities will likely require later-stage venture (equity) or debt financing in the amount of \$5.0M.  Therefore, estimated total funds for commercialization of the Gradient TURN technology is \$8.5M.


%------------------------------
{\bf Marketing Expertise:}
Gradient has tremendous experience in the R\&D/T\&E of new technologies, as well as some recent experience in product development.  To supplement that technical skill set, Gradient will retain the services of consultants and contractors experienced in appropriate fields.  In addition, Gradient has formed an Advisory Board to round out the necessary skill-set until full-time employees can be justified by business and technology maturity.  The skill-sets contained on the Gradient Advisory Board include product marketing, venture capital/venture development, industrial product development, technology commercialization, and intellectual property protection.








%------------------------------------------------------------
\section*{The Innovation}

Increased flight endurance is the most sought after improvement across the drone industry.  Unfortunately, tube-and-wing design aircraft have already reached their optimal structural and aerodynamic limits.  Thus, achieving a step-change improvement in aerodynamic performance requires a radical new concept approach.




%------------------------------
\subsection*{Radical New Concept Approach}
%------------------------------

The next section presents a novel UAS, which combines the best features of both gliders and helicopters, while minimizing their respective deficiencies.  Before introducing the concept vehicle, pros and cons of both gliders and helicopters are presented, to better illustrate the benefits of this radical new design.

\begin{figure} [!b]
\begin{floatrow}
\hspace{-0.1cm}
\ffigbox[\FBwidth]{
\includegraphics 
[ width=\Whalf, height=2.0in, clip, trim={0 0 0 0} ]
{../Figures/Glider}}
{\caption{Glider with High Aspect Ratio Wing}
\label{Fig:Glider}}
\hspace{-0.4cm}
\ffigbox[\FBwidth]{
\includegraphics 
[ width=\Whalf, height=2.0in, clip, trim={0 0 0 0} ]
{../Figures/Helicopter}}
{\caption{Helicopter Rotors with Deflection}
\label{Fig:Helicopter}}
\end{floatrow}
\end{figure}


%------------------------------
{\bf Glider Attributes:}
Gliders are some of the most aerodynamically efficient aircraft, because they strive to minimize drag.  They typically employ long slender wings, as seen in Figure \ref{Fig:Glider}, thereby increasing the aspect ratio, which is desirable from an aerodynamic drag perspective, but there are limitations with this approach.  Slender wings introduce flexibility, which can exhibit bend and twist during flight.  These wings are subjected to greater bending moments at the wing root, so added structural material must reinforce this connection point.  Finally, like all fixed-wing aircraft, gliders need forward velocity to generate airflow over the airfoil which produces lift, so they cannot hover or takeoff vertically.


%------------------------------
{\bf Helicopter Attributes:}
Unlike fixed-wing aircraft, helicopters have desirable VTOL and hovering capabilities.  Typical rotor blades are so thin, that they bend and deflect under their own weight, demonstrated in Figure \ref{Fig:Helicopter}.  Yet, these flimsy structures lift substantially heavy vehicles, because centrifugal forces provide stiffening throughout the rotor element.  Unfortunately, helicopter rotors are not aerodynamically efficient, because they suffer from triangular span loading.  Outboard sections produce a majority of the lift, while the inboard sections are quite ineffectual.  Finally, rotor mechanisms are extremely complex and must withstand massive internal structural forces.


%------------------------------
{\bf Best of Both Worlds:}
A completely new aerial concept vehicle incorporates features from both glider and helicopter methodologies.  Gliders utilize extremely efficient wings, but must mitigate flexibility and bending moments between the wing and the fuselage.  Helicopters support large payloads on flimsy rotors with centrifugal stiffening, but suffer from inefficient aerodynamic drag characteristics.  A new concept combines the best attributes of both types of aircraft, and minimizes the deficiencies of each, where centrifugal stiffening mitigates flexibility and moments found on gliders, while replacing inefficient rotors with extremely aerodynamic, low drag, high aspect ratio airfoils.




%------------------------------
\subsection*{Introducing the Tethered Uni-Rotor Network}
%------------------------------

The design objective reduced structural mass, maintained robust structural configuration, and simultaneously increased the lift-to-drag ratio.  A tensegrity approach minimized compression by putting structural loads in tension, where advanced composites provide incredible strength to weight.  The concept combines the best features of high efficiency glider and helicopter designs, reduces deficiencies within each class of aircraft, and yields better attributes than either can provide on their own.  This novel aerial vehicle, called the Tethered Uni-Rotor Network (TURN), illustrated in Figures \ref{Fig:TURN_Retract} and \ref{Fig:TURN_Extend}, utilizes a unique alternative approach, which exceeds existing capabilities, outpaces current research efforts, and vastly reduces power consumption.

\begin{figure} [!b]
\includegraphics 
[ width=\Wwhole, height=3.2in, clip, trim={0 0 0 0} ]
{../Figures/TURN_Retract}
\caption{Tethers are Retracted for Takeoff and Landing}
\label{Fig:TURN_Retract}
\end{figure}


%------------------------------
{\bf Vehicle Layout:}
The TURN system has a central hub which stores the vehicle payload.  Four small diameter cable tethers radiate outwards from the central hub and attach to satellite bodies.  Each has typical aircraft components, including: airfoil wing, propeller for thrust, stabilizers and control surfaces, and a fuselage for fuel, hardware, and sensors.  Each satellite resembles a flying-wing aircraft, which provides all the lift, propulsion and control for the TURN system.  A propeller is mounted on the leading edge of the outer wingtip; and immediately behind the prop, vertical stabilizers and control surfaces are located directly in the prop wash.  The concept is named \emph{Tethered Uni-Rotor Network} because a network of aircraft systems are tethered together to form a much larger singular rotor system.


%------------------------------
{\bf Hover Operation:}
The vehicle operates in a perpetual state of rotation, where each satellite drives the system rotation with its respective propulsion system.  This is opposite of a conventional helicopter, where a torque acts on a central shaft; with a tip driven system, no torque is transmitted back to the central hub.  As the system spins, centrifugal forces keep the tethers taught, and mitigate moments common within thin wings.  Lift is generated on each of the winged airfoil sections, which counteracts the weight of the satellite, and indirectly supports the weight of the central hub.

\begin{figure}
\includegraphics 
[ width=\Wwhole, height=3.2in, clip, trim={0 0 0 0} ]
{../Figures/TURN_Extend}
\caption{Normal Flight Operation with Tethers Fully Extended}
\label{Fig:TURN_Extend}
\end{figure}


%------------------------------
{\bf Flight Operation:}
Individual satellites are controlled through their propeller and control surfaces, and the central hub is controlled by coordinating the tether forces imparted from the satellite vehicles.  Two types of translation are considered, vertical and horizontal, which have a parallel in helicopter terminology as collective and cyclic commands.  Each type of translation has two associated control modes which can be implemented through different control inputs.


%------------------------------
{\bf Vertical Translation:}
Collective commands alter each satellite control input in unison.  Adjusting throttle increases or decreases the velocity of the satellite, and thus the angular rate of the TURN system.  This changes the airflow across the wing, which increases or decreases the total lift generated, and causes the vehicle to ascend or descend.  Alternatively, adjusting the pitch of each satellite through the elevator control surface, causes each satellite to nose up or down, thus the entire system will climb or fall as each satellite moves through a spiral trajectory.


%------------------------------
{\bf Horizontal Translation:}
Cyclic commands alter the control inputs in a sinusoidal fashion throughout the rotation.  The first cyclic command, applied to the rudder control surface, manipulates the radial tension on the tether.  At one point in the rotation there is maximum tension, and exactly opposite there is a minimum tension; this imbalance induces horizontal translation.  Another approach uses cyclic elevator commands, such that at one point in the rotation a satellite passes a low elevation, and it traverses a high elevation exactly opposite that point. This tilts the plane of rotation, which inclines the thrust vector of the airfoils, and yields a horizontal force component, which is identical to how traditional multirotors translate horizontally.


%------------------------------
{\bf Vertical Takeoff and Landing:}
The TURN system begins at rest on the tarmac with retracted tethers, as depicted in Figure \ref{Fig:TURN_Retract}.  Each wing is initially secured to the central hub with a docking mechanism which permits roll and pitch articulation within each satellite wing.  As the propellers ramp up, the TURN system begins to spin, while rolling on small wheels embedded within the vertical stabilizers.  Once sufficient angular rate has been attained, the system induces liftoff and rises above any local obstructions.  At a suitable altitude, each of the satellites are released from their docking mechanisms, and the tethers are extended to their normal flight operation.  Landing follows an identical procedure in the reverse order.




%------------------------------
\subsection*{System Benefits}
%------------------------------

Many unique benefits arise from the TURN system.  Advantageous centrifugal stiffening is similar to traditional helicopters, except thin tether filaments replace wasted material with detrimental aerodynamic properties.  Outboard sections are free to pivot, so there is no adverse bending moment typically found at a wing root.  This permits much higher aspect ratio, increased lift-to-drag ratio, and lower thickness-to-chord ratio, than can be attained by a tube-and-wing aircraft.  This new approach dramatically lowers structural weight, greatly reduces drag, and provides a far more robust structure than existing designs.  Consider some of the system benefits in more detail.


%------------------------------
{\bf Centrifugal Stiffening:}
Thin airfoils are especially susceptible to bending moments, so centrifugal stiffening is one of the most desirable attributes of the TURN system methodology, since very little material is needed within the spar.  The NASA/AeroVironment Helios, shown in Figure \ref{Fig:Helios}, is an example of a HALE aircraft experiencing very high bending moments across a large wing span, which ultimately failed from a lack of structural stiffness.  The TURN design places most of the mass at the largest possible radius, thereby reducing the rotor-wing speed, and increasing the amount of structural stiffening.  Utilizing a large tip mass, the TURN system only experiences a few g's of radial acceleration.

\begin{figure} [b!]
\captionsetup{width=0.9\textwidth}
\centering
\begin{minipage}{0.6\textwidth}
\captionsetup{width=0.9\textwidth}
\centering
\includegraphics 
[ width=1.01\linewidth, height=2.3in, clip, trim={0 0 0 0} ]
{../Figures/Helios_Flight}
\end{minipage}%
\begin{minipage}{0.4\textwidth}
\captionsetup{width=0.9\textwidth}
\centering
\includegraphics 
[ width=1.0\linewidth, height=2.3in, clip, trim={120 0 0 0} ]
{../Figures/Helios_Break}
\end{minipage}
\caption{Helios Aircraft with Severe Bending Moment and the Outcome From a Lack of Stiffness}
\label{Fig:Helios}
\end{figure}


%------------------------------
{\bf High Aspect Ratios:}
High aspect ratio wings improve aerodynamic efficiency, but the wing weight fraction is another closely related performance indicator.  Typical aircraft wings are commonly 10-20\% of the gross aircraft weight, and rarely exceed aspect ratios of ten.  Most HALE aircraft increase the aspect ratio of a tube-and-wing design, while pushing the limits of exotic structural materials.  While targeting aspect ratios of 20-25, they all experience similar wing weight fractions around 40-50\% of the total aircraft, indicating the wing receives extra material to reinforce a flimsy structure.  Conversely, most rotorcraft blades have aspect ratios of 40-50 and typically only represent 2-3\% of the gross helicopter weight.  Clearly, centrifugally stiffened thin structures can substantially reduce weight while using greater aspect ratios.  This approach used in the TURN concept permits high aspect ratios within very thin and lightweight airfoil structures.


%------------------------------
{\bf Tether:}
Helicopter rotors are inefficient, because only the outermost portions generate lift.  The inner sections are quite ineffectual, but still subjected to induced drag.  With this concept, the tether is 80\% of the total radius, which eliminates an underutilized structure.  At first glance, the tether may appear to significantly increase the parasitic drag; however, it is made of advanced aerospace grade compressed Spectra, which provides incredible strength-to-weight, so even a small tether cable diameter is structurally efficient at carrying tensile loads.  Because the average tether speed is only 40\% of the tip speed, the tether drag only constitutes around 5-10\% of the total drag acting on the system.


%------------------------------
{\bf High Lift-to-Drag Ratios:}
Ultra-endurance flight must resolve the conflict between the forces needed to keep the system aloft versus the power required to do so.  Thus, a high lift-to-drag ratio is extremely important.  Low bending moments permit very thin airfoil sections with low thickness-to-chord ratios, which is especially important for aircraft operating at a low Reynolds number.  This maintains laminar flow, which achieves extremely low levels of induced and parasitic drag.  With this concept, the tether comprises the innermost 80\% of the arm, so the satellites have a large radius of rotation which enables the wings to achieve elliptic span loading, rather than an inefficient triangular load distribution found on helicopters.  Helicopter rotors perform in close proximity of the downwash of the blade rotating in front of it, which increases the required angle of attack, and leads to high induced drag.  Each TURN satellite has a large spatial separation, so they operate independently of other downwash fields.  Because the total rotor disc area is directly related to the induced drag, the TURN approach offers a distinct advantage over conventional rotor designs.


%------------------------------
{\bf Rotating System:}
Conventional aircraft designs require that payload volumes move at the same speed as the rest of the vehicle, which results in substantial parasitic drag from bulky antenna or optic payloads.  Conversely, all the lift from the TURN concept is generated within the moving satellites, which are independent from the stationary hub.  This means larger and aerodynamically inefficient payload volumes can be incorporated into the central hub without incurring similar drag penalties, because the payload is isolated from the cruise speed components which generate lift.




%------------------------------
\subsection*{Various Scale TURN Embodiments}
%------------------------------

Improved aerodynamic efficiency, increased flight endurance, and a VTOL capability, offers many unique possibilities at a variety of different scales.  Consider some applications for three specific embodiments.


%------------------------------
{\bf Solar Powered Persistently Enduring Flight:}
Previous doctoral research, performed by the PI, investigated the TURN system as a means of attaining persistently enduring flight.  Utilizing readily available solar cell and battery technology, the TURN system shows great promise for attaining this goal at a wide range of latitudes through the entire year.  The results that are displayed in Table \ref{Tbl:SolarMap} show the amount of solar energy collection at a specified day and latitude.  Entries are coded such that red indicates there is not solar energy available for the task, green attains the goal with a 250 pound payload drawing 2000W of power, and yellow attains persistently enduring flight while supporting a 100 pound payload which consumes 750W of power.

\newcommand{\cp}[1]{$\;\;$#1$\;\;$}
\begin{table}
\small
\renewcommand{\arraystretch}{0.93}
\begin{tabular}{[c!cccccccccc]}
\thickhline
\bf Lat/Day&
\bf \cp{00}  & 
\bf \cp{40}  & 
\bf \cp{80}  & 
\bf \cp{120} & 
\bf \cp{160} &
\bf \cp{200} & 
\bf \cp{240} & 
\bf \cp{280} & 
\bf \cp{320} & 
\bf \cp{360} \\
\thickhline
\bf    90 &
\p   0.00 & \p  0.00 & \p  0.00 & \y  4.50 & \g  6.99 & 
\g   6.40 & \y  2.91 & \p  0.00 & \p  0.00 & \p  0.00 \\
\bf    85 &
\p   0.00 & \p  0.00 & \p  0.77 & \g  6.00 & \g  8.41 &
\g   7.84 & \y  4.44 & \p  0.00 & \p  0.00 & \p  0.00 \\
\bf    80 &
\p   0.00 & \p  0.00 & \y  1.55 & \g  7.45 & \g  9.76 &
\g   9.22 & \g  5.94 & \p  0.40 & \p  0.00 & \p  0.00 \\
\bf    75 &
\p   0.00 & \p  0.00 & \y  2.32 & \g  8.85 & \g 11.04 &
\g  10.53 & \g  5.58 & \y  1.09 & \p  0.00 & \p  0.00 \\
\bf    70 &
\p   0.00 & \p  0.38 & \y  3.07 & \g  7.87 & \g 12.23 &
\g  11.76 & \g  5.96 & \y  1.82 & \p  0.03 & \p  0.00 \\
\bf    65 &
\p   0.09 & \p  0.97 & \y  3.80 & \g  8.02 & \g 11.88 &
\g  10.70 & \g  6.44 & \y  2.56 & \p  0.46 & \p  0.08 \\
\bf    60 &
\p   0.54 & \y  1.63 & \y  4.50 & \g  8.30 & \g 11.12 &
\g  10.39 & \g  6.93 & \y  3.29 & \y  1.05 & \p  0.51 \\
\bf    55 &
\y   1.12 & \y  2.33 & \g  5.16 & \g  8.60 & \g 10.91 &
\g  10.34 & \g  7.40 & \y  4.01 & \y  1.70 & \y  1.08 \\
\bf    50 &
\y   1.77 & \y  3.04 & \g  5.79 & \g  8.89 & \g 10.83 &
\g  10.36 & \g  7.83 & \y  4.69 & \y  2.39 & \y  1.73 \\
\bf    45 &
\y   2.46 & \y  3.75 & \g  6.37 & \g  9.14 & \g 10.76 &
\g  10.38 & \g  8.22 & \g  5.35 & \y  3.10 & \y  2.42 \\
\bf    40 &
\y   3.17 & \y  4.45 & \g  6.91 & \g  9.34 & \g 10.69 &
\g  10.38 & \g  8.55 & \g  5.96 & \y  3.81 & \y  3.13 \\
\bf    35 &
\y   3.89 & \g  5.12 & \g  7.39 & \g  9.49 & \g 10.57 &
\g  10.33 & \g  8.83 & \g  6.53 & \y  4.51 & \y  3.85 \\
\bf    30 &
\y   4.60 & \g  5.77 & \g  7.82 & \g  9.58 & \g 10.42 &
\g  10.24 & \g  9.04 & \g  7.06 & \g  5.20 & \y  4.56 \\
\bf    25 &
\g   5.30 & \g  6.38 & \g  8.19 & \g  9.61 & \g 10.21 &
\g  10.09 & \g  9.19 & \g  7.53 & \g  5.85 & \g  5.26 \\
\bf    20 &
\g   5.98 & \g  6.95 & \g  8.49 & \g  9.57 & \g  9.95 &
\g   9.88 & \g  9.28 & \g  7.95 & \g  6.48 & \g  5.94 \\
\bf    15 &
\g   6.63 & \g  7.48 & \g  8.74 & \g  9.47 & \g  9.63 &
\g   9.61 & \g  9.30 & \g  8.31 & \g  7.07 & \g  6.60 \\
\bf    10 &
\g   7.24 & \g  7.96 & \g  8.91 & \g  9.30 & \g  9.26 &
\g   9.29 & \g  9.25 & \g  8.61 & \g  7.62 & \g  7.22 \\
\bf     5 &
\g   7.82 & \g  8.38 & \g  9.02 & \g  9.07 & \g  8.84 &
\g   8.91 & \g  9.13 & \g  8.84 & \g  8.11 & \g  7.79 \\
\bf     0 &
\g   8.35 & \g  8.75 & \g  9.07 & \g  8.78 & \g  8.36 &
\g   8.48 & \g  8.95 & \g  9.01 & \g  8.56 & \g  8.33 \\
\thickhline
\end{tabular}
{\caption{Solar Collection Map: Daily Energy per Solar Array Area (MJ m${}^{\text{-2}}$ day${}^{\text{-1}}$)}
\label{Tbl:SolarMap}}
\end{table}


%------------------------------
{\bf Combustion Engine Ultra-Endurance UAV:}
Gradient recently completed a Phase I SBIR research project, sponsored by the Air Force, to investigate the potential for a medium-altitude combustion engine TURN embodiment.  Existing surveillance drones are limited by their flight endurance, creating logistical problems, from finding a suitable launch area, to deploying multiple UAVs in waves to maintain continuous coverage.  An ultra-endurance flight capability solves these problems.  When flight endurance is no longer part of the equation, the launch site does not need to be within a particular radius of the target area, and multiple drones are not required to maintain continuous coverage of a desired target.  An individual TURN system can be deployed, navigate over to the point of interest several thousand miles away, and station-keep over the desired location for several weeks.  Finally, utilizing high TRL components, makes this a very reliable system, without depending on unproven experimental technology.  Some results from this work are depicted in Figure \ref{Fig:Endur}, which illustrates that this system is capable of attaining a 30-day flight endurance.

\begin{figure}
\includegraphics 
[ width=\Wwhole, height=3.2in, clip, trim={0 0 0 0} ]
{../Figures/Endur.png}
\caption{Flight endurance for combustion engine TURN system at various airspeeds and gust tolerance.}
\label{Fig:Endur}
\end{figure}


%------------------------------
{\bf Electric Remote Deployment UAV:}
Finally, this proposed NSF research will investigate the smallest TURN embodiment.  Existing 10-foot wingspan UAVs can carry 5 pound payloads for roughly 90 minutes, and many require a runway or catapult system for operations.  Conversely, a TURN system with an identical payload and battery mass can offer a VTOL capability, and is expected to attain a flight endurance of 4-7 hours.  With 6-foot rotor wings, and a system weight of about 35 pounds, this TURN embodiment would easily fit within a truck or van, could be transported to remote deployment locations, and operate without the need for a runway or ancillary equipment.  Because the research is utilizing a spiral development process, small prototypes are laying the groundwork for larger prototype development.  As such, this system represents the first commercial offering, before advancing to larger and more elaborate TURN embodiments.








%------------------------------------------------------------
\section*{The Company/Team}

Developing the TURN aircraft requires a multidisciplinary effort, which will utilize specialized research groups across several domains.  Gradient Consulting will manage the overall project, and expand upon the existing dynamic system model and controls architecture.  DAR Corporation is responsible for developing the CFD, FEA, and aeroelastic analysis for the TURN system.  StarLab is a VICON motion capture studio which will collect high fidelity flight data used for parameter identification.




%------------------------------
\subsection*{Gradient Consulting [Dynamic Modeling and Controls Development]}
%------------------------------

The crux of the spiral development cycle resides within the model validation process.  Matching computer simulations to their physical counterparts, and identifying and assessing the discrepancies, increases confidence within the analytic tools utilized during subsequent prototype development.  Gradient Consulting specializes in modeling nonlinear dynamics of multibody systems, and implementing advanced model reference adaptive controls (MRAC) architectures on multiple-input multiple-output (MIMO) plants, to mitigate modeling uncertainty and environmental disturbances.  Parameter identification is used while investigating new concept aircraft architectures, which obtains the physical characteristics of an unknown system.  Multisine signal injection routines are sinusoidal signals superimposed on the control inputs to facilitate parameter convergence to their true values.  Once the vehicle dynamics are known through flight testing, simulation models are optimized to match their physical counterparts, yielding an analytic tool for subsequent designs.




%------------------------------
\subsection*{DAR Corporation [CFD/FEA/Aeroelastic Analysis]}
%------------------------------

For a low-power ultra-endurance aircraft, aerodynamic efficiency and structural robustness are of paramount importance.  Unfortunately, they are also the most difficult tasks to model properly.  Realizing the full potential of the TURN methodology, requires slender wings with high aspect ratios, large tip masses to lower the system rotation rate, and unconventional control surface placement; thus, commonly used ``best-practices'' in aircraft design are not likely sufficient to accurately depict the flight performance.  As such, developing computational fluid dynamic models, performing finite element analysis, and then evaluating them against collected flight data from a physical system, is the primary goal of the next prototype installments.  This will identify discrepancies and ultimately validate the models, which are then used as a design tool to optimize the aerodynamic performance of the next TURN system.  DAR Corporation is a leading industry expert in small aircraft analysis and design, has a proven track record for aeroelastic research, and their experience with propeller and wind turbine systems is extremely well suited for the rotating TURN concept.




%------------------------------
\subsection*{StarLab [VICON Motion Capture Studio]}
%------------------------------

During Phase I SBIR research for the Air Force, two prototypes mitigated the largest risk: developing custom avionics hardware, and demonstrating a suitable flight controller for a TURN architecture.  Results from early flight tests indicated that standard attitude estimation algorithms cannot account for the rotating nature of the TURN system.  An initial modification was sufficient to accomplish the stated research objectives, but additional testing is needed to refine the algorithm, and complete the system identification testing.  Improving the attitude heading and reference system (AHRS) algorithm requires an external localization solution, independent of the on-board sensors, which can serve as ground truth.  StarLab is a VICON motion capture studio, which collects high-quality flight data, sampled at 500 Hz, with sub-millimeter resolution.  Furthermore, as an indoor testing facility, there is no external disturbance from wind, which can bias the results.  Once that data is collected, the on-board AHRS algorithm can be fine-tuned by filtering signal noise and adjusting gain settings, and parameter identification can validate the multibody dynamic simulation model.








%------------------------------------------------------------
\section*{Technical Discussion and R\&D Plan}

The goal of this proposed NSF SBIR project is to take the next step within the spiral development process which is developing and validating the analytic tools and models needed to design a TURN system with confidence.  Previous PhD research and Air Force SBIR funding have developed:
\begin{itemize}
\item Nonlinear multibody dynamic model of the TURN system;
\item Adaptive controls methodology for stabilization, inner-loop, and outer-loop components;
\item Custom avionics to facilitate the prototype development process;
\item Prototype testing platform to evaluate custom avionics; and
\item Modular prototype platform for multibody dynamic model validation.
\end{itemize}
The following section outlines the overall spiral development process, the role of motion capture and parameter identification routines applied to an existing airframe which will validate the multibody dynamic model, the need for CFD/FEA models, and how those results will contribute towards the next prototype installment.




%------------------------------
\subsection*{Spiral Development Plan}
%------------------------------

The goal is to develop the set of tools and models to design a TURN system with confidence.  Spiral development is an iterative approach, where prototypes address the largest risk, simulation models are refined with flight data, and enhanced analytic tools aid with subsequent vehicle designs.


%------------------------------
{\bf Custom Avionics Platform [Mark 0]:}
Before any type of TURN prototype system could be controlled, a custom designed flight control system (FCS) was needed.  This prototype, depicted in Figure \ref{Fig:Mark0}, was developed during the PhD research effort, and served as a dynamic test bench which: field tested hardware, debugged software, characterized sensor noise, implemented filter routines, evaluated new state estimation algorithms, and assessed controller gain selection.

\begin{figure}
\includegraphics 
[ width=\Wwhole, height=3.2in, clip, trim={0 0 0 0} ]
{../Figures/Mark0.jpg}
\caption{\emph{Mark 0} prototype for custom avionics development and evaluation.}
\label{Fig:Mark0}
\end{figure}


%------------------------------
{\bf Multibody Dynamics Validation [Mark 1]:}
The TURN vehicle is a multibody system, with a complex set of intertwined dynamics.  This prototype, shown in Figure \ref{Fig:Mark1}, was built during the Phase I SBIR Air Force research project, and established a testing platform to validate accuracy and identify limitations within the existing computer models and simulation routines.  Modularity was the emphasis of this prototype design, which will assess the system dynamics across several parameter variables and flight conditions.

\begin{figure}
\includegraphics 
[ width=\Wwhole, height=3.2in, clip, trim={0 0 0 0} ]
{../Figures/Mark1.jpg}
\caption{\emph{Mark 1} prototype for AHRS testing and multibody dynamic validation.}
\label{Fig:Mark1}
\end{figure}


%------------------------------
{\bf Aerodynamic Model Refinement [Mark 2]:}
Aerodynamic efficiency has the strongest influence on aircraft dynamics and performance, so accurate models must be developed as early as possible.  The goal of this NSF proposed research is to develop a CFD/FEA/aeroelastic models, perform optimizations and trade studies, and make predictions concerning the flight characteristics and endurance of the next prototype.  When the prototype is built with future funding, the intention is to run parameter identification on flight data, and then compare the analysis results against physical reality, which yields an analytic tool for subsequent designs.




%------------------------------
\subsection*{Multibody Dynamics Validation}
%------------------------------

A nonlinear multibody dynamic model and several adaptive control laws were devised from previous PhD research, and an avionics controller was developed during the Phase I effort.  The \emph{Mark 1} system has been built and performed initial flight testing, and is ready for more extensive VICON motion capture flight data.


%------------------------------
{\bf Objective:}
The goal is to validate the nonlinear multibody dynamic simulation model, so it can be used as a design tool for subsequent vehicle iterations.  The process uses the \emph{Mark 1} prototype, displayed in Figure \ref{Fig:Mark1}, collects flight data across a wide range of operating points, iterates through different physical parameters, and then compares recorded flight data to the results produced by the simulation model.  To accomplish this task, VICON motion capture provides external ground truth while recording system states, parameter identification yields the steady state and transient dynamics of the system, and a modular prototype framework is utilized to encompass a wide array of physical geometries.


%------------------------------
{\bf Motion Capture:}
The VICON motion capture studio is an indoor testing facility equipped with 44 infrared cameras that detect reflective markers placed on an object.  Position and orientation states are collected at each sample step, which are used to calculate all the linear velocities and angular rates, which completely describes the motion of each rigid body element.  Data is sampled at 500 Hz, with sub-millimeter resolution, which offers higher fidelity than the on-board sensors, and provides a localization solution within a GPS-denied environment.  Obtaining this high quality flight data is of paramount importance for implementing the parameter identification process.


%------------------------------
{\bf Parameter Identification Process:}
The simulation model must match physical reality in two different ways.  First, the trimmed steady state operating points are the set of states and control inputs which achieve a static equilibrium, such as hover or a constant rate of ascent.  Second, the transient dynamic response is described by the elements within the state space matrices, and are used for reference tracking and disturbance rejection.  Parameter identification algorithms run through the collected flight data and yield both the trimmed equilibrium operating points, and the state space dynamic model.  The state space form is populated with the partial derivatives of each state and control input with respect to the state derivatives, which mathematically describes the dynamic response of the system.  Similarly, the simulation model contains both trimming and linearization routines, which provide the state space representation that exactly matches the form provided by the parameter identification algorithms.  So, running multisine signal injection during the flight testing, and post processing the data via parameter identification, will directly convey any discrepancies between the physical flight characteristics and the simulated dynamics.  Validating the simulation across a wide range of operating conditions and physical parameters necessitates a modular prototype design.


%------------------------------
{\bf Elements of the Modular Design:}
The \emph{Mark 1} system was designed as a modular testing platform.  Several attributes within the prototype are adjustable, and intended to collect key parameters which are needed to validate the multibody dynamic simulation model.  A total of six wing blades were constructed, each with a 40'' wingspan.  Four of the blades have aspect ratios of 12, and will be used together while demonstrating a complete flight capability.  The other two blades have aspect ratios of 10 and 14, which will be utilized for model validation purposes.  Modularity within the system includes: aspect ratio, prop diameter, elevator control surface area, and tail boom length.  Additionally, the battery location is adjustable, such that the CG of the satellite aligns with the quarter-chord position of the wing.
\begin{itemize}
\item Two prop diameters will be tested, which will assess throttle commands versus power consumption, and identify the control bandwidth available to maintain equal spacing and coordinate the satellites.
\item There are three tail boom lengths to alter the moment arm distance between the elevator control surface and the CG of the satellite, which will evaluate the pitch dynamics within the simulation model.
\item Testing two sizes of elevator control surface, which dictates the force generated from elevator deflections, and will evaluate the engineering tradeoff between drag and control bandwidth.
\item Angle of attack will be evaluated at 3\degree, 5\degree, and 7\degree{} increments, which influences the required airspeed across the airfoil, and will help identify proper lift and drag coefficients across the operating region.
\item Each blade has the same wingspan, but aspect ratio is evaluated at 10, 12, and 14, which is synonymous with adjusting the wing surface area, thus directly influencing the angular rate and centrifugal force within the TURN system, which will review the rotational aspects within the simulation model.
\end{itemize}


%------------------------------
{\bf Additional Battery:}
Each satellite can be flown with one or two battery packs.  The goal of adding a second battery is not to increase the flight endurance, rather it increases the weight of each satellite.  Since weight is directly counteracted with lift, this modification directly influences the angular rate and centrifugal force on each tether arm.  Time permitting, this option would provide another dimension to evaluate multiple physical parameters (weight/lift, drag/thrust, control bandwidth, rotational influences) within the simulation model.  Since all of these parameters can be obtained directly from the previous adjustments, this testing would serve as an optional redundancy check within the model validation process.


%------------------------------
{\bf Testing Process:}
Altering propellers, tail boom lengths, and elevator surface areas are all easily adjustable, so testing each set of these parameters could be accomplished within a single day.  Then, proceeding through three different aspect ratios and three angle of attack settings, would require nine full days of VICON testing.  Repeating the full set of testing with a second battery configuration would double the testing duration to 18 days.  If needed, a smaller subset of testing with a second battery, may focus on operational areas that are most meaningful to the simulation validation.  To be safe, 20 days of VICON testing are allocated within the NSF SBIR Phase I budget proposal.


%------------------------------
{\bf Anticipated Outcomes:}
Any simulation model can be manipulated to fit a single operating condition, so evaluating a full range of physical parameters at different operating points will improve the level of confidence within this tool.  This activity yields elements within the state space matrices which represent the partial derivatives of the states and control inputs.  These values are needed to refine the accuracy of the dynamic models, and identify boundaries and limitations that need to be observed during the design process.  Since, the multibody simulation model is the primary means of sizing TURN components and evaluating controller gains, validating this tool is a crucial step prior to proceeding onto further prototype development.




%------------------------------
\subsection*{Aeroelastic Model Development}
%------------------------------

Having validated the multibody dynamic model, the next most important design tool is the aeroelastic analysis.  A computational fluid dynamic (CFD) model provides aerodynamic forces and moments, and a finite element analysis (FEA) provides insight into structural and bending properties.  Together they form a complete aeroelastic analysis which can be used during the custom airfoil design process.


%------------------------------
{\bf Objective:}
While the existing prototype focuses on validating the multibody dynamic simulation model, this activity develops the remaining tools needed to completely design a TURN system.  The primary purpose is to formulate the aerodynamic and structural models, which form a complete aeroelastic representation of the vehicle, and will be used to design a custom airfoil to fully leverage centrifugal stiffening within the TURN concept design.  Similar to the previous activity, the CFD/FEA/aeroelastic analysis will provide insight into larger TURN designs.  However the accuracy of such an analysis, specifically addressing the rotating aspect of the system, has not yet been assessed.  These tools will facilitate the next prototype design, and after construction in future research, the models will be validated through flight testing similar to the \emph{Mark 1}.


%------------------------------
{\bf Computational Fluid Dynamic Models:}
The goal is to develop an analytic tool that depicts accurate force and moment values acting at the CG of each rigid body element, while operating at different flight conditions.  A complete TURN system and its rotation will be depicted within the simulation model.  An initial baseline configuration will obtain states and control inputs that achieve steady-state equilibrium, which is synonymous with the hover operating condition.  Then, a number of parameters will be independently adjusted away from the trim condition, which yield a new set of forces and moments.  Because these points are no longer trimmed values, the forces and moments will influence the dynamics of the system and alter its operating state.  Adjustable parameters under investigation include: throttle input, elevator input, airfoil angle of attack, and airflow velocity over the wing.  For each of the four parameters, two points above and two points below the trimmed operating point will be selected.  A complete set of forces and moments at various operating conditions provides an array of values that can be matched against the physical flight data.


%------------------------------
{\bf Finite Element Analysis:}
Following the CFD task, DAR will develop a Finite Element Analysis (FEA) model, to assess static and dynamic structural loads.  The culmination of these two models will provide a full aeroelastic model analysis.  With larger scale vehicles utilizing higher aspect ratio wings, understanding the structural and aeroelastic properties is of utmost importance.  DAR will review: flutter, which is a type of dynamic instability undergoing simple harmonic motion; divergence, where a deflected load increases the magnitude of the deflection; and control surface reversal, where inputs have the opposite expected response.  Finally, a refined propeller model utilizing blade element theory, will better assess prop wash interactions on the fuselage, stabilizers and control surfaces.


%------------------------------
{\bf Custom Airfoil Design:}
Up to this point, each prototype utilized an existing airfoil profile with known characteristics.  This accelerated the development process, and reduced program risk by utilizing components with familiar capabilities.  While these serve their primary purpose, they do not approach the full potential that centrifugal stiffening offers from the TURN architecture.  DAR Corp will be tasked with developing a custom airfoil profile for the TURN system.  Their CFD and FEA models will provide a baseline for the dimensions and characteristics desired within the custom airfoil.  Given these inputs, they will develop a tailored pressure distribution to maximize lift and minimize drag, while exploring geometries (thin thickness, high camber) that are not typically appropriate for standard traditional fixed-wing aircraft.


%------------------------------
{\bf Model Validation:}
Building the next prototype embodiment is reserved for future work.  But once that effort is underway, identical to the \emph{Mark 1} prototype, parameter identification algorithms will yield the state space representation of the system.  As before, the matrix elements represent the partial derivatives of the system states, and prescribe the mathematical relationships between control inputs and system outputs.  However, unlike the multibody validation, where this relationship was directly obtained from the linearization routine within the simulation, the CFD validation requires one intermediate step: identifying the forces and moments acting on each of the TURN components.  The CFD analysis produces forces and moments for a given set of system states and control inputs.  This induces translation and rotation in a known manner because it must adhere to the laws of physics.  Therefore, with known mass and moment inertia parameters, the time-varying force and moment values can be derived directly from the system states and control inputs via the recorded flight data.  The set of forces and moments that arise during flight testing will be compared to the CFD results, which will identify and assess any major discrepancies, fine tune the parameter settings, and validate the CFD modeling process.


%------------------------------
{\bf Anticipated Outcomes:}
The outcome of this effort will yield all the tools needed to successfully design a TURN system.  The next installment in the spiral development process will use these tools to build the next largest prototype, where flight testing will validate the models developed within this current research activity.  These tools will help predict performance and flight endurance, responsiveness and handling characteristics of the system, illuminate any required revisions within subsequent designs, and they will provide the foundation for performing design trade studies and engineering optimizations.




%------------------------------
\subsection*{Design Trade Studies and Engineering Optimizations}
%------------------------------

Following the CFD/FEA/aeroelastic modeling, new tools will be available to further evaluate the TURN concept vehicle through design trade studies and engineering optimizations.


%------------------------------
{\bf Objective:}
Having validated the primary set of tools needed to design a TURN aircraft, the next task utilizes those tools and investigates how to maximize the performance of subsequent TURN prototypes.  Some design parameters are in conflict with one another, so trade studies will showcase the relative cost/benefit of each embodiment.  Similarly, there are many potential design optimizations within the TURN architecture, which can now be more formally investigated via the CFD analysis.


%------------------------------
{\bf Design Trade Studies:}
The following analyses will evaluate the cost/benefit associated with each feature.

\emph{Aspect Ratio:}
Lift induced drag is heavily influenced by the aspect ratio of the wing.  While, a high aspect ratio offers a definitive improvement in aerodynamic performance, slender wings also introduce flexibility within the structure.  Centrifugal stiffening within the TURN system offers a decided advantage, but a formal investigation is still needed to identify the diminishing returns on aerodynamic performance.

\emph{Winglets:}
These improve aerodynamic performance by decreasing detrimental wingtip vortices, but they add weight to the system and have a more pronounced influence on low aspect ratio wings.  This analysis will assess whether any reduction in drag is worth the added weight penalty.

\emph{Washout:}
This feature varies the incidence of the airfoil across the wingspan.  In general, it can decrease aerodynamic efficiency because the entirety of the wing no longer operates at an ideal angle-of-attack.  However, it offers two potential benefits.  First, it can be used to attain an elliptic load distribution, which increases the Oswald efficiency factor, and reduces induced drag from wingtip vortices.  Second, different portions of the wing stall at different times, which adds robustness into the flight envelope.  These benefits will be weighed against any degradation in aerodynamic efficiency.


%------------------------------
{\bf Engineering Optimizations:}
The following optimizations will identify ideal values for each design feature.

\emph{Outboard Horizontal Stabilizer (OHS):}
Research shows that a horizontal stabilizer properly placed within the wingtip vortex can reduce drag and contribute some amount of lift.  Appropriate sizing and placement is highly dependent on the system, so this task will find an ideal layout to capture the maximum benefit.

\emph{Dihedral:}
Centrifugal stiffening will greatly mitigate bending moments within the wing, but some may still be present.  Introducing a curvature with some dihedral can offset any remaining bending, which will help keep the wing horizontal during operation.

\emph{Asymmetric Span Profile:}
An asymmetric nonuniform profile distribution is a unique feature available to a TURN system.  From the system rotation, the outboard wingtip travels faster than the inboard section.  An asymmetric wingspan can exploit this feature, such that each segment is optimized for its intended airspeed.


%------------------------------
{\bf Anticipated Outcomes:}
These analyses will provide valuable insight for subsequent TURN prototypes, help increase aerodynamic performance, improve flight handling characteristics, and save time and money by seeking optimized TURN designs from an early stage of development.








%------------------------------------------------------------
\newpage \cfoot{ }
{\color{red} \bf \Huge References Cited}
%------------------------------------------------------------

%%------------------------------------------------------------
%\newpage
%\begin{thebibliography}{99}
%%------------------------------------------------------------
%
%
%\PubSpace
%\bibitem{Keshmiri}
%Keshmiri, M., A. K. Misra, and V. J. Modi. ``General Formulation for N-body Tethered Satellite System Dynamics.'' \emph{Journal of Guidance, Control, and Dynamics}; Vol. 19, No. 1, pp. 75-83; 1996. \url{http://arc.aiaa.org/doi/10.2514/3.21582}
%
%
%\PubSpace
%\bibitem{Mattos}
%Mattos, Bento Silva de, Ney R. Secco, Eduardo F. Salles. ``Optimal Design of a High-Altitude Solar-Powered Unmanned Airplane.'' \emph{Journal of Aerospace Technology Management}; Vol. 5, No. 3, pp. 349-361; 2013. \url{http://www.jatm.com.br/ojs/index.php/jatm/article/view/223}
%
%
%\PubSpace
%\bibitem{Nickol}
%Nickol, Craig L., Mark D. Guynn, Lisa L. Kohout, and Thomas A. Ozoroski. ``High Altitude Long Endurance Air Vehicle Analysis of Alternatives and Technology Requirements Development.'' \emph{Aerospace Sciences Meeting}; Reno, NV; Jan 2007. \url{http://arc.aiaa.org/doi/abs/10.2514/6.2007-1050}
%
%
%\PubSpace
%\bibitem{Pizarro-Chong}
%Pizarro-Chong, Ary, and Arun K. Misra.  ``Dynamics of Multi-Tethered Satellite Formations Containing a Parent Body.'' \emph{Acta Astronautica}; Vol. 63, pp. 1188-1202; 2008. \url{http://www.sciencedirect.com/science/article/pii/S0094576508002452}
%
%
%\PubSpace
%\bibitem{Selfridge_TURN}
%Selfridge, Justin M. \emph{Achieving Eternal Flight With a Tethered Uni-Rotor Network (TURN) Aircraft: A Complete Development of the Nonlinear Dynamic Model and Controller Architecture}. Doctorate of Philosophy dissertation. University of Virginia; Charlottesville, VA; May 2017 [pending].
%
%
%\PubSpace
%\bibitem{Selfridge_OF_IEEE}
%Selfridge, Justin M., and Gang Tao. ``Multivariable Output Feedback MRAC for a Quadrotor UAV.'' \emph{IEEE American Control Conference (ACC)}; Boston, MA; Jul 2016.  \url{http://ieeexplore.ieee.org/document/7524962/}
%
%
%\PubSpace
%\bibitem{Selfridge_Plant}
%Selfridge, Justin M., and Gang Tao. ``Centrifugally Stiffened Rotor: A Complete Derivation of the Plant Model with Nonlinear Dynamics.'' \emph{AIAA Aviation Technology, Integration, and Operations Conference}; Dallas, TX; Jun 2015.  \url{http://arc.aiaa.org/doi/abs/10.2514/6.2015-2735}
%
%
%\PubSpace
%\bibitem{Selfridge_Inner}
%Selfridge, Justin M., and Gang Tao. ``Centrifugally Stiffened Rotor: A Complete Derivation and Simulation of the Inner Loop Controller.'' \emph{AIAA Guidance Navigation and Control (GNC) Conference}; Kissimmee, FL; Jan 2015.  \url{http://arc.aiaa.org/doi/abs/10.2514/6.2015-0073}
%
%
%\PubSpace
%\bibitem{Selfridge_SF_SSCE}
%Selfridge, Justin M., and Gang Tao. ``A Multivariable Adaptive Controller for a Quadrotor with Guaranteed Matching Conditions.'' \emph{System Science and Control Engineering}; Vol. 2, No. 1, pp. 24-33; 2014. \url{http://www.tandfonline.com/doi/pdf/10.1080/21642583.2013.879050}
%
%
%\PubSpace
%\bibitem{Selfridge_SF_IEEE}
%Selfridge, Justin M., and Gang Tao. ``A Multivariable Adaptive Controller for A Quadrotor with Guaranteed Matching Conditions.'' \emph{IEEE American Control Conference (ACC)}; Portland, OR; Jun 2014.  \url{http://ieeexplore.ieee.org/document/6859355/}
%
%
%\PubSpace
%\bibitem{Selfridge_Local_GNC}
%Selfridge, Justin M. ``An On-Board Camera-Based Localization Solution.'' \emph{AIAA Guidance Navigation and Control (GNC) Conference}; Minneapolis, MN; Aug 2012. \url{http://arc.aiaa.org/doi/abs/10.2514/6.2012-4847}
%
%
%\PubSpace
%\bibitem{Selfridge_Local_AUVSI}
%Selfridge, Justin M. ``An On-Board Camera-Based Localization Solution.'' \emph{AUVSI Unmanned Systems Conference}; Las Vegas, NV; Aug 2012.  \url{https://www.researchgate.net/publication/268558033_An_On-Board_Camera-Based_Localization_Solution}
%
%
%\PubSpace
%\bibitem{Selfridge_Master}
%Selfridge, Justin M. \emph{A Complete Autonomous Vehicle Solution: An ASV Case Study}. Master's thesis. Old Dominion University; Norfolk, VA; May 2012.  \url{https://books.google.com/books/about/A_Complete_Autonomous_Vehicle_Solution.html?id=31qEMwEACAAJ}
%
%
%\PubSpace
%\bibitem{Stoneking}
%Stoneking, Eric. ``Newton-Euler Dynamic Equations of Motion for a Multi-Body Spacecraft.'' \emph{AIAA Guidance, Navigation and Control Conference and Exhibit}; Hilton Head, SC; Aug 2007. \url{http://arc.aiaa.org/doi/abs/10.2514/6.2007-6441}
%
%
%\PubSpace
%\bibitem{Tealgroup2015}
%Teal Group. ``Teal Group Predicts Worldwide UAV Production Will Total \$93 Billion in Its 2015 UAV Market Profile and Forecast.''  Website; Aug 14, 2015. \url{http://www.prnewswire.com/news-releases/teal-group-predicts-worldwide-uav-production-will-total-93-billion-in-its-2015-uav-market-profile-and-forecast-300128745.html}
%
%
%\PubSpace
%\bibitem{Tealgroup2016}
%Teal Group. ``Teal Group Predicts Worldwide Civil UAS Production Will Total \$65 Billion in Its 2016 UAS Market Profile and Forecast.'' Website; Jul 7, 2016. \url{http://www.prnewswire.com/news-releases/teal-group-predicts-worldwide-civil-uas-production-will-total-65-billion-in-its-2016-uas-market-profile-and-forecast-300295255.html}
%
%
%\PubSpace
%\bibitem{Xian-Zhong}
%Xian-Zhong, Gao, Hou Zhong-Xi, Guo Zheng, Zhu Xiong-Feng, Liu Jian-Xia, and Chen Xiao-Qian. ``Parameter Determination for Concept Design of Solar-Powered, High-Altitude Long-Endurance UAV.'' \emph{Aircraft Engineering and Aerospace Technology}; Vol. 85, No. 4, pp. 293-303; 2013. \url{http://www.emeraldinsight.com/doi/abs/10.1108/AEAT-Jan-2012-0011}
%
%
%\PubSpace
%\bibitem{Zhao}
%Zhao, Zhenjun, and Gexue Ren. ``Multibody Dynamic Approach of Flight Dynamics and Nonlinear Aeroelasticity of Flexible Aircraft.'' \emph{AIAA Journal}; Vol. 49, No. 1, pp. 41-54; 2011. \url{http://arc.aiaa.org/doi/abs/10.2514/1.45334?journalCode=aiaaj}
%
%
%\PubSpace
%\bibitem{Xiongfeng}
%Zhu Xiongfeng, Guo Zheng, Fan Rongfei, Hou Zhongxi, and Gao Xianzhong. ``How High Can Solar-Powered Airplanes Fly.'' \emph{Journal of Aircraft}; Vol. 51, No. 5, pp. 1653-1659; 2014. \url{http://arc.aiaa.org/doi/abs/10.2514/1.C032333}
%
%
%\end{thebibliography}

Include Patent Application Number.








%------------------------------------------------------------
\newpage
{\bf \Huge Biographical Sketches}
%------------------------------------------------------------


%------------------------------
\vspace{0.2cm}
{\bf \Large Justin M Selfridge, PhD (Principal Investigator)}

United States Citizen \\
PhD in Electrical and Computer 
Engineering - University of Virginia (UVA) \\
Gradient Consulting, LLC - Owner, Managing Member \\
NASA Langley - Independent Contractor


%------------------------------
{\bf Biography} \\
Dr Justin M Selfridge completed his PhD in Electrical Engineering at the University of Virginia.  Research areas include: multivariable adaptive control, system identification, autonomous vehicles, custom avionics hardware development, and UAV design and prototype fabrication.  For six years, Dr Selfridge worked within the Dynamic Systems and Controls Branch at NASA Langley.  His most recent NASA project developed the controller for the \emph{GL-10} concept vehicle; a ten rotor, tilt wing/tail, transformational flight vehicle, with VTOL capabilities and an extended flight endurance.  His PhD developed the nonlinear dynamic model, adaptive controller methodology, and numerical multibody simulation tools, for the TURN concept system; a novel aircraft seeking persistently enduring flight.  The past year he has been pursuing the TURN vehicle development full-time, working under his consulting company, Gradient Consulting, LLC.


%------------------------------
{\bf Education} \\
\begin{tabular}{llll}
\vspace{-0.15in}
\hspace{2.2in} &
\hspace{1.3in} &
\hspace{1.3in} &
\hspace{0.0in} \\
University of Virginia &
Charlottesville, VA &
Electrical Engr &
Ph.D., 2017 \\
Old Dominion University &
Norfolk, VA &
Mechanical Engr &
MS, 2012 \\
University of Virginia &
Charlottesville, VA &
Mechanical Engr &
BS, 2004
\end{tabular}


%------------------------------
{\bf Research Experience} \\
\begin{tabular}{llll}
\vspace{-0.15in}
\hspace{2.2in} &
\hspace{1.3in} &
\hspace{1.3in} &
\hspace{0.0in} \\
Air Force Phase I SBIR &
Newport News, VA &
Principle Investigator &
2017-present \\
NASA Langley Research Center &
Hampton, VA &
Contractor &
2013-2017 \\
NASA Langley Research Center &
Hampton, VA &
Work Study &
2011-2013 \\
University of Virginia &
Charlottesville, VA &
Teach/Research Asst &
2012-2014 \\
Old Dominion University &
Norfolk, VA &
Teach/Research Asst &
2010-2012
\end{tabular}


%------------------------------
{\bf Authored Publications} \\
\vspace{-0.2in}

\PubSpace
Selfridge, Justin M.
\emph{Achieving Eternal Flight With a Tethered Uni-Rotor Network (TURN) Aircraft: A Complete Development of the Nonlinear Dynamic Model and Controller Architecture}.
Doctorate of Philosophy dissertation.
University of Virginia; Charlottesville, VA;
May 2017.

\PubSpace
Selfridge, Justin M., and Gang Tao.
``Multivariable Output Feedback MRAC for a Quadrotor UAV.''
\emph{IEEE American Control Conference (ACC).}
Boston, MA.
Jul 2016.

\PubSpace
Selfridge, Justin M., and Gang Tao.
``Centrifugally Stiffened Rotor: A Complete Derivation of the Plant Model with Nonlinear Dynamics.''
\emph{AIAA Aviation Technology, Integration, and Operations Conference.}
Dallas, TX.
Jun 2015.

\PubSpace
Selfridge, Justin M., and Gang Tao.
``Centrifugally Stiffened Rotor: A Complete Derivation and Simulation of the Inner Loop Controller.''
\emph{AIAA Guidance Navigation and Control (GNC) Conference.}
Kissimmee, FL.
Jan 2015.

\PubSpace
Selfridge, Justin M., and Gang Tao.
``A Multivariable Adaptive Controller for A Quadrotor with Guaranteed Matching Conditions.''
\emph{IEEE American Control Conference (ACC).}
Portland, OR.
Jun 2014.

\PubSpace
Selfridge, Justin M.
``A Multivariable Adaptive Controller for a Quadrotor with Guaranteed Matching Conditions.''
\emph{System Science and Control Engineering.}
Vol 2, No 1, pp 24-33; 2014.

\PubSpace
Selfridge, Justin M.
\emph{A Complete Autonomous Vehicle Solution: An ASV Case Study}.
Master's thesis.
Old Dominion University; Norfolk, VA.
May 2012.








%------------------------------------------------------------
\newpage
{\bf \Huge Budget Justification}
%------------------------------------------------------------


{\bf Senior Personnel:} Payscale.com indicates early career PhD income as \$103k/yr.  Dr Selfridge collects \$100k/yr.  Working full time on a 6.0 month project yields \$50k for the proposed budget.


{\bf Other Personnel:} None.


{\bf Fringe Benefits:} \\
\begin{tabular}{lll}
Vacation:     &  5 work days/yr         & = 2\%  \\
Sick Pay:     &  5 work days/yr         & = 2\%  \\
Retirement:   & (\$4k/yr) / (\$100k/yr) & = 4\%  \\
Insurance:    & (\$8k/yr) / (\$100k/yr) & = 8\%  \\
\end{tabular}  \\
Total fringe benefit percentage is 16\% of direct labor cost, which equates to \$8k for the proposed budget.


{\bf Equipment:} None.


{\bf Travel Costs:} An amount of \$2k is allocated for Dr Selfridge to attend the ``Grantee Workshop.''  The ``Beat-the-Odds Boot Camp'' is accounted for within the ``Other Direct Costs'' section.  No other travel arrangements, foreign or domestic, are budgeted.


{\bf Participant Support Costs:} None.


{\bf Materials and Supplies:} None.


{\bf Publication/Documentation Costs:} None.


{\bf Consultant Services:} None.


{\bf Computer Services:} None.


{\bf Subawards:}
\begin{itemize}
\item StarLab (VICON Motion Capture Studio): 20 days of flight testing at \$1000/day equates to \$20k allocated within the budget.
\item DAR Corporation (CFD Development): Aerodynamic Analysis was quoted at \$68,800.  Performing trade studies and optimizations was quoted at \$60,320.
\end{itemize}


{\bf Other Direct Costs:}
\begin{itemize}
\item Accounting: An amount of \$5k is allocated to hire a CPA accountant to prepare financial statements and perform financial viability assessment, and purchase project cost accounting system software.
\item An amount of \$10k is allocated to cover the costs of Dr Selfridge participating within the NSF's ``Beat-the-Odds Boot Camp'' to facilitate the customer discovery process.
\end{itemize}


{\bf Indirect Costs:} Indirect costs are waived for this project, to fully fund the VICON and CFD research efforts.


{\bf Business Fee:} The total direct and indirect costs sum to \$224,120.  A business fee of \$880 attains the maximum allotted amount of \$225,000, and represents 0.4\% of the total research program cost.


{\bf Resumes for External Personnel:} The following pages contain resumes for Dr Willem AJ Annemat (DAR Corporation) and Dr James E Hubbard (StarLab).




%------------------------------
\newpage
{\bf \Large Willem AJ Anemaat, PhD}

United States Citizen \\
PhD in Aerospace Engineering - University of Kansas (KU) \\
DAR Corporation - Founder, Owner


%------------------------------
{\bf Biography} \\
Dr Willem AJ Anemaat is frequently a session chair for SAE, AIAA and ICAS conferences, and is a current member of the SAE WATC (Wichita Aviation Technology Conference and Exposition) Committee, where he served as the program technical chair in 2008.  Dr Anemaat has been a regular reviewer of airplane design related articles for the AIAA Journal of Aircraft and is a judge for AIAA Design competitions.  He was member of the AIAA Journal of Aircraft Editorial Advisory Board from 2004 until 2009, and then graduated to became an Associate Editor for the AIAA Journal of Aircraft dealing with Aircraft Design topics.  He is the Chair of the AIAA Aircraft Design Technical Committee, and is an AIAA Associate Fellow.  Dr Anemaat became member of the Kansas UAV Consortium Executive Board in 2005, is the recipient of the SAE 2010 Forest R. McFarland Award, was inducted into the University of Kansas Aerospace Engineering Honor Roll in 2017, and is a member of the KUAE Advisory Board, having served as Chair from 2016-2017.


%------------------------------
{\bf Authored Publications} \\
\vspace{-0.2in}

\PubSpace
Anemaat, Willem, D. van Dommelen, S. Johnson, P.B. Sargent, and W. Liu.
``Comparison of Aerodynamic Analysis Tools for Rotorcraft in Hover.''
\emph{AIAA Aerospace Sciences Meeting.}
Kissimmee, FL.
Jan, 2018.

\PubSpace
Anemaat, Willem, M. Schuurman, and W. Liu.
``Aerodynamic Design, Analysis and Testing of Propellers for Small Unmanned Aerial Vehicle.''
\emph{AIAA Aerospace Sciences Meeting.}
Grapevine, TX.
Jan, 2017.

\PubSpace
Anemaat, Willem.
``Smart Aircraft Modeler for Aircraft Conceptual Design.''
\emph{Collaboration in Aircraft Design (SCAD), Council of European Aerospace Societies (CEAS), Technical Committee Aircraft Design (TCAD), Research Section.}
Naples, Italy.
Oct, 2015.

\PubSpace
Anemaat, Willem, B. Kaushik, J. Carroll, and J. Jeffery.
``Software Tool Development to Improve the Airplane Preliminary Design Process.''
\emph{ISPE International Conference on Concurrent Engineering.}
Melbourne, Australia.
Sept, 2013.

\PubSpace
Anemaat, Willem, W. Liu, B. Kaushik, M. Yang, M. Brown, and R. Hale.
``Design, Build and Fly: NASA Lockheed P-3 Orion with External Antenna Fairings.''
\emph{AIAA Aerospace Sciences Meeting.}
Grapevine, TX.
Jan 2013.

\PubSpace
Anemaat, Willem, R.D. Hale, and N. Ramabadran.
``AAARaven: Knowledge-Based Aircraft Conceptual and Preliminary Design.''
\emph{AIAA/ASME/ASCE/AHS/ASC Structures, Structural Dynamics, and Materials Conference.}
Honolulu, HI.
Apr 2007.

\PubSpace
Anemaat, Willem, and B. Kaushik.
``Geometry Design Assistant for Airplane Preliminary Design.''
\emph{AIAA Aerospace Sciences Meeting.}
Orlando, FL.
Jan 2001.




%------------------------------
\newpage
{\bf \Large James E Hubbard, Jr, PhD}

United States Citizen \\
PhD in Mechanical Engineering - Massachusetts Institute of Technology (MIT)  \\
Langley Distinguished Professor - University of Maryland (UM)  \\
Samuel P Langley Professor - National Institute of Aerospace (NIA) \\
StarLab VICON Motion Capture Studio - Founder


%------------------------------
{\bf Biography} \\
Dr James E Hubbard's career has spanned some 20 years and includes work in aero-acoustics for noise-control, adaptive structures, spatially distributed transducers and the extension of modern time domain control methodologies into the spatial domain for the real-time control of distributed systems.  His work has resulted in a dozen patents which have demonstrated the efficacy and practicality of the techniques that he and his students have developed over these years.  These techniques have been viewed as innovative and revolutionary by his colleagues, and to recognize of these accomplishments, the University of Maryland named Dr Hubbard the University of Maryland Langley Professor at the National Institute of Aerospace.


%------------------------------
{\bf Honors and Awards} \\
\vspace{-0.2in}

\PubSpace
2016 - Inducted into the National Academy of Engineering for advances in the modeling, design, analyses, and application of adaptive structures.

\PubSpace
2016 - Inducted into the Virginia Academy of Engineering, Science and Medicine.

\PubSpace
2016 - Society of Photonics and Instrumentation Engineers (SPIE) Smart Structures and Materials (SSM) Lifetime Achievement Award for those viewed as luminaries in the fields of SSM.

\PubSpace
2015 - Best Paper in Structures Award from the Adaptive Structures and Material Systems (ASMS) branch of the Aerospace Division of the American Society of Mechanical Engineers (ASME).

\PubSpace
2015 - Elected American Society of Mechanical Engineers (ASME) Fellow having made noteworthy invention, discovery or advancement in the state of the art as evidenced by publication of widely accepted materials, by receipt of major patents, or by having products or processes in the marketplace.

\PubSpace
2012 - Elected American Institute of Aeronautics and Astronautics (AIAA) Fellow, where this distinction is conferred upon those members of the Institute who have made notable and valuable contributions to the arts, sciences, or technology of aeronautics and astronautics.

\PubSpace
2011 - The International Society for Optical Engineering (SPIE) conferred the grade of Senior Member of the Society ``in recognition of significant achievements within the optics and photonics community.''

\PubSpace
2002 - In July was awarded ``The Mass High Tech'' Movers and Innovators award ``Recognized for shaping the rich entrepreneurial community of New England.''

\PubSpace
2002 - Recipient of the Black Engineer of the Year President's Award.  Recognizes high individual merits that have a broad effect on people in many disciplines, and represent high value to the society as a whole.  Goes to an individual of high personal achievements who has made a major impact on a company's products and profits and who has broad managerial reach.








%------------------------------------------------------------
\newpage
{\bf \Huge Current and Pending Support \\}
%------------------------------------------------------------


{\bf Current Support:}  None.  Air Force Phase I SBIR research just finished, no currently funded projects.


{\bf Pending Support:}  There are three pending proposal submissions.  None have been awarded; they are all awaiting selection notification.

TURN Combustion Engine Embodiment  \\
\begin{tabular}{ll}
Organization: & Air Force        \\
Type:         & SBIR Phase II    \\
Amount:       & \$1,500,000      \\
Duration:     & 12 month         \\
Submitted:    & May 25th, 2018   \\
Period:       & 08/18 - 07/19    \\
\end{tabular}

TURN Persistently Enduring Flight  \\
\begin{tabular}{ll}
Organization: & DARPA            \\
Type:         & TTO BAA          \\
Amount:       & \$999,855        \\
Duration:     & 12 month         \\
Submitted:    & June 10th, 2018  \\
Period:       & 11/18 - 10/19    \\
\end{tabular}

TURN Small Mobile Remote Deployment  \\
\begin{tabular}{ll}
Organization: & NSF              \\
Type:         & SBIR Phase I     \\
Amount:       & \$225,000        \\
Duration:     & 6 month          \\
Submitted:    & July 10th, 2018  \\
Period:       & 02/19 - 08/19    \\
\end{tabular}


{\bf Upcoming Submissions:}  None. No planned upcoming submissions.








%------------------------------------------------------------
\newpage
{\bf \Huge Collaborators and Other Affiliations}
%------------------------------------------------------------

Not applicable; Dr Justin M Selfridge works exclusively for Gradient Consulting, LLC.  The only other past affiliation was with NASA Langley, where he worked within the Dynamic Systems and Control Branch between 2011-2017.








%------------------------------------------------------------
\newpage
{\bf \Huge Facilities and Equipment}
%------------------------------------------------------------

The primary goal of this Phase I research is to take the existing \emph{Mark 1} TURN prototype, collect high fidelity flight data within a VICON motion capture studio, validate the multibody dynamic models through parameter identification, and assess the aerodynamic performance through a CFD analysis.  As such, very little is needed in the form of facilities and equipment.


%------------------------------
{\bf Hardware for Avionics Development:}
All the tools and equipment needed to build custom avionic hardware and concept prototype aircraft are available to Gradient Consulting.  This includes a prototyping workshop that has previously constructed numerous unconventional multirotor and fixed-wing platforms, and developed multiple custom fabricated avionics boards.  Equipment commonly utilized includes: soldering station, power supply, oscilloscopes, prototype boards, motors, props, servos, electronic components, and model aircraft building supplies.  More specialized equipment includes: a 3D printer, RF transceivers, dedicated GCS laptop computers, and RC radios.  Additionally, the PI is registered with the FAA and holds a valid Part 107 Remote Pilot Certification.


%------------------------------
{\bf Software for Analysis and Design:}
DAR Corp routinely employs Star-CCM+ and FlightStream software for CFD, FEA, and aeroelastic modeling an analysis.  Gradient relies heavily on the Matlab software package, for vehicle modeling and simulation, controls development, post processing flight data, and running parameter identification.  Finally, Gradient will use SolidWorks CAD modeling software to coordinate the modeling activities with DAR Corp.


%------------------------------
{\bf Motion Capture Studio:}
StarLab is a VICON motion capture studio, which is an indoor testing facility equipped with 44 infrared cameras that detect reflective markers placed on an object.  Position and orientation states are collected at each sample step, which are used to calculate all the linear velocities and angular rates, which completely describes the motion of each rigid body element.  Data is sampled at 500 Hz, with sub-millimeter resolution, which offers higher fidelity than the on-board sensors, and provides a localization solution within a GPS-denied environment.  Obtaining this high quality flight data is of paramount importance for implementing the parameter identification process.








%------------------------------------------------------------
\newpage
{\bf \Huge Data Management Plan (DMP)}
%------------------------------------------------------------

All data generated in this SBIR Phase I project is considered proprietary.








%------------------------------------------------------------
\newpage
{\bf \Huge Mentoring Plan}
%------------------------------------------------------------

Not applicable; no postdoctoral scholars are part of the proposed research.








%------------------------------------------------------------
\newpage
{\bf \Huge Supplementary Documents \\}
%------------------------------------------------------------

{\bf \Large Letters of Support}  \\
Letters of Support are appended following this section.  \\

{\bf \Large Company Commercialization History}  \\
Not applicable; no previous Phase II awards.  \\

{\bf \Large Human Subjects Documentation}  \\
Not applicable; no human subjects are part of the proposed research.  \\

{\bf \Large Vertebrate Animals Documentation}  \\
Not applicable; no vertebrate animals are part of the proposed research.  \\

{\bf \Large Resubmission Change Description}  \\
Not applicable; this is the first proposal for this research topic.  \\
















\end{document}
















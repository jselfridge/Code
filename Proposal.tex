

%============================================================
%  Justin M Selfridge, PhD
%  Gradient Consulting, LLC
%  NSF SBIR 2018 Phase I Proposal
%  Achieving Persistently Enduring Flight With A
%  Tethered Uni-Rotor Network (TURN) System
%============================================================


\documentclass[11pt]{article}


% Define usepackages
\usepackage{amsmath,amsfonts,amssymb}
\usepackage{caption}
\usepackage{enumitem}
\usepackage{fancyhdr}
\usepackage{floatrow}
\usepackage{graphicx}
\usepackage{lipsum}
\usepackage{textcomp,gensymb}
\usepackage{times}
\usepackage[table]{xcolor}


% Document margins
\usepackage[
  top    = 1.0in,
  bottom = 1.0in,
  left   = 1.0in,
  right  = 1.0in ]
  {geometry}


% Footer spacing
\setlength{\footskip}{20pt}


% Section title formatting
\makeatletter
\renewcommand\section{
\@startsection{section}{1}{\z@}%
{4.0ex \@plus 0.1ex \@minus 0.1ex}%
{1.0ex \@plus 0.1ex}%
{\normalfont\LARGE\bfseries}}
\makeatother


% Subsection title formatting
\makeatletter
\renewcommand\subsection{
\@startsection{subsection}{2}{\z@}%
{1.0ex \@plus 0.1ex \@minus 0.1ex}%
{0.01ex \@plus 0.01ex}%
{\normalfont\Large\bfseries}}
\makeatother


% Subsubsection title formatting
\makeatletter
\renewcommand\subsubsection{
\@startsection{subsubsection}{3}{\z@}%
{1.0ex \@plus 0.1ex \@minus 0.1ex}%
{0.0ex \@plus 0.1ex}%
{\normalfont\normalsize\bfseries}}
\makeatother


% Paragraph formatting
\setlength{\parindent}{0pt}
\setlength{\parskip}{6pt plus1pt minus1pt}
\setlist{nolistsep}


% Adjust TOC Indents
\makeatletter
\renewcommand*\l@section{\@dottedtocline{1}{1.5em}{2.3em}}
\renewcommand*\l@subsection{\@dottedtocline{2}{3.8em}{2.3em}}
\renewcommand*\l@subsubsection{\@dottedtocline{3}{6.0em}{2.9em}}
\makeatother


% TOC formatting
\makeatletter
\renewcommand\tableofcontents{%
\null\hfill
\textbf{\normalsize TABLE OF CONTENTS}
\hfill\null
\par
\hspace{0.55cm}
\textbf{\normalsize Section}
\hfill
\textbf{\normalsize Page}
\par
\@starttoc{toc}%
\newpage}
\makeatother


% Caption formatting
\floatsetup[table]{capposition=top}
\captionsetup{labelsep=period}
\captionsetup{labelfont=bf}
\captionsetup{font={small,bf}}


% Custom spacing commands
\newcommand{\PubSpace}{\vspace{-0.12cm}}  % Publication padding
\newcommand{\SecSpace}{\vspace{0.5cm}}  % Add padding to sections


% Common figure dimensions
\newcommand{\Wwhole}   {6.30in}
\newcommand{\Whalf}    {3.00in}
\newcommand{\Wthird}   {2.10in}
\newcommand{\Hwhole}   {8.60in}
\newcommand{\Hhalf}    {3.80in}


% Custom lineweights
\makeatletter
\newcommand{\thickhline}{%
\noalign {\ifnum 0=`}\fi \hrule height 1pt
\futurelet \reserved@a \@xhline }
\newcolumntype{[}{@{\vrule width 1pt\hspace{6pt}}} \newcolumntype{]}{@{\hspace{6pt}\vrule width 1pt}} \newcolumntype{!}{@{\hskip\tabcolsep\vrule width 1pt\hskip\tabcolsep}}
\makeatother


% Color specification
\definecolor{darkblue}{rgb}{0.2,0.2,0.4}


% Table cell background colors
\newcommand{\p}{\cellcolor{red!10}}
\newcommand{\y}{\cellcolor{yellow!10}}
\newcommand{\g}{\cellcolor{green!10}}


% Link customization
\PassOptionsToPackage{hyphens}{url}\usepackage{hyperref}
\hypersetup{colorlinks=true}
\hypersetup{citecolor=darkblue}
\hypersetup{linkcolor=darkblue}
\hypersetup{urlcolor=darkblue}


\usepackage{nameref}
\makeatletter
\newcommand*{\currentname}{\@currentlabelname}
\makeatother


% Header information
\pagestyle{fancy}
\fancyhf{}
\lhead{
  NSF Phase I SBIR 18-550 / Other Topics \\
  Gradient Consulting, LLC }
\rhead{
  Persistently Enduring Flight Utilizing A \\
  Tethered Uni-Rotor Network (TURN) System}
\cfoot{ }


%% Custom hyperlinked subtasks
%\makeatletter
%\let\oldhypertarget\hypertarget
%\renewcommand{\hypertarget}[2]{%
%\oldhypertarget{#1}{#2}%
%\protected@write\@mainaux{}{%
%\string\expandafter\string\gdef
%\string\csname\string\detokenize{#1}\string\endcsname{#2}%
%}%
%}
%\newcommand{\task}[1]{%
%\hyperlink{#1}{\textit{\csname #1\endcsname}}%
%}
%\makeatother








%------------------------------------------------------------
% PROPOSAL INSTRUCTIONS
%------------------------------------------------------------

% Phase I proposals may be submitted for up to $225,000 in R&D funding intended to support projects from 6-12 months in duration. Typically, small businesses will be notified of the award decision four to six months after the submission deadline.

% Phase I Proposal and Program Objectives. An SBIR Phase I proposal must describe the research effort needed to establish the feasibility of the proposed scientific or technical innovation. The primary objectives of the Phase I effort are to (i) determine whether the innovation has sufficient intellectual/technical and broader/commercial impact merit for proceeding into a Phase II project and (ii) to assess commercial feasibility of the proposed innovation. The deliverable of an SBIR Phase I grant is a report describing the technical accomplishments and outcomes of the Phase I project.

% The aim of a Phase I project should be to demonstrate technical feasibility of the proposed innovation and thereby bring the innovation closer to commercialization. Proposals should describe the development of an innovation that demonstrates the following characteristics:
% - Involves a high degree of technical risk – for example: Has never been attempted and/or successfully done before; Is still facing technical hurdles (that the NSF-funded R&D work is intended to overcome).
% - Has the potential for significant commercial impact and/or societal benefit, as evidenced by: Having the potential to disrupt the targeted market segment; Having good product-market fit (as validated by customers); Presenting barriers to entry for competition; Offering potential for societal benefit (through commercialization under a sustainable business model).

% Mandatory Sections:
% - Project Summary (1 pg max)
% - Project Description (15 pg max)
% - References Cited
% - Biographical Sketches
% - Budget and Subaward Budgets
% - Budget Justification
% - Current and Pending Support
% - Collaborators and Other Affiliations
% - Facilities, Equipment and Other Resources
% - Supplementary Documents (all that are applicable)







%------------------------------------------------------------
\begin{document}
%------------------------------------------------------------

\begin{titlepage}

\begin{center}
{$ $}  \\
\LARGE {\bf
Persistently Enduring Flight Utilizing A  \\
Tethered Uni-Rotor Network (TURN) System
}  \\
\vspace{0.5in}
\LARGE \emph{
NSF Phase I SBIR 18-550  \\
Other Topics  {$\quad$}  Proposal: ???
}  \\
\vspace{0.5in}
\Large {
Gradient Consulting, LLC  \\
Justin M Selfridge, PhD
}  \\
\vspace{0.75in}
\end{center}

\begin{figure} [h!]
\includegraphics
[ width=\Wwhole, height=3.2in, clip, trim={150 350 150 0} ]
{../Figures/TURN.jpg}
\end{figure}

\end{titlepage}








%------------------------------------------------------------
\newpage {$ $} \\ 
{\color{red} \bf \Huge Project Summary \\}  % 1 page max
%------------------------------------------------------------
%Information MUST be entered into all three text boxes, or the proposal will not be accepted. Do not upload your Project Summary as a PDF file.


\subsection*{\color{red} Overview, Key Words, and Subtopic Name}
%Describe the potential outcome(s) of the proposed activity in terms of a product, process, or service. Provide a list of key words or phrases that identify the areas of technical expertise to be invoked in reviewing the proposal; and the areas of application that are the initial target of the technology. Provide the subtopic name.

Insert.


\subsection*{\color{red} Intellectual Merit}
%This section MUST begin with "This Small Business Innovation Research Phase I project..." Address the intellectual merits of the proposed activity. Do not include proprietary information in the summary. Briefly describe the technical hurdle(s) that will be addressed by the proposed R&D (which should be crucial to successful commercialization of the innovation), the goals of the proposed R&D, and a high-level summary of the plan to reach those goals.

Insert.


\subsection*{\color{red} Broader/Commercial Impact}
%In this field, discuss the expected outcomes in terms of how the proposed project will bring the innovation closer to commercialization under a sustainable business model. In this box, also describe the potential commercial and market impacts that such a commercialization effort would have, if successful.  As appropriate, also discuss potential broader societal impacts of the innovation (e.g. educational, environmental, scientific, societal, or other impacts on the nation and the world).

Insert.








%------------------------------------------------------------
\newpage
\setcounter{page}{1}
\cfoot{\thepage}
{\bf \Huge Project Description}  % 15 page max
%------------------------------------------------------------
% The project description is the core of the proposal document, where you convince the SBIR Program Director and the expert reviewers that your proposed R&D project meets the NSF's criteria for intellectual merit and broader/commercial impact. Present evidence that the proposed technology is innovative, that development of it entails high technical risk, and that you have a credible plan to establish technical feasibility during Phase I. Convince the reviewers that the company and the project team have the necessary expertise, resources, and support to carry out the project, and that they are committed to building a viable business around the product/service being developed. Finally, present a compelling case that the project objectives will significantly advance the readiness of the technology and strengthen and validate its commercial position. The below format is strongly suggested for the Project Description.

Flight has always captured man's imagination, which is evidenced by the great variety of aerial vehicles that exist today.  Everything from fixed-wing to rotorcraft; satellites to spaceships; mono-wing to quadrotor.  However, despite the wide variety of existing aircraft, none have attained persistently enduring flight.  Accomplishing this aviation milestone is one of the great challenges still facing the aerospace community.




%------------------------------------------------------------
\section*{\color{red} Elevator Pitch}  % (no more than one page)
%The Customer. Describe the expected customer for the innovation. What customer needs or market pain points are you addressing?
%The Value Proposition. What are the benefits to the customer of your proposed innovation? What is the key differentiator of your company or technology? What is the potential societal value of your innovation?
%The Innovation: Succinctly describe your innovation. This section can contain proprietary information that could not be discussed in the Project Summary. What aspects are original, unusual, novel, disruptive, or transformative compared to the current state of the art?


%------------------------------
\subsection*{\color{red} The Customer}
%------------------------------

%------------------------------
\subsection*{\color{red} The Value Proposition}
%------------------------------

%------------------------------
\subsection*{\color{red} The Innovation}
%------------------------------








%------------------------------------------------------------
\newpage
\section*{The Commercial Opportunity}  % (recommended length: 2-4 pages)
% Is there a broader societal need you are trying to address with this commercial opportunity? Please describe.
% Describe the market and addressable market for the innovation. Discuss the business economics and market drivers in the target industry.
% How has the market opportunity been validated? Describe your customers and your basic business model.
% Describe the competition. How do you expect the competitive landscape may change by the time your product/service enters the market?
% What are the key risks in bringing your innovation to market?
% Describe your commercialization approach. Discuss the potential economic benefits associated with your innovation, and provide estimates of the revenue potential, detailing your underlying assumptions.
% Describe the resources you expect will be needed to implement your commercialization approach.
% Describe your plan and expected timeline to secure these resources.

Insert.




%------------------------------
\subsection*{Market Assessment}
%------------------------------

Unmanned aerial systems are the fastest growing segment within the aerospace industry.  Even so, the number of potential applications is still limited by some common technical challenges.


%------------------------------
{\bf Unmanned System Market Growth:}
According to the Teal Group's 2017 World Civil UAS Market Profile and Forecast, non-military UAS production will total \$73.5 billion in the next decade, soaring from \$2.8 billion worldwide in 2017 to \$11.8 billion in 2026, which is a 15.5\% compound annual growth rate in constant dollars.  Commercial use will be the fastest growing civil segment, rising more than twelve-fold from \$512 million in 2017 to \$6.5 billion.  As indicated in Figure \ref{Fig:Comm_Invest}, technology companies and venture capitalists have invested more than \$3 billion into the UAS sector since 2014.  Acquisition patterns follow investment patterns, and large companies are seeking UAS technologies.  Intel, Verizon, Facebook and Google have all made significant investments or acquisitions within their UAS portfolios.  Other technology leaders, including Amazon and General Electric, have focused on building up their own internal capabilities.

\begin{figure} [h!]
\includegraphics 
[ scale=0.92, clip, trim={0 0 0 0} ]
{../Figures/Comm_Invest.png}
\caption{Investments in UAS Companies Since 2014}
\label{Fig:Comm_Invest}
\end{figure}


%------------------------------
{\bf Technical Challenges:}
While the UAS market is expected to grow rapidly, their usefulness is largely predicated on flight endurance.  Flight endurance is the key parameter that dictates: range, time on station, size and weight of the vehicle and payload, coverage area, downtime for refueling, and number of vehicles required for a specified mission; just to name a few.  For comparable size/weight/cost, low-power and aerodynamically efficient UAVs are more desirable because they offer extended flight endurance.  Furthermore, for many potential applications, a VTOL capability is an absolute necessity.  Under these scenarios, multirotors are the only readily available option, but they can only operate for a small fraction of the time as a comparable fixed-wing system.  The TURN system addresses each of these challenges, offers a VTOL capability, and holds the promise of true ultra-endurance flight.


%------------------------------
{\bf Competitive Landscape:}
The UAV market is populated with many competitive equipment manufacturers.  As previously stated, they have been successful in attracting venture capital, so investment patterns are a reasonable way to identify potential competitors.  Figure \ref{Fig:Comm_Comp} presents the most significant competitors, and the total third-party investment made within each company.  While the space is densely populated, a TURN architecture offers significant advantages over existing solutions.  For small systems, multirotors typically offer 20-30 minute flight operations, while fixed-wing aircraft average around 90 minutes.  Medium size systems, primarily for ISR missions, typically support payloads of a few hundred pounds for several days at a time.  Finally, low-cost, high-altitude, ultra-endurance UAS for internet and communications, promise to create an entirely new segment of the market.  Airbus has already begun low rate production of solar-powered UAS, and both Google and Facebook have been developing their own technologies.  This analysis identified several important elements of a commercialization strategy, including: a broad range of possible customers in several different markets, some potential ``hot spot'' sectors that would benefit most from the unique attributes of a TURN system, and a focus on verticals that provide a large addressable market.

\begin{figure} [h!]
\includegraphics 
[ scale=0.8, clip, trim={0 0 0 0} ]
{../Figures/Comm_Comp.png}
\caption{Venture Capital Investments in Specific Drone Companies}
\label{Fig:Comm_Comp}
\end{figure}




%------------------------------
\subsection*{Target Customers}
%------------------------------

UAV platforms, and their potential use in commercial and military applications, have become nearly ubiquitous in the United States. Applications for UAV technologies are wide-ranging, from precision agriculture, inspection, infrastructure monitoring, crowd control, first-responder incident evaluation and management, to name a few.  As previously noted, the UAV market is both large and growing rapidly.


%------------------------------
{\bf Target Customers:}
Based on this analysis, the ideal target customers during first market entry are within the \emph{Infrastructure} and \emph{Agriculture} market segments.  Combined, those markets have potential for more than \$77 billion of product and service potential, as indicated in Figure \ref{Fig:Comm_Industry}.

\begin{figure}
\includegraphics 
[ scale=0.85, clip, trim={0 0 0 0} ]
{../Figures/Comm_Industry.png}
\caption{Market Opportunity by UAS Industry Sector}
\label{Fig:Comm_Industry}
\end{figure}


%------------------------------
{\bf Infrastructure:}
Infrastructure (including the construction sector) will lead the commercial market.  All ten of the largest worldwide construction firms are experimenting with unmanned systems and are working to quickly deploy fleets worldwide.  Three of the largest construction equipment suppliers have agreements with drone manufacturers to provide everything from off-the-shelf systems to full end-to-end services.  This addressable market segment, and its related services, is estimated at more than \$45 billion dollars by 2021.


%------------------------------
{\bf Agriculture:}
Agriculture, which will adopt UAVs more slowly, will still rank second worldwide over the next decade.  The ability to provide imagery that can detect when and where to apply fertilizer, pesticides and water offers tremendous potential.  While the agricultural UAV market will grow more slowly than construction or communication sectors, gradual adoption by skeptical farmers and improved technology will allow it to grow considerably and reach its full potential beyond the forecast period.  This addressable market segment is estimated to be more than \$32 billion dollars by 2021.




%------------------------------
\subsection*{Commercialization Implementation}
%------------------------------

Transitioning the TURN technology from basic research into a commercially viable product will require additional outside capital and specialized expertise from key individuals.


%------------------------------
{\bf Financing Plan:}
To arrive at an estimate of required funds to commercialize the TURN system, a preliminary top-level financial plan is presented.  It includes the technology/product development effort described for the current Phase II SBIR, and then considers potential spin-offs of other TURN vehicles at various scales serving different demographics.  From that foundation, initial required funding is comprised of: (i) \$3.0M of additional research development and testing, including subsequent prototype building and allocation for two engineers in the second year; (ii) \$50K for preparation and filing of utility patent applications; (iii) \$150K for business development including: fostering of strategic partnerships, travel funds, attorney fees, and other expenses; and (iv) \$300K for initial commercialization and marketing/advertising/promotional endeavors.  Thus, approximately \$3.5M is needed to fund the remaining research and initial commercialization activities contemplated above.  After developing the thirty-day endurance TURN system for the Air Force, ancillary markets will be pursued, including the small scale remote deployment system and the larger persistently enduring flight embodiment.  Growing the enterprise at this level will require executive leadership recruitment, additional business and marketing managers, and an expanded engineering staff.  These activities will likely require later-stage venture (equity) or debt financing in the amount of \$5.0M.  Therefore, estimated total funds for commercialization of the Gradient TURN technology is \$8.5M.


%------------------------------
{\bf Marketing Expertise:}
Gradient has tremendous experience in the R\&D/T\&E of new technologies, as well as some recent experience in product development.  To supplement that technical skill set, Gradient will retain the services of consultants and contractors experienced in appropriate fields.  In addition, Gradient has formed an Advisory Board to round out the necessary skill-set until full-time employees can be justified by business and technology maturity.  The skill-sets contained on the Gradient Advisory Board include product marketing, venture capital/venture development, industrial product development, technology commercialization, and intellectual property protection.








%------------------------------------------------------------
\section*{The Innovation}  % (recommended length: 1-3 pages)
%Briefly describe the innovation. At what stage of technical development is the innovation? (A more detailed description can be provided in the Technical Discussion and R&D Plan, as described below).
%Describe the key technical challenges and risks in bringing the innovation to market. Which of these will be your focus in the proposed Phase I project?
%Describe the status of the intellectual property associated with this project and how you plan to protect it.
%NSF Lineage: Does your project have roots in non-SBIR/STTR NSF funding, either to the company or other organizations/institutions? If possible, please list the NSF award number(s) and division(s).

Increased flight endurance is the most sought after improvement across the drone industry.  Unfortunately, tube-and-wing design aircraft have already reached their optimal structural and aerodynamic limits.  Thus, achieving a step-change improvement in aerodynamic performance requires a radical new concept approach.




%------------------------------
\subsection*{Radical New Concept Approach}
%------------------------------

The next section presents a novel UAS, which combines the best features of both gliders and helicopters, while minimizing their respective deficiencies.  Before introducing the concept vehicle, pros and cons of both gliders and helicopters are presented, to better illustrate the benefits of this radical new design.

\begin{figure} [!b]
\begin{floatrow}
\hspace{-0.1cm}
\ffigbox[\FBwidth]{
\includegraphics 
[ width=\Whalf, height=2.0in, clip, trim={0 0 0 0} ]
{../Figures/Glider}}
{\caption{Glider with High Aspect Ratio Wing}
\label{Fig:Glider}}
\hspace{-0.4cm}
\ffigbox[\FBwidth]{
\includegraphics 
[ width=\Whalf, height=2.0in, clip, trim={0 0 0 0} ]
{../Figures/Helicopter}}
{\caption{Helicopter Rotors with Deflection}
\label{Fig:Helicopter}}
\end{floatrow}
\end{figure}


%------------------------------
{\bf Glider Attributes:}
Gliders are some of the most aerodynamically efficient aircraft, because they strive to minimize drag.  They typically employ long slender wings, as seen in Figure \ref{Fig:Glider}, thereby increasing the aspect ratio, which is desirable from an aerodynamic drag perspective, but there are limitations with this approach.  Slender wings introduce flexibility, which can exhibit bend and twist during flight.  These wings are subjected to greater bending moments at the wing root, so added structural material must reinforce this connection point.  Finally, like all fixed-wing aircraft, gliders need forward velocity to generate airflow over the airfoil which produces lift, so they cannot hover or takeoff vertically.


%------------------------------
{\bf Helicopter Attributes:}
Unlike fixed-wing aircraft, helicopters have desirable VTOL and hovering capabilities.  Typical rotor blades are so thin, that they bend and deflect under their own weight, demonstrated in Figure \ref{Fig:Helicopter}.  Yet, these flimsy structures lift substantially heavy vehicles, because centrifugal forces provide stiffening throughout the rotor element.  Unfortunately, helicopter rotors are not aerodynamically efficient, because they suffer from triangular span loading.  Outboard sections produce a majority of the lift, while the inboard sections are quite ineffectual.  Finally, rotor mechanisms are extremely complex and must withstand massive internal structural forces.


%------------------------------
{\bf Best of Both Worlds:}
A completely new aerial concept vehicle incorporates features from both glider and helicopter methodologies.  Gliders utilize extremely efficient wings, but must mitigate flexibility and bending moments between the wing and the fuselage.  Helicopters support large payloads on flimsy rotors with centrifugal stiffening, but suffer from inefficient aerodynamic drag characteristics.  A new concept combines the best attributes of both types of aircraft, and minimizes the deficiencies of each, where centrifugal stiffening mitigates flexibility and moments found on gliders, while replacing inefficient rotors with extremely aerodynamic, low drag, high aspect ratio airfoils.




%------------------------------
\subsection*{Introducing the Tethered Uni-Rotor Network}
%------------------------------

The design objective reduced structural mass, maintained robust structural configuration, and simultaneously increased the lift-to-drag ratio.  A tensegrity approach minimized compression by putting structural loads in tension, where advanced composites provide incredible strength to weight.  The concept combines the best features of high efficiency glider and helicopter designs, reduces deficiencies within each class of aircraft, and yields better attributes than either can provide on their own.  This novel aerial vehicle, called the Tethered Uni-Rotor Network (TURN), illustrated in Figures \ref{Fig:TURN_Retract} and \ref{Fig:TURN_Extend}, utilizes a unique alternative approach, which exceeds existing capabilities, outpaces current research efforts, and vastly reduces power consumption.

\begin{figure} [!b]
\includegraphics 
[ width=\Wwhole, height=3.2in, clip, trim={0 0 0 0} ]
{../Figures/TURN_Retract}
\caption{Tethers are Retracted for Takeoff and Landing}
\label{Fig:TURN_Retract}
\end{figure}


%------------------------------
{\bf Vehicle Layout:}
The TURN system has a central hub which stores the vehicle payload.  Four small diameter cable tethers radiate outwards from the central hub and attach to satellite bodies.  Each has typical aircraft components, including: airfoil wing, propeller for thrust, stabilizers and control surfaces, and a fuselage for fuel, hardware, and sensors.  Each satellite resembles a flying-wing aircraft, which provides all the lift, propulsion and control for the TURN system.  A propeller is mounted on the leading edge of the outer wingtip; and immediately behind the prop, vertical stabilizers and control surfaces are located directly in the prop wash.  The concept is named \emph{Tethered Uni-Rotor Network} because a network of aircraft systems are tethered together to form a much larger singular rotor system.


%------------------------------
{\bf Hover Operation:}
The vehicle operates in a perpetual state of rotation, where each satellite drives the system rotation with its respective propulsion system.  This is opposite of a conventional helicopter, where a torque acts on a central shaft; with a tip driven system, no torque is transmitted back to the central hub.  As the system spins, centrifugal forces keep the tethers taught, and mitigate moments common within thin wings.  Lift is generated on each of the winged airfoil sections, which counteracts the weight of the satellite, and indirectly supports the weight of the central hub.

\begin{figure}
\includegraphics 
[ width=\Wwhole, height=3.2in, clip, trim={0 0 0 0} ]
{../Figures/TURN_Extend}
\caption{Normal Flight Operation with Tethers Fully Extended}
\label{Fig:TURN_Extend}
\end{figure}


%------------------------------
{\bf Flight Operation:}
Individual satellites are controlled through their propeller and control surfaces, and the central hub is controlled by coordinating the tether forces imparted from the satellite vehicles.  Two types of translation are considered, vertical and horizontal, which have a parallel in helicopter terminology as collective and cyclic commands.  Each type of translation has two associated control modes which can be implemented through different control inputs.


%------------------------------
{\bf Vertical Translation:}
Collective commands alter each satellite control input in unison.  Adjusting throttle increases or decreases the velocity of the satellite, and thus the angular rate of the TURN system.  This changes the airflow across the wing, which increases or decreases the total lift generated, and causes the vehicle to ascend or descend.  Alternatively, adjusting the pitch of each satellite through the elevator control surface, causes each satellite to nose up or down, thus the entire system will climb or fall as each satellite moves through a spiral trajectory.


%------------------------------
{\bf Horizontal Translation:}
Cyclic commands alter the control inputs in a sinusoidal fashion throughout the rotation.  The first cyclic command, applied to the rudder control surface, manipulates the radial tension on the tether.  At one point in the rotation there is maximum tension, and exactly opposite there is a minimum tension; this imbalance induces horizontal translation.  Another approach uses cyclic elevator commands, such that at one point in the rotation a satellite passes a low elevation, and it traverses a high elevation exactly opposite that point. This tilts the plane of rotation, which inclines the thrust vector of the airfoils, and yields a horizontal force component, which is identical to how traditional multirotors translate horizontally.


%------------------------------
{\bf Vertical Takeoff and Landing:}
The TURN system begins at rest on the tarmac with retracted tethers, as depicted in Figure \ref{Fig:TURN_Retract}.  Each wing is initially secured to the central hub with a docking mechanism which permits roll and pitch articulation within each satellite wing.  As the propellers ramp up, the TURN system begins to spin, while rolling on small wheels embedded within the vertical stabilizers.  Once sufficient angular rate has been attained, the system induces liftoff and rises above any local obstructions.  At a suitable altitude, each of the satellites are released from their docking mechanisms, and the tethers are extended to their normal flight operation.  Landing follows an identical procedure in the reverse order.




%------------------------------
\subsection*{System Benefits}
%------------------------------

Many unique benefits arise from the TURN system.  Advantageous centrifugal stiffening is similar to traditional helicopters, except thin tether filaments replace wasted material with detrimental aerodynamic properties.  Outboard sections are free to pivot, so there is no adverse bending moment typically found at a wing root.  This permits much higher aspect ratio, increased lift-to-drag ratio, and lower thickness-to-chord ratio, than can be attained by a tube-and-wing aircraft.  This new approach dramatically lowers structural weight, greatly reduces drag, and provides a far more robust structure than existing designs.  Consider some of the system benefits in more detail.


%------------------------------
{\bf Centrifugal Stiffening:}
Thin airfoils are especially susceptible to bending moments, so centrifugal stiffening is one of the most desirable attributes of the TURN system methodology, since very little material is needed within the spar.  The NASA/AeroVironment Helios, shown in Figure \ref{Fig:Helios}, is an example of a HALE aircraft experiencing very high bending moments across a large wing span, which ultimately failed from a lack of structural stiffness.  The TURN design places most of the mass at the largest possible radius, thereby reducing the rotor-wing speed, and increasing the amount of structural stiffening.  Utilizing a large tip mass, the TURN system only experiences a few g's of radial acceleration.

\begin{figure} [b!]
\captionsetup{width=0.9\textwidth}
\centering
\begin{minipage}{0.6\textwidth}
\captionsetup{width=0.9\textwidth}
\centering
\includegraphics 
[ width=1.01\linewidth, height=2.3in, clip, trim={0 0 0 0} ]
{../Figures/Helios_Flight}
\end{minipage}%
\begin{minipage}{0.4\textwidth}
\captionsetup{width=0.9\textwidth}
\centering
\includegraphics 
[ width=1.0\linewidth, height=2.3in, clip, trim={120 0 0 0} ]
{../Figures/Helios_Break}
\end{minipage}
\caption{Helios Aircraft with Severe Bending Moment and the Outcome From a Lack of Stiffness}
\label{Fig:Helios}
\end{figure}


%------------------------------
{\bf High Aspect Ratios:}
High aspect ratio wings improve aerodynamic efficiency, but the wing weight fraction is another closely related performance indicator.  Typical aircraft wings are commonly 10-20\% of the gross aircraft weight, and rarely exceed aspect ratios of ten.  Most HALE aircraft increase the aspect ratio of a tube-and-wing design, while pushing the limits of exotic structural materials.  While targeting aspect ratios of 20-25, they all experience similar wing weight fractions around 40-50\% of the total aircraft, indicating the wing receives extra material to reinforce a flimsy structure.  Conversely, most rotorcraft blades have aspect ratios of 40-50 and typically only represent 2-3\% of the gross helicopter weight.  Clearly, centrifugally stiffened thin structures can substantially reduce weight while using greater aspect ratios.  This approach used in the TURN concept permits high aspect ratios within very thin and lightweight airfoil structures.


%------------------------------
{\bf Tether:}
Helicopter rotors are inefficient, because only the outermost portions generate lift.  The inner sections are quite ineffectual, but still subjected to induced drag.  With this concept, the tether is 80\% of the total radius, which eliminates an underutilized structure.  At first glance, the tether may appear to significantly increase the parasitic drag; however, it is made of advanced aerospace grade compressed Spectra, which provides incredible strength-to-weight, so even a small tether cable diameter is structurally efficient at carrying tensile loads.  Because the average tether speed is only 40\% of the tip speed, the tether drag only constitutes around 5-10\% of the total drag acting on the system.


%------------------------------
{\bf High Lift-to-Drag Ratios:}
Ultra-endurance flight must resolve the conflict between the forces needed to keep the system aloft versus the power required to do so.  Thus, a high lift-to-drag ratio is extremely important.  Low bending moments permit very thin airfoil sections with low thickness-to-chord ratios, which is especially important for aircraft operating at a low Reynolds number.  This maintains laminar flow, which achieves extremely low levels of induced and parasitic drag.  With this concept, the tether comprises the innermost 80\% of the arm, so the satellites have a large radius of rotation which enables the wings to achieve elliptic span loading, rather than an inefficient triangular load distribution found on helicopters.  Helicopter rotors perform in close proximity of the downwash of the blade rotating in front of it, which increases the required angle of attack, and leads to high induced drag.  Each TURN satellite has a large spatial separation, so they operate independently of other downwash fields.  Because the total rotor disc area is directly related to the induced drag, the TURN approach offers a distinct advantage over conventional rotor designs.


%------------------------------
{\bf Rotating System:}
Conventional aircraft designs require that payload volumes move at the same speed as the rest of the vehicle, which results in substantial parasitic drag from bulky antenna or optic payloads.  Conversely, all the lift from the TURN concept is generated within the moving satellites, which are independent from the stationary hub.  This means larger and aerodynamically inefficient payload volumes can be incorporated into the central hub without incurring similar drag penalties, because the payload is isolated from the cruise speed components which generate lift.




%------------------------------
\subsection*{Various Scale TURN Embodiments}
%------------------------------

Improved aerodynamic efficiency, increased flight endurance, and a VTOL capability, offers many unique possibilities at a variety of different scales.  Consider some applications for three specific embodiments.


%------------------------------
{\bf Solar Powered Persistently Enduring Flight:}
Previous doctoral research, performed by the PI, investigated the TURN system as a means of attaining persistently enduring flight.  Utilizing readily available solar cell and battery technology, the TURN system shows great promise for attaining this goal at a wide range of latitudes through the entire year.  The results that are displayed in Table \ref{Tbl:SolarMap} show the amount of solar energy collection at a specified day and latitude.  Entries are coded such that red indicates there is not solar energy available for the task, green attains the goal with a 250 pound payload drawing 2000W of power, and yellow attains persistently enduring flight while supporting a 100 pound payload which consumes 750W of power.

\newcommand{\cp}[1]{$\;\;$#1$\;\;$}
\begin{table}
\small
\renewcommand{\arraystretch}{0.93}
\begin{tabular}{[c!cccccccccc]}
\thickhline
\bf Lat/Day&
\bf \cp{00}  & 
\bf \cp{40}  & 
\bf \cp{80}  & 
\bf \cp{120} & 
\bf \cp{160} &
\bf \cp{200} & 
\bf \cp{240} & 
\bf \cp{280} & 
\bf \cp{320} & 
\bf \cp{360} \\
\thickhline
\bf    90 &
\p   0.00 & \p  0.00 & \p  0.00 & \y  4.50 & \g  6.99 & 
\g   6.40 & \y  2.91 & \p  0.00 & \p  0.00 & \p  0.00 \\
\bf    85 &
\p   0.00 & \p  0.00 & \p  0.77 & \g  6.00 & \g  8.41 &
\g   7.84 & \y  4.44 & \p  0.00 & \p  0.00 & \p  0.00 \\
\bf    80 &
\p   0.00 & \p  0.00 & \y  1.55 & \g  7.45 & \g  9.76 &
\g   9.22 & \g  5.94 & \p  0.40 & \p  0.00 & \p  0.00 \\
\bf    75 &
\p   0.00 & \p  0.00 & \y  2.32 & \g  8.85 & \g 11.04 &
\g  10.53 & \g  5.58 & \y  1.09 & \p  0.00 & \p  0.00 \\
\bf    70 &
\p   0.00 & \p  0.38 & \y  3.07 & \g  7.87 & \g 12.23 &
\g  11.76 & \g  5.96 & \y  1.82 & \p  0.03 & \p  0.00 \\
\bf    65 &
\p   0.09 & \p  0.97 & \y  3.80 & \g  8.02 & \g 11.88 &
\g  10.70 & \g  6.44 & \y  2.56 & \p  0.46 & \p  0.08 \\
\bf    60 &
\p   0.54 & \y  1.63 & \y  4.50 & \g  8.30 & \g 11.12 &
\g  10.39 & \g  6.93 & \y  3.29 & \y  1.05 & \p  0.51 \\
\bf    55 &
\y   1.12 & \y  2.33 & \g  5.16 & \g  8.60 & \g 10.91 &
\g  10.34 & \g  7.40 & \y  4.01 & \y  1.70 & \y  1.08 \\
\bf    50 &
\y   1.77 & \y  3.04 & \g  5.79 & \g  8.89 & \g 10.83 &
\g  10.36 & \g  7.83 & \y  4.69 & \y  2.39 & \y  1.73 \\
\bf    45 &
\y   2.46 & \y  3.75 & \g  6.37 & \g  9.14 & \g 10.76 &
\g  10.38 & \g  8.22 & \g  5.35 & \y  3.10 & \y  2.42 \\
\bf    40 &
\y   3.17 & \y  4.45 & \g  6.91 & \g  9.34 & \g 10.69 &
\g  10.38 & \g  8.55 & \g  5.96 & \y  3.81 & \y  3.13 \\
\bf    35 &
\y   3.89 & \g  5.12 & \g  7.39 & \g  9.49 & \g 10.57 &
\g  10.33 & \g  8.83 & \g  6.53 & \y  4.51 & \y  3.85 \\
\bf    30 &
\y   4.60 & \g  5.77 & \g  7.82 & \g  9.58 & \g 10.42 &
\g  10.24 & \g  9.04 & \g  7.06 & \g  5.20 & \y  4.56 \\
\bf    25 &
\g   5.30 & \g  6.38 & \g  8.19 & \g  9.61 & \g 10.21 &
\g  10.09 & \g  9.19 & \g  7.53 & \g  5.85 & \g  5.26 \\
\bf    20 &
\g   5.98 & \g  6.95 & \g  8.49 & \g  9.57 & \g  9.95 &
\g   9.88 & \g  9.28 & \g  7.95 & \g  6.48 & \g  5.94 \\
\bf    15 &
\g   6.63 & \g  7.48 & \g  8.74 & \g  9.47 & \g  9.63 &
\g   9.61 & \g  9.30 & \g  8.31 & \g  7.07 & \g  6.60 \\
\bf    10 &
\g   7.24 & \g  7.96 & \g  8.91 & \g  9.30 & \g  9.26 &
\g   9.29 & \g  9.25 & \g  8.61 & \g  7.62 & \g  7.22 \\
\bf     5 &
\g   7.82 & \g  8.38 & \g  9.02 & \g  9.07 & \g  8.84 &
\g   8.91 & \g  9.13 & \g  8.84 & \g  8.11 & \g  7.79 \\
\bf     0 &
\g   8.35 & \g  8.75 & \g  9.07 & \g  8.78 & \g  8.36 &
\g   8.48 & \g  8.95 & \g  9.01 & \g  8.56 & \g  8.33 \\
\thickhline
\end{tabular}
{\caption{Solar Collection Map: Daily Energy per Solar Array Area (MJ m${}^{\text{-2}}$ day${}^{\text{-1}}$)}
\label{Tbl:SolarMap}}
\end{table}


%------------------------------
{\bf Combustion Engine Ultra-Endurance UAV:}
Gradient recently completed a Phase I SBIR research project, sponsored by the Air Force, to investigate the potential for a medium-altitude combustion engine TURN embodiment.  Existing surveillance drones are limited by their flight endurance, creating logistical problems, from finding a suitable launch area, to deploying multiple UAVs in waves to maintain continuous coverage.  An ultra-endurance flight capability solves these problems.  When flight endurance is no longer part of the equation, the launch site does not need to be within a particular radius of the target area, and multiple drones are not required to maintain continuous coverage of a desired target.  An individual TURN system can be deployed, navigate over to the point of interest several thousand miles away, and station-keep over the desired location for several weeks.  Finally, utilizing high TRL components, makes this a very reliable system, without depending on unproven experimental technology.  Some results from this work are depicted in Figure \ref{Fig:Endur}, which illustrates that this system is capable of attaining a 30-day flight endurance.

\begin{figure}
\includegraphics 
[ width=\Wwhole, height=3.2in, clip, trim={0 0 0 0} ]
{../Figures/Endur.png}
\caption{Flight endurance for combustion engine TURN system at various airspeeds and gust tolerance.}
\label{Fig:Endur}
\end{figure}


%------------------------------
{\bf Electric Remote Deployment UAV:}
Finally, this proposed NSF research will investigate the smallest TURN embodiment.  Existing 10-foot wingspan UAVs can carry 5 pound payloads for roughly 90 minutes, and many require a runway or catapult system for operations.  Conversely, a TURN system with an identical payload and battery mass can offer a VTOL capability, and is expected to attain a flight endurance of 4-7 hours.  With 6-foot rotor wings, and a system weight of about 35 pounds, this TURN embodiment would easily fit within a truck or van, could be transported to remote deployment locations, and operate without the need for a runway or ancillary equipment.  Because the research is utilizing a spiral development process, small prototypes are laying the groundwork for larger prototype development.  As such, this system represents the first commercial offering, before advancing to larger and more elaborate TURN embodiments.








%------------------------------------------------------------
\section*{The Company/Team}  % (recommended length: 1-3 pages)
%Describe the company founders or key participants in this proposed project. What level of effort will these persons devote to the proposed Phase I activities? How does the background and experience of the team enhance the credibility of the effort; have they previously taken similar products/services to market?
%Describe your vision for the company and the company's expected impact over the next five years.
%If the company has existing operations, describe how the proposed effort would fit into these activities. Describe the revenue history, if any, for the past three years. Include government funding and private investment in this discussion.
%Will you have consultants or subawardees working on this project? If so, what is their expertise, affiliation, and contribution to the project?

Developing the TURN aircraft requires a multidisciplinary effort, which will utilize specialized research groups across several domains.  Gradient Consulting will manage the overall project, and expand upon the existing dynamic system model and controls architecture.  DAR Corporation is responsible for developing the CFD, FEA, and aeroelastic analysis for the TURN system.  StarLab is a VICON motion capture studio which will collect high fidelity flight data used for parameter identification.




%------------------------------
\subsection*{Gradient Consulting [Dynamic Modeling and Controls Development]}
%------------------------------

The crux of the spiral development cycle resides within the model validation process.  Matching computer simulations to their physical counterparts, and identifying and assessing the discrepancies, increases confidence within the analytic tools utilized during subsequent prototype development.  Gradient Consulting specializes in modeling nonlinear dynamics of multibody systems, and implementing advanced model reference adaptive controls (MRAC) architectures on multiple-input multiple-output (MIMO) plants, to mitigate modeling uncertainty and environmental disturbances.  Parameter identification is used while investigating new concept aircraft architectures, which obtains the physical characteristics of an unknown system.  Multisine signal injection routines are sinusoidal signals superimposed on the control inputs to facilitate parameter convergence to their true values.  Once the vehicle dynamics are known through flight testing, simulation models are optimized to match their physical counterparts, yielding an analytic tool for subsequent designs.




%------------------------------
\subsection*{DAR Corporation [CFD/FEA/Aeroelastic Analysis]}
%------------------------------

For a low-power ultra-endurance aircraft, aerodynamic efficiency and structural robustness are of paramount importance.  Unfortunately, they are also the most difficult tasks to model properly.  Realizing the full potential of the TURN methodology, requires slender wings with high aspect ratios, large tip masses to lower the system rotation rate, and unconventional control surface placement; thus, commonly used ``best-practices'' in aircraft design are not likely sufficient to accurately depict the flight performance.  As such, developing computational fluid dynamic models, performing finite element analysis, and then evaluating them against collected flight data from a physical system, is the primary goal of the next prototype installments.  This will identify discrepancies and ultimately validate the models, which are then used as a design tool to optimize the aerodynamic performance of the next TURN system.  DAR Corporation is a leading industry expert in small aircraft analysis and design, has a proven track record for aeroelastic research, and their experience with propeller and wind turbine systems is extremely well suited for the rotating TURN concept.




%------------------------------
\subsection*{StarLab [VICON Motion Capture Studio]}
%------------------------------

During Phase I SBIR research for the Air Force, two prototypes mitigated the largest risk: developing custom avionics hardware, and demonstrating a suitable flight controller for a TURN architecture.  Results from early flight tests indicated that standard attitude estimation algorithms cannot account for the rotating nature of the TURN system.  An initial modification was sufficient to accomplish the stated research objectives, but additional testing is needed to refine the algorithm, and complete the system identification testing.  Improving the attitude heading and reference system (AHRS) algorithm requires an external localization solution, independent of the on-board sensors, which can serve as ground truth.  StarLab is a VICON motion capture studio, which collects high-quality flight data, sampled at 500 Hz, with sub-millimeter resolution.  Furthermore, as an indoor testing facility, there is no external disturbance from wind, which can bias the results.  Once that data is collected, the on-board AHRS algorithm can be fine-tuned by filtering signal noise and adjusting gain settings, and parameter identification can validate the multibody dynamic simulation model.








%------------------------------------------------------------
\section*{Technical Discussion and R\&D Plan}  % (minimum length: 5 pages, recommended length: 5-7 pages)
%Describe the innovation in sufficient technical depth for a knowledgeable reviewer to understand why it is innovative and how it can provide benefits in the target applications. Supplement this description with any necessary background information.
%Describe the key objectives to be accomplished during the Phase I research, including the questions that must be answered to determine the technical AND commercial feasibility of the proposed concept.
%Describe the critical technical milestones that must be met to get the product or service to market.
%Present an R&D plan, with timeline. What are the objectives, and what experiments, computations, etc. are planned to reach those objectives?

The goal of this proposed NSF SBIR project is to take the next step within the spiral development process which is developing and validating the analytic tools and models needed to design a TURN system with confidence.  Previous PhD research and Air Force SBIR funding have developed:
\begin{itemize}
\item Nonlinear multibody dynamic model of the TURN system;
\item Adaptive controls methodology for stabilization, inner-loop, and outer-loop components;
\item Custom avionics to facilitate the prototype development process;
\item Prototype testing platform to evaluate custom avionics; and
\item Modular prototype platform for multibody dynamic model validation.
\end{itemize}
The following section outlines the overall spiral development process, the role of motion capture and parameter identification routines applied to an existing airframe which will validate the multibody dynamic model, the need for CFD/FEA models, and how those results will contribute towards the next prototype installment.




%------------------------------
\subsection*{Spiral Development Plan}
%------------------------------

The goal is to develop the set of tools and models to design a TURN system with confidence.  Spiral development is an iterative approach, where prototypes address the largest risk, simulation models are refined with flight data, and enhanced analytic tools aid with subsequent vehicle designs.


%------------------------------
{\bf Custom Avionics Platform [Mark 0]:}
Before any type of TURN prototype system could be controlled, a custom designed flight control system (FCS) was needed.  This prototype, depicted in Figure \ref{Fig:Mark0}, was developed during the PhD research effort, and served as a dynamic test bench which: field tested hardware, debugged software, characterized sensor noise, implemented filter routines, evaluated new state estimation algorithms, and assessed controller gain selection.

\begin{figure}
\includegraphics 
[ width=\Wwhole, height=3.2in, clip, trim={0 0 0 0} ]
{../Figures/Mark0.jpg}
\caption{\emph{Mark 0} prototype for custom avionics development and evaluation.}
\label{Fig:Mark0}
\end{figure}


%------------------------------
{\bf Multibody Dynamics Validation [Mark 1]:}
The TURN vehicle is a multibody system, with a complex set of intertwined dynamics.  This prototype, shown in Figure \ref{Fig:Mark1}, was built during the Phase I SBIR Air Force research project, and established a testing platform to validate accuracy and identify limitations within the existing computer models and simulation routines.  Modularity was the emphasis of this prototype design, which will assess the system dynamics across several parameter variables and flight conditions.

\begin{figure}
\includegraphics 
[ width=\Wwhole, height=3.2in, clip, trim={0 0 0 0} ]
{../Figures/Mark1.jpg}
\caption{\emph{Mark 1} prototype for AHRS testing and multibody dynamic validation.}
\label{Fig:Mark1}
\end{figure}


%------------------------------
{\bf Aerodynamic Model Refinement [Mark 2]:}
Aerodynamic efficiency has the strongest influence on aircraft dynamics and performance, so accurate models must be developed as early as possible.  The goal of this NSF proposed research is to develop a CFD/FEA/aeroelastic models, perform optimizations and trade studies, and make predictions concerning the flight characteristics and endurance of the next prototype.  When the prototype is built with future funding, the intention is to run parameter identification on flight data, and then compare the analysis results against physical reality, which yields an analytic tool for subsequent designs.




%------------------------------
\subsection*{Multibody Dynamics Validation}
%------------------------------

A nonlinear multibody dynamic model and several adaptive control laws were devised from previous PhD research, and an avionics controller was developed during the Phase I effort.  The \emph{Mark 1} system has been built and performed initial flight testing, and is ready for more extensive VICON motion capture flight data.


%------------------------------
{\bf Objective:}
The goal is to validate the nonlinear multibody dynamic simulation model, so it can be used as a design tool for subsequent vehicle iterations.  The process uses the \emph{Mark 1} prototype, displayed in Figure \ref{Fig:Mark1}, collects flight data across a wide range of operating points, iterates through different physical parameters, and then compares recorded flight data to the results produced by the simulation model.  To accomplish this task, VICON motion capture provides external ground truth while recording system states, parameter identification yields the steady state and transient dynamics of the system, and a modular prototype framework is utilized to encompass a wide array of physical geometries.


%------------------------------
{\bf Motion Capture:}
The VICON motion capture studio is an indoor testing facility equipped with 44 infrared cameras that detect reflective markers placed on an object.  Position and orientation states are collected at each sample step, which are used to calculate all the linear velocities and angular rates, which completely describes the motion of each rigid body element.  Data is sampled at 500 Hz, with sub-millimeter resolution, which offers higher fidelity than the on-board sensors, and provides a localization solution within a GPS-denied environment.  Obtaining this high quality flight data is of paramount importance for implementing the parameter identification process.


%------------------------------
{\bf Parameter Identification Process:}
The simulation model must match physical reality in two different ways.  First, the trimmed steady state operating points are the set of states and control inputs which achieve a static equilibrium, such as hover or a constant rate of ascent.  Second, the transient dynamic response is described by the elements within the state space matrices, and are used for reference tracking and disturbance rejection.  Parameter identification algorithms run through the collected flight data and yield both the trimmed equilibrium operating points, and the state space dynamic model.  The state space form is populated with the partial derivatives of each state and control input with respect to the state derivatives, which mathematically describes the dynamic response of the system.  Similarly, the simulation model contains both trimming and linearization routines, which provide the state space representation that exactly matches the form provided by the parameter identification algorithms.  So, running multisine signal injection during the flight testing, and post processing the data via parameter identification, will directly convey any discrepancies between the physical flight characteristics and the simulated dynamics.  Validating the simulation across a wide range of operating conditions and physical parameters necessitates a modular prototype design.


%------------------------------
{\bf Elements of the Modular Design:}
The \emph{Mark 1} system was designed as a modular testing platform.  Several attributes within the prototype are adjustable, and intended to collect key parameters which are needed to validate the multibody dynamic simulation model.  A total of six wing blades were constructed, each with a 40'' wingspan.  Four of the blades have aspect ratios of 12, and will be used together while demonstrating a complete flight capability.  The other two blades have aspect ratios of 10 and 14, which will be utilized for model validation purposes.  Modularity within the system includes: aspect ratio, prop diameter, elevator control surface area, and tail boom length.  Additionally, the battery location is adjustable, such that the CG of the satellite aligns with the quarter-chord position of the wing.
\begin{itemize}
\item Two prop diameters will be tested, which will assess throttle commands versus power consumption, and identify the control bandwidth available to maintain equal spacing and coordinate the satellites.
\item There are three tail boom lengths to alter the moment arm distance between the elevator control surface and the CG of the satellite, which will evaluate the pitch dynamics within the simulation model.
\item Testing two sizes of elevator control surface, which dictates the force generated from elevator deflections, and will evaluate the engineering tradeoff between drag and control bandwidth.
\item Angle of attack will be evaluated at 3\degree, 5\degree, and 7\degree{} increments, which influences the required airspeed across the airfoil, and will help identify proper lift and drag coefficients across the operating region.
\item Each blade has the same wingspan, but aspect ratio is evaluated at 10, 12, and 14, which is synonymous with adjusting the wing surface area, thus directly influencing the angular rate and centrifugal force within the TURN system, which will review the rotational aspects within the simulation model.
\end{itemize}


%------------------------------
{\bf Additional Battery:}
Each satellite can be flown with one or two battery packs.  The goal of adding a second battery is not to increase the flight endurance, rather it increases the weight of each satellite.  Since weight is directly counteracted with lift, this modification directly influences the angular rate and centrifugal force on each tether arm.  Time permitting, this option would provide another dimension to evaluate multiple physical parameters (weight/lift, drag/thrust, control bandwidth, rotational influences) within the simulation model.  Since all of these parameters can be obtained directly from the previous adjustments, this testing would serve as an optional redundancy check within the model validation process.


%------------------------------
{\bf Testing Process:}
Altering propellers, tail boom lengths, and elevator surface areas are all easily adjustable, so testing each set of these parameters could be accomplished within a single day.  Then, proceeding through three different aspect ratios and three angle of attack settings, would require nine full days of VICON testing.  Repeating the full set of testing with a second battery configuration would double the testing duration to 18 days.  If needed, a smaller subset of testing with a second battery, may focus on operational areas that are most meaningful to the simulation validation.  To be safe, 20 days of VICON testing are allocated within the NSF SBIR Phase I budget proposal.


%------------------------------
{\bf Anticipated Outcomes:}
Any simulation model can be manipulated to fit a single operating condition, so evaluating a full range of physical parameters at different operating points will improve the level of confidence within this tool.  This activity yields elements within the state space matrices which represent the partial derivatives of the states and control inputs.  These values are needed to refine the accuracy of the dynamic models, and identify boundaries and limitations that need to be observed during the design process.  Since, the multibody simulation model is the primary means of sizing TURN components and evaluating controller gains, validating this tool is a crucial step prior to proceeding onto further prototype development.




%------------------------------
\subsection*{Aeroelastic Model Development}
%------------------------------

Having validated the multibody dynamic model, the next most important design tool is the aeroelastic analysis.  A computational fluid dynamic (CFD) model provides aerodynamic forces and moments, and a finite element analysis (FEA) provides insight into structural and bending properties.  Together they form a complete aeroelastic analysis which can be used during the custom airfoil design process.


%------------------------------
{\bf Objective:}
While the existing prototype focuses on validating the multibody dynamic simulation model, this activity develops the remaining tools needed to completely design a TURN system.  The primary purpose is to formulate the aerodynamic and structural models, which form a complete aeroelastic representation of the vehicle, and will be used to design a custom airfoil to fully leverage centrifugal stiffening within the TURN concept design.  Similar to the previous activity, the CFD/FEA/aeroelastic analysis will provide insight into larger TURN designs.  However the accuracy of such an analysis, specifically addressing the rotating aspect of the system, has not yet been assessed.  These tools will facilitate the next prototype design, and after construction in future research, the models will be validated through flight testing similar to the \emph{Mark 1}.


%------------------------------
{\bf Computational Fluid Dynamic Models:}
The goal is to develop an analytic tool that depicts accurate force and moment values acting at the CG of each rigid body element, while operating at different flight conditions.  A complete TURN system and its rotation will be depicted within the simulation model.  An initial baseline configuration will obtain states and control inputs that achieve steady-state equilibrium, which is synonymous with the hover operating condition.  Then, a number of parameters will be independently adjusted away from the trim condition, which yield a new set of forces and moments.  Because these points are no longer trimmed values, the forces and moments will influence the dynamics of the system and alter its operating state.  Adjustable parameters under investigation include: throttle input, elevator input, airfoil angle of attack, and airflow velocity over the wing.  For each of the four parameters, two points above and two points below the trimmed operating point will be selected.  A complete set of forces and moments at various operating conditions provides an array of values that can be matched against the physical flight data.


%------------------------------
{\bf Finite Element Analysis:}
Following the CFD task, DAR will develop a Finite Element Analysis (FEA) model, to assess static and dynamic structural loads.  The culmination of these two models will provide a full aeroelastic model analysis.  With larger scale vehicles utilizing higher aspect ratio wings, understanding the structural and aeroelastic properties is of utmost importance.  DAR will review: flutter, which is a type of dynamic instability undergoing simple harmonic motion; divergence, where a deflected load increases the magnitude of the deflection; and control surface reversal, where inputs have the opposite expected response.  Finally, a refined propeller model utilizing blade element theory, will better assess prop wash interactions on the fuselage, stabilizers and control surfaces.


%------------------------------
{\bf Custom Airfoil Design:}
Up to this point, each prototype utilized an existing airfoil profile with known characteristics.  This accelerated the development process, and reduced program risk by utilizing components with familiar capabilities.  While these serve their primary purpose, they do not approach the full potential that centrifugal stiffening offers from the TURN architecture.  DAR Corp will be tasked with developing a custom airfoil profile for the TURN system.  Their CFD and FEA models will provide a baseline for the dimensions and characteristics desired within the custom airfoil.  Given these inputs, they will develop a tailored pressure distribution to maximize lift and minimize drag, while exploring geometries (thin thickness, high camber) that are not typically appropriate for standard traditional fixed-wing aircraft.


%------------------------------
{\bf Model Validation:}
Building the next prototype embodiment is reserved for future work.  But once that effort is underway, identical to the \emph{Mark 1} prototype, parameter identification algorithms will yield the state space representation of the system.  As before, the matrix elements represent the partial derivatives of the system states, and prescribe the mathematical relationships between control inputs and system outputs.  However, unlike the multibody validation, where this relationship was directly obtained from the linearization routine within the simulation, the CFD validation requires one intermediate step: identifying the forces and moments acting on each of the TURN components.  The CFD analysis produces forces and moments for a given set of system states and control inputs.  This induces translation and rotation in a known manner because it must adhere to the laws of physics.  Therefore, with known mass and moment inertia parameters, the time-varying force and moment values can be derived directly from the system states and control inputs via the recorded flight data.  The set of forces and moments that arise during flight testing will be compared to the CFD results, which will identify and assess any major discrepancies, fine tune the parameter settings, and validate the CFD modeling process.


%------------------------------
{\bf Anticipated Outcomes:}
The outcome of this effort will yield all the tools needed to successfully design a TURN system.  The next installment in the spiral development process will use these tools to build the next largest prototype, where flight testing will validate the models developed within this current research activity.  These tools will help predict performance and flight endurance, responsiveness and handling characteristics of the system, illuminate any required revisions within subsequent designs, and they will provide the foundation for performing design trade studies and engineering optimizations.




%------------------------------
\subsection*{Design Trade Studies and Engineering Optimizations}
%------------------------------

Following the CFD/FEA/aeroelastic modeling, new tools will be available to further evaluate the TURN concept vehicle through design trade studies and engineering optimizations.


%------------------------------
{\bf Objective:}
Having validated the primary set of tools needed to design a TURN aircraft, the next task utilizes those tools and investigates how to maximize the performance of subsequent TURN prototypes.  Some design parameters are in conflict with one another, so trade studies will showcase the relative cost/benefit of each embodiment.  Similarly, there are many potential design optimizations within the TURN architecture, which can now be more formally investigated via the CFD analysis.


%------------------------------
{\bf Design Trade Studies:}
The following analyses will evaluate the cost/benefit associated with each feature.

\emph{Aspect Ratio:}
Lift induced drag is heavily influenced by the aspect ratio of the wing.  While, a high aspect ratio offers a definitive improvement in aerodynamic performance, slender wings also introduce flexibility within the structure.  Centrifugal stiffening within the TURN system offers a decided advantage, but a formal investigation is still needed to identify the diminishing returns on aerodynamic performance.

\emph{Winglets:}
These improve aerodynamic performance by decreasing detrimental wingtip vortices, but they add weight to the system and have a more pronounced influence on low aspect ratio wings.  This analysis will assess whether any reduction in drag is worth the added weight penalty.

\emph{Washout:}
This feature varies the incidence of the airfoil across the wingspan.  In general, it can decrease aerodynamic efficiency because the entirety of the wing no longer operates at an ideal angle-of-attack.  However, it offers two potential benefits.  First, it can be used to attain an elliptic load distribution, which increases the Oswald efficiency factor, and reduces induced drag from wingtip vortices.  Second, different portions of the wing stall at different times, which adds robustness into the flight envelope.  These benefits will be weighed against any degradation in aerodynamic efficiency.


%------------------------------
{\bf Engineering Optimizations:}
The following optimizations will identify ideal values for each design feature.

\emph{Outboard Horizontal Stabilizer (OHS):}
Research shows that a horizontal stabilizer properly placed within the wingtip vortex can reduce drag and contribute some amount of lift.  Appropriate sizing and placement is highly dependent on the system, so this task will find an ideal layout to capture the maximum benefit.

\emph{Dihedral:}
Centrifugal stiffening will greatly mitigate bending moments within the wing, but some may still be present.  Introducing a curvature with some dihedral can offset any remaining bending, which will help keep the wing horizontal during operation.

\emph{Asymmetric Span Profile:}
An asymmetric nonuniform profile distribution is a unique feature available to a TURN system.  From the system rotation, the outboard wingtip travels faster than the inboard section.  An asymmetric wingspan can exploit this feature, such that each segment is optimized for its intended airspeed.


%------------------------------
{\bf Anticipated Outcomes:}
These analyses will provide valuable insight for subsequent TURN prototypes, help increase aerodynamic performance, improve flight handling characteristics, and save time and money by seeking optimized TURN designs from an early stage of development.








%------------------------------------------------------------
\newpage \cfoot{ }
{\color{red} \bf \Huge References Cited}
%------------------------------------------------------------
% Provide a comprehensive listing of relevant references, including patent numbers and other relevant intellectual property citations. A list of References Cited must be uploaded into the system. If there are no references cited in the proposal, please indicate this by putting the statement "No References Cited" into this module.

Insert.








%------------------------------------------------------------
\newpage
{\bf \Huge Biographical Sketches}
%------------------------------------------------------------
% Provide a resume for the Principal Investigator (PI) and senior personnel (individuals with critical expertise who will be working on the project and are employed at the proposing company or at a subaward institution). Information regarding consultants should also be provided in this format but instead uploaded as part of the Budget Justification. Biographical sketches should not exceed two pages per person. You are not required to use the NSF Biographical Sketch template. Do not include personal information such as home address in biographical sketches. Provide information in the following sections: (I) Education: Institution, Location, Major/Concentration, Degree, and Year. (II) Relevant Experience: Include technical and/or commercial experience. List in reverse chronological order beginning with the current position. (III) Products: Includes patents, publications, etc. Up to 5 may be listed that are related to the proposed work and up to 5 that are significant but not related to the proposed work.


%------------------------------
\vspace{0.2cm}
{\bf \Large Justin M Selfridge, PhD (Principal Investigator)}

United States Citizen \\
PhD in Electrical and Computer 
Engineering - University of Virginia (UVA) \\
Gradient Consulting, LLC - Owner, Managing Member \\
NASA Langley - Independent Contractor


%------------------------------
{\bf Biography} \\
Dr Justin M Selfridge completed his PhD in Electrical Engineering at the University of Virginia.  Research areas include: multivariable adaptive control, system identification, autonomous vehicles, custom avionics hardware development, and UAV design and prototype fabrication.  For six years, Dr Selfridge worked within the Dynamic Systems and Controls Branch at NASA Langley.  His most recent NASA project developed the controller for the \emph{GL-10} concept vehicle; a ten rotor, tilt wing/tail, transformational flight vehicle, with VTOL capabilities and an extended flight endurance.  His PhD developed the nonlinear dynamic model, adaptive controller methodology, and numerical multibody simulation tools, for the TURN concept system; a novel aircraft seeking persistently enduring flight.  The past year he has been pursuing the TURN vehicle development full-time, working under his consulting company, Gradient Consulting, LLC.


%------------------------------
{\bf Education} \\
\begin{tabular}{llll}
\vspace{-0.15in}
\hspace{2.2in} &
\hspace{1.3in} &
\hspace{1.3in} &
\hspace{0.0in} \\
University of Virginia &
Charlottesville, VA &
Electrical Engr &
Ph.D., 2017 \\
Old Dominion University &
Norfolk, VA &
Mechanical Engr &
MS, 2012 \\
University of Virginia &
Charlottesville, VA &
Mechanical Engr &
BS, 2004
\end{tabular}


%------------------------------
{\bf Research Experience} \\
\begin{tabular}{llll}
\vspace{-0.15in}
\hspace{2.2in} &
\hspace{1.3in} &
\hspace{1.3in} &
\hspace{0.0in} \\
Air Force Phase I SBIR &
Newport News, VA &
Principle Investigator &
2017-present \\
NASA Langley Research Center &
Hampton, VA &
Contractor &
2013-2017 \\
NASA Langley Research Center &
Hampton, VA &
Work Study &
2011-2013 \\
University of Virginia &
Charlottesville, VA &
Teach/Research Asst &
2012-2014 \\
Old Dominion University &
Norfolk, VA &
Teach/Research Asst &
2010-2012
\end{tabular}


%------------------------------
{\bf Authored Publications} \\
\vspace{-0.2in}

\PubSpace
Selfridge, Justin M.
\emph{Achieving Eternal Flight With a Tethered Uni-Rotor Network (TURN) Aircraft: A Complete Development of the Nonlinear Dynamic Model and Controller Architecture}.
Doctorate of Philosophy dissertation.
University of Virginia; Charlottesville, VA;
May 2017.

\PubSpace
Selfridge, Justin M., and Gang Tao.
``Multivariable Output Feedback MRAC for a Quadrotor UAV.''
\emph{IEEE American Control Conference (ACC).}
Boston, MA.
Jul 2016.

\PubSpace
Selfridge, Justin M., and Gang Tao.
``Centrifugally Stiffened Rotor: A Complete Derivation of the Plant Model with Nonlinear Dynamics.''
\emph{AIAA Aviation Technology, Integration, and Operations Conference.}
Dallas, TX.
Jun 2015.

\PubSpace
Selfridge, Justin M., and Gang Tao.
``Centrifugally Stiffened Rotor: A Complete Derivation and Simulation of the Inner Loop Controller.''
\emph{AIAA Guidance Navigation and Control (GNC) Conference.}
Kissimmee, FL.
Jan 2015.

\PubSpace
Selfridge, Justin M., and Gang Tao.
``A Multivariable Adaptive Controller for A Quadrotor with Guaranteed Matching Conditions.''
\emph{IEEE American Control Conference (ACC).}
Portland, OR.
Jun 2014.

\PubSpace
Selfridge, Justin M.
``A Multivariable Adaptive Controller for a Quadrotor with Guaranteed Matching Conditions.''
\emph{System Science and Control Engineering.}
Vol 2, No 1, pp 24-33; 2014.

\PubSpace
Selfridge, Justin M.
\emph{A Complete Autonomous Vehicle Solution: An ASV Case Study}.
Master's thesis.
Old Dominion University; Norfolk, VA.
May 2012.








%------------------------------------------------------------
\newpage
{\bf \Huge Budget Justification}
%------------------------------------------------------------
%The Budget Justification is uploaded in the Budget Module of FastLane as a single PDF file. Provide details for each non-zero line item of the budget, including a description and cost estimates. Identify each line item by its letter and number (e.g., G.5 - Subawards). Each >non-zero line item should be described in the Budget Justification, but several sections also require more specific information as detailed below. Please note that there is no page limit for the Budget and Subaward Budget Justification. You will receive and should disregard a FastLane warning about exceeding the three-page limit when you submit the proposal. In other words, the PAPPG restriction on page limits is not enforced for this solicitation (but all information included in this section must be related to the budget). You can find a sample budget and subaward budget, with justifications, here: https://seedfund.nsf.gov/fastlane/.


%------------------------------
%Line A - Senior Personnel. List the Principal Investigator and Senior Personnel by name, their time commitments (in calendar months), and the dollar amount requested. Senior Personnel are individuals with critical expertise who are employed at the proposing company. The PI must be budgeted for a minimum of one month to the proposed project and may be budgeted for more than two months (deviates from PAPPG-Chapter II.C.2g(i)(a)). The best source in determining an appropriate salary request is the Bureau of Labor Statistics. In the Budget Justification provide the title; annual, monthly, or hourly salary rate; time commitment; a calculation of the total requested salary; and a description of responsibilities for the PI and each of the Senior Personnel.
{\bf Senior Personnel:} Payscale.com indicates early career PhD income as \$103k/yr.  Dr Selfridge collects \$100k/yr.  Working full time on a 6.0 month project yields \$50k for the proposed budget.


%------------------------------
%Line B - Other Personnel. List the number of additional general personnel (technicians, programmers, etc.) and the total monetary and time commitment for these personnel. These personnel must be employed at the proposing company. The details of the individual commitments, roles, and requested funds should be provided in the Budget Justification. Do NOT list company employees under B.1, B.3, or B.4 in the main budget. Post-doctoral scholars and students (undergraduate and graduate) should be listed on a subaward budget to a research institution, unless they are employees of the company, in which case they may be listed under Lines A, B.2, or B.6, as appropriate.
{\bf Other Personnel:} None.


%------------------------------
%Line C - Fringe Benefits. It is recommended that proposers allot funds for fringe benefits here ONLY if the proposer's usual (established) accounting practices provide that fringe benefits be treated as direct costs. Otherwise, fringe benefits should be included in Line I, Indirect costs.
{\bf Fringe Benefits:} \\
\begin{tabular}{lll}
Vacation:     &  5 work days/yr         & = 2\%  \\
Sick Pay:     &  5 work days/yr         & = 2\%  \\
Retirement:   & (\$4k/yr) / (\$100k/yr) & = 4\%  \\
Insurance:    & (\$8k/yr) / (\$100k/yr) & = 8\%  \\
\end{tabular}  \\
Total fringe benefit percentage is 16\% of direct labor cost, which equates to \$8k for the proposed budget.


%------------------------------
%Line D - Equipment. Equipment may NOT be purchased on an NSF SBIR Phase I grant. Equipment is defined as an item of property that has an acquisition cost of $5,000 or more (unless the organization has established lower levels) and an expected service life of more than one year.
{\bf Equipment:} None.


%------------------------------
%Line E.1 - Travel. One domestic travel trip for up to two persons (the PI is required to attend, and we recommend also including an individual who leads the related business/commercial efforts, if not the PI) should be budgeted to attend a three-day Grantee Conference in the DC area. The intent of this workshop is to discuss the research program with the cognizant Program Director, learn about preparing a Phase II proposal, and hear from experts on various topics of interest to technical entrepreneurs. An explicit statement acknowledging attendance at the mandatory grantee workshop is required on the Budget Justification page. A reasonable budget estimate is $2,000 per person to cover the conference registration fees and travel expenses. Outside of this event, all other budgeted travel must be necessary for the successful execution of the Phase I R&D. Travel for purposes other than the project R&D (e.g. marketing, customer engagements) is not permitted in the Phase I budget, EXCEPT as permitted for participation in the Beat The Odds Boot Camp, see discussion below for Line G.6.
%Line E.2 - Foreign travel expenses (Line E.2) are NOT permitted.
{\bf Travel Costs:} An amount of \$2k is allocated for Dr Selfridge to attend the ``Grantee Workshop.''  The ``Beat-the-Odds Boot Camp'' is accounted for within the ``Other Direct Costs'' section.  No other travel arrangements, foreign or domestic, are budgeted.


%------------------------------
%Line F - Participant Support Costs. Participant support costs are NOT permitted on a Phase I grant.
{\bf Participant Support Costs:} None.


%------------------------------
%Line G.1 - Materials and Supplies. Materials and supplies are defined as tangible personal property, other than equipment, costing less than $5,000, or other lower threshold consistent with the policy established by the proposing organization. The proposal Budget Justification should indicate the specifics of the materials and supplies required, including an estimated cost for each item. Items with a total cost exceeding $5,000 should be accompanied by pricing documentation (e.g. quote, link to online price list, prior purchase order or invoice), to be included in the budget justification.
{\bf Materials and Supplies:} None.


%------------------------------
%Line G.2 - Publication Costs/Documentation Costs. Publication Costs/Documentation costs are NOT permitted on a Phase I proposal.
{\bf Publication/Documentation Costs:} None.


%------------------------------
%Line G.3 - Consultant Services. Consultant services include specialized work that will be performed by professionals that are not employees of the proposing small business. Purchases of analytical services, other services, or fabricated components from commercial sources should not be listed under consultant services and should instead be reported in the budget under Other Direct Costs/Other (Line G.6). No person who is an equity holder, employee, or officer of the proposing small business may be paid as a consultant unless an exception is recommended by the Program Director and approved by the Division Director for the Division of Industrial Innovation and Partnerships. All research on an SBIR project, including that conducted by consultants, must be carried out in the U.S. (Place of Performance is defined as: For both Phases I and II, all research must be performed in the United States. "United States" means the 50 states, the territories and possessions of the U.S. Federal Government, the Commonwealth of Puerto Rico, the District of Columbia, the Republic of the Marshall Islands, the Federated States of Micronesia, and the Republic of Palau).
{\bf Consultant Services:} None.


%------------------------------
%Line G.4 - Computer Services. Funds may be allocated for computer services. Requested items with a total cost exceeding $5,000 should be accompanied by pricing documentation (e.g. quote, link to online price list, prior purchase order or invoice), to be included in the budget justification.
{\bf Computer Services:} None.


%------------------------------
%Line G.5 - Subawards. Subawards may be utilized when a significant portion of the work will be performed by another organization and when the work to be done is not widely commercially available. Work performed by a university or research laboratory is one example of a common subaward. Purchases of analytical services, other services, or fabricated components from commercial sources should not be listed under subawards and should instead be reported in the budget under Other Direct Costs/Other (Line G.6). No person who is an equity holder, employee, or officer of the proposing small business may be paid under a subaward unless an exception is recommended by the Program Director and approved by the Division Director for the Division of Industrial Innovation and Partnerships. Subawards require a separate subaward budget and subaward budget justification, in the same format as the main budget. Subawardees (the institution, not the individual PI or researcher) should also provide a letter of collaboration that confirms the role of the subaward organization in the project and explicitly states the subaward amount. Provide this letter as part of the Budget Justification and NOT as a Supplementary Document.
{\bf Subawards:}
\begin{itemize}
\item StarLab (VICON Motion Capture Studio): 20 days of flight testing at \$1000/day equates to \$20k allocated within the budget.
\item DAR Corporation (CFD Development): Aerodynamic Analysis was quoted at \$68,800.  Performing trade studies and optimizations was quoted at \$60,320.
\end{itemize}


%------------------------------
%Line G.6 - Other. This line includes the purchase of analytical services, other services, or fabricated components from commercial sources. Requested items with a total cost exceeding $5,000 should be accompanied by pricing documentation (e.g. quote, link to online price list, prior purchase order or invoice), to be included in the budget justification. In addition to the above, there are two other activities for which NSF permits the inclusion of additional funds on this budget line; see below. The funds noted below may NOT be spent for any other commercial or business purpose not explicitly permitted below. The funds noted below may NOT be spent for any other commercial or business purpose not explicitly permitted below.
%Firstly, the applicant may budget up to $10,000 as a direct charge on line G.6 to this Phase I award for the following specific purposes related to financials and accounting:
%Hiring a certified public accountant (CPA) to prepare audited, compiled, or reviewed financial statements
%Hiring a CPA to perform an initial financial viability assessment based on standard financial ratios so the awardee organization would have time to improve their financial position prior to submitting the Phase II proposal
%Hiring a CPA to review the adequacy of the awardee's project cost accounting system
%Purchasing a project cost accounting system
%If the applicant elects to budget funds for one of the above purposes, the budget justification should include a brief description of the desired use of funds, and the use of funds must be approved by the cognizant Program Director, prior to award.
%Second, the applicant may budget up to $20,000 to cover costs related to NSF's "Beat-The-Odds Boot Camp" which is offered to all Phase I awardees. This program is based on the NSF's Innovation Corps program, and more information can be found here. All Phase I awardees are strongly encouraged to participate in this activity. Costs that are allowable are limited to travel costs related to customer discovery (this could include costs associated with registration/attendance at events for the purpose of customer discovery) and salary/wages for team members who participated in the Boot Camp. All costs related to the Boot Camp must be in line with approved salary rates and other relevant Federal guidelines. International travel cannot be reimbursed, nor can any salary/wages for work done while outside of the United States. NSF recommends that, for the purposes of the proposal budget, applicants that plan to participate in this activity budget $10,000 and simply list this as "Boot Camp" costs in the budget justification.
{\bf Other Direct Costs:}
\begin{itemize}
\item Accounting: An amount of \$5k is allocated to hire a CPA accountant to prepare financial statements and perform financial viability assessment, and purchase project cost accounting system software.
\item An amount of \$10k is allocated to cover the costs of Dr Selfridge participating within the NSF's ``Beat-the-Odds Boot Camp'' to facilitate the customer discovery process.
\end{itemize}


%------------------------------
%Line I - Indirect Costs. Indirect costs are defined as costs that are necessary and appropriate for the operation of the business, but which are not specifically allocated to the NSF SBIR project. Specify the base and rate. Common indirect cost expenses include legal and accounting expenses, employee health insurance, fringe benefits, rent, and utilities. The following expenses will NOT be funded as part of the indirect cost pools, so any established indirect costs rates and calculations for a company should be reduced for the purposes of this proposal to exclude:
% - Independent research and development
% - Patent and patent related expenses will not be funded as either a direct or indirect cost
% - Sales and marketing expenses
% - Business development
% - Manufacturing and production expenses
{\bf Indirect Costs:} Indirect costs are waived for this project, to fully fund the VICON and CFD research efforts.


%------------------------------
%Line K - Small Business Fee. Up to seven percent (7%) of the total indirect and direct project costs may be requested as a fee. The fee is intended to be consistent with normal profit margins provided to profit-making firms for R&D work. The fee applies solely to the small business concern receiving the award and not to any other participant in the project. The fee is not a direct or indirect "cost" item and may be used by the small business concern for any purpose, including additional effort under the SBIR award (i.e., the "Prohibited Expenditures" list does not apply).
{\bf Business Fee:} The total direct and indirect costs sum to \$224,120.  A business fee of \$880 attains the maximum allotted amount of \$225,000, and represents 0.4\% of the total research program cost.


%------------------------------
{\bf Resumes for External Personnel} \\
The following pages contain resumes for Dr Willem AJ Annemat (DAR Corporation) and Dr James E Hubbard (StarLab).




%------------------------------
\newpage
{\bf \Large Willem AJ Anemaat, PhD}

United States Citizen \\
PhD in Aerospace Engineering - University of Kansas (KU) \\
DAR Corporation - Founder, Owner


%------------------------------
{\bf Biography} \\
Dr Willem AJ Anemaat is frequently a session chair for SAE, AIAA and ICAS conferences, and is a current member of the SAE WATC (Wichita Aviation Technology Conference and Exposition) Committee, where he served as the program technical chair in 2008.  Dr Anemaat has been a regular reviewer of airplane design related articles for the AIAA Journal of Aircraft and is a judge for AIAA Design competitions.  He was member of the AIAA Journal of Aircraft Editorial Advisory Board from 2004 until 2009, and then graduated to became an Associate Editor for the AIAA Journal of Aircraft dealing with Aircraft Design topics.  He is the Chair of the AIAA Aircraft Design Technical Committee, and is an AIAA Associate Fellow.  Dr Anemaat became member of the Kansas UAV Consortium Executive Board in 2005, is the recipient of the SAE 2010 Forest R. McFarland Award, was inducted into the University of Kansas Aerospace Engineering Honor Roll in 2017, and is a member of the KUAE Advisory Board, having served as Chair from 2016-2017.


%------------------------------
{\bf Authored Publications} \\
\vspace{-0.2in}

\PubSpace
Anemaat, Willem, D. van Dommelen, S. Johnson, P.B. Sargent, and W. Liu.
``Comparison of Aerodynamic Analysis Tools for Rotorcraft in Hover.''
\emph{AIAA Aerospace Sciences Meeting.}
Kissimmee, FL.
Jan, 2018.

\PubSpace
Anemaat, Willem, M. Schuurman, and W. Liu.
``Aerodynamic Design, Analysis and Testing of Propellers for Small Unmanned Aerial Vehicle.''
\emph{AIAA Aerospace Sciences Meeting.}
Grapevine, TX.
Jan, 2017.

\PubSpace
Anemaat, Willem.
``Smart Aircraft Modeler for Aircraft Conceptual Design.''
\emph{Collaboration in Aircraft Design (SCAD), Council of European Aerospace Societies (CEAS), Technical Committee Aircraft Design (TCAD), Research Section.}
Naples, Italy.
Oct, 2015.

\PubSpace
Anemaat, Willem, B. Kaushik, J. Carroll, and J. Jeffery.
``Software Tool Development to Improve the Airplane Preliminary Design Process.''
\emph{ISPE International Conference on Concurrent Engineering.}
Melbourne, Australia.
Sept, 2013.

\PubSpace
Anemaat, Willem, W. Liu, B. Kaushik, M. Yang, M. Brown, and R. Hale.
``Design, Build and Fly: NASA Lockheed P-3 Orion with External Antenna Fairings.''
\emph{AIAA Aerospace Sciences Meeting.}
Grapevine, TX.
Jan 2013.

\PubSpace
Anemaat, Willem, R.D. Hale, and N. Ramabadran.
``AAARaven: Knowledge-Based Aircraft Conceptual and Preliminary Design.''
\emph{AIAA/ASME/ASCE/AHS/ASC Structures, Structural Dynamics, and Materials Conference.}
Honolulu, HI.
Apr 2007.

\PubSpace
Anemaat, Willem, and B. Kaushik.
``Geometry Design Assistant for Airplane Preliminary Design.''
\emph{AIAA Aerospace Sciences Meeting.}
Orlando, FL.
Jan 2001.




%------------------------------
\newpage
{\bf \Large James E Hubbard, Jr, PhD}

United States Citizen \\
PhD in Mechanical Engineering - Massachusetts Institute of Technology (MIT)  \\
Langley Distinguished Professor - University of Maryland (UM)  \\
Samuel P Langley Professor - National Institute of Aerospace (NIA) \\
StarLab VICON Motion Capture Studio - Founder


%------------------------------
{\bf Biography} \\
Dr James E Hubbard's career has spanned some 20 years and includes work in aero-acoustics for noise-control, adaptive structures, spatially distributed transducers and the extension of modern time domain control methodologies into the spatial domain for the real-time control of distributed systems.  His work has resulted in a dozen patents which have demonstrated the efficacy and practicality of the techniques that he and his students have developed over these years.  These techniques have been viewed as innovative and revolutionary by his colleagues, and to recognize of these accomplishments, the University of Maryland named Dr Hubbard the University of Maryland Langley Professor at the National Institute of Aerospace.


%------------------------------
{\bf Honors and Awards} \\
\vspace{-0.2in}

\PubSpace
2016 - Inducted into the National Academy of Engineering for advances in the modeling, design, analyses, and application of adaptive structures.

\PubSpace
2016 - Inducted into the Virginia Academy of Engineering, Science and Medicine.

\PubSpace
2016 - Society of Photonics and Instrumentation Engineers (SPIE) Smart Structures and Materials (SSM) Lifetime Achievement Award for those viewed as luminaries in the fields of SSM.

\PubSpace
2015 - Best Paper in Structures Award from the Adaptive Structures and Material Systems (ASMS) branch of the Aerospace Division of the American Society of Mechanical Engineers (ASME).

\PubSpace
2015 - Elected American Society of Mechanical Engineers (ASME) Fellow having made noteworthy invention, discovery or advancement in the state of the art as evidenced by publication of widely accepted materials, by receipt of major patents, or by having products or processes in the marketplace.

\PubSpace
2012 - Elected American Institute of Aeronautics and Astronautics (AIAA) Fellow, where this distinction is conferred upon those members of the Institute who have made notable and valuable contributions to the arts, sciences, or technology of aeronautics and astronautics.

\PubSpace
2011 - The International Society for Optical Engineering (SPIE) conferred the grade of Senior Member of the Society ``in recognition of significant achievements within the optics and photonics community.''

\PubSpace
2002 - In July was awarded ``The Mass High Tech'' Movers and Innovators award ``Recognized for shaping the rich entrepreneurial community of New England.''

\PubSpace
2002 - Recipient of the Black Engineer of the Year President's Award.  Recognizes high individual merits that have a broad effect on people in many disciplines, and represent high value to the society as a whole.  Goes to an individual of high personal achievements who has made a major impact on a company's products and profits and who has broad managerial reach.








%------------------------------------------------------------
\newpage
{\bf \Huge Current and Pending Support \\}
%------------------------------------------------------------
% Information in this module is collected so that reviewers have visibility into the potential availability of company personnel during the period of performance if awarded.

%Types of Support / Activities. For the PI and each of the senior personnel listed on line A or B of the budget, provide information regarding each of the following that could require effort during the proposed Phase I performance period, regardless of whether the person will receive a salary from the activity:
% - All current and pending support for ongoing projects and proposals (from any source), including continuing grants funding.
% - Proposals submitted. Note that concurrent submission of a proposal to other organizations will not influence its review by NSF.
% - Upcoming submissions.
% - The Phase I proposal being submitted – note that this is considered "pending" and therefore MUST appear in the Current and Pending Support form for each PI and senior personnel.

%Information Needed. For each listed item, please include the following information:
% - Name of sponsoring organization.
% - Total award amount (if already awarded) or expected award amount (if pending) for the entire award period covered (including indirect costs).
% - Title and performance period of the proposal or award.
% - Annual person-months (calendar months) devoted to the project by the PI or senior personnel.

{\bf Current Support:} None.  Air Force Phase I SBIR research just finished, no currently funded projects.

{\bf Pending Support:} There are three pending proposal submissions.  None have been awarded; they are all awaiting selection notification.

TURN Combustion Engine Embodiment  \\
\begin{tabular}{ll}
Organization: & Air Force        \\
Type:         & SBIR Phase II    \\
Amount:       & \$1,500,000      \\
Duration:     & 12 month         \\
Submitted:    & May 25th, 2018   \\
Period:       & 08/18 - 07/19    \\
\end{tabular}

TURN Persistently Enduring Flight  \\
\begin{tabular}{ll}
Organization: & DARPA            \\
Type:         & TTO BAA          \\
Amount:       & \$999,855        \\
Duration:     & 12 month         \\
Submitted:    & June 10th, 2018  \\
Period:       & 11/18 - 10/19    \\
\end{tabular}

TURN Small Mobile Remote Deployment  \\
\begin{tabular}{ll}
Organization: & NSF              \\
Type:         & SBIR Phase I     \\
Amount:       & \$225,000        \\
Duration:     & 6 month          \\
Submitted:    & July 10th, 2018  \\
Period:       & 02/19 - 08/19    \\
\end{tabular}

{\bf Upcoming Submissions:} None. No planned upcoming submissions.








%------------------------------------------------------------
\newpage
{\bf \Huge Collaborators and Other Affiliations}
%------------------------------------------------------------
% For the PI and each of the senior personnel, list all institutional affiliations (other employers, consulting relationships, officer/director/trustee roles, etc.) and collaborators (co-authors, scientific partners, student/advisor relationships) that have occurred in the last four years, using the instructions and spreadsheet template found at https://www.nsf.gov/bfa/dias/policy/coa.jsp. This document will not be viewable by reviewers, but will be used by NSF to help identify potential conflicts or bias in the selection of reviewers. Also see guidance in the PAPPG.

Not applicable; Dr Justin M Selfridge works exclusively for Gradient Consulting, LLC.  The only other past affiliation was with NASA Langley, where he worked within the Dynamic Systems and Control Branch between 2011-2017.








%------------------------------------------------------------
\newpage
{\bf \Huge Facilities and Equipment}
%------------------------------------------------------------
%Specify the availability and location of significant equipment, instrumentation, computers, and physical facilities necessary to complete the portion of the research that is to be carried out by the proposing firm in Phase I. Purchase of equipment is NOT permitted in a Phase I project. If the equipment, instrumentation, computers, and facilities for this research are not the property (owned or leased) of the proposing firm, include a statement signed by the owner or lessor which affirms the availability of these facilities for use in the proposed research, reasonable lease or rental costs for their use, and any other associated costs. Upload images of the scanned statements into this section.

The primary goal of this Phase I research is to take the existing \emph{Mark 1} TURN prototype, collect high fidelity flight data within a VICON motion capture studio, validate the multibody dynamic models through parameter identification, and assess the aerodynamic performance through a CFD analysis.  As such, very little is needed in the form of facilities and equipment.


%------------------------------
{\bf Hardware for Avionics Development:}
All the tools and equipment needed to build custom avionic hardware and concept prototype aircraft are available to Gradient Consulting.  This includes a prototyping workshop that has previously constructed numerous unconventional multirotor and fixed-wing platforms, and developed multiple custom fabricated avionics boards.  Equipment commonly utilized includes: soldering station, power supply, oscilloscopes, prototype boards, motors, props, servos, electronic components, and model aircraft building supplies.  More specialized equipment includes: a 3D printer, RF transceivers, dedicated GCS laptop computers, and RC radios.  Additionally, the PI is registered with the FAA and holds a valid Part 107 Remote Pilot Certification.


%------------------------------
{\bf Software for Analysis and Design:}
DAR Corp routinely employs Star-CCM+ and FlightStream software for CFD, FEA, and aeroelastic modeling an analysis.  Gradient relies heavily on the Matlab software package, for vehicle modeling and simulation, controls development, post processing flight data, and running parameter identification.  Finally, Gradient will purchase SolidWorks CAD modeling software to coordinate the modeling activities with DAR Corp.


%------------------------------
{\bf Motion Capture Studio:}
StarLab is a VICON motion capture studio, which is an indoor testing facility equipped with 44 infrared cameras that detect reflective markers placed on an object.  Position and orientation states are collected at each sample step, which are used to calculate all the linear velocities and angular rates, which completely describes the motion of each rigid body element.  Data is sampled at 500 Hz, with sub-millimeter resolution, which offers higher fidelity than the on-board sensors, and provides a localization solution within a GPS-denied environment.  Obtaining this high quality flight data is of paramount importance for implementing the parameter identification process.








%------------------------------------------------------------
\newpage
{\bf \Huge Data Management Plan (DMP)}
%------------------------------------------------------------
%Data Management Plan (required). Proposals MUST contain a supplementary document labeled "Data Management Plan (DMP)", which should include the statement, "All data generated in this SBIR Phase I project is considered proprietary." This single sentence is sufficient to fulfill the DMP requirement, but applicants may add more detail about how the resulting data will be managed if they desire.

All data generated in this SBIR Phase I project is considered proprietary.








%------------------------------------------------------------
\newpage
{\bf \Huge Mentoring Plan}
%------------------------------------------------------------
%Mentoring Plan (required if the budget includes subawards requesting funds for postdoctoral scholars). If a proposal requests funding to support post-doctoral scholars at a research institution (through a subaward), a Postdoctoral Mentoring Plan MUST be uploaded to the system. Describe only the mentoring activities that will be provided to all postdoctoral researchers supported by the project. See more information and instructions on this requirement in the PAPPG here.

Not applicable; no postdoctoral scholars are part of the proposed research.








%------------------------------------------------------------
\newpage
{\bf \Huge Supplementary Documents \\}
%------------------------------------------------------------
%The supplementary documents permitted in a Phase I proposal are limited to the following (if applicable). The Data Management Plan and Mentoring Plan have their own dedicated modules within the "Supplementary Documents" section of Fastlane. All the other items below, if included, should be uploaded in the "Other Supplementary Docs" section as a single PDF file. Please ignore the modules entitled "Project Summary with Special Characters", "GOALI - Industrial PI Confirmation Letter", and "RAISE - Program Officer Concurrence Emails".


%------------------------------
{\bf \Large Letters of Support}
%------------------------------
%Strongly recommended; no more than three letters. Letters of support act as an indication of market validation for the proposed innovation and add significant credibility to the proposed effort. Letters of support should demonstrate that the company has initiated dialogue with relevant stakeholders (potential customers, strategic partners or investors) for the proposed innovation and that a legitimate business opportunity may exist should the technology prove feasible. The letter(s) must contain affiliation and contact information for the signatory stakeholder. Letters and supporting documents from consultants and subcontractors (or any personnel identified in the Budget Justification) are NOT considered letters of support and instead should be included in the Budget Justification section.

Letters of Support are appended following this section.  \\


%------------------------------
{\bf \Large Company Commercialization History}
%------------------------------
%A Company Commercialization History is required for all proposers certifying receipt of previous Phase II awards from any Federal agency on the third page of the Cover Page in question # 11. The NSF Commercialization History Template MUST be used. All items must be addressed in the format outlined in this template. Changes to the NSF template, additional narratives and/or commercialization history documents from other agencies are not permitted.

Not applicable; no previous Phase II awards.  \\


%------------------------------
{\bf \Large Human Subjects Documentation}
%------------------------------
%Projects involving research with human subjects must ensure that subjects are protected from research risks in conformance with the relevant Federal policy known as the Common Rule (Federal Policy for the Protection of Human Subjects, 45 CFR 690).

Not applicable; no human subjects are part of the proposed research.  \\


%------------------------------
{\bf \Large Vertebrate Animals Documentation}
%------------------------------
%Any project proposing use of vertebrate animals for research or education shall comply with the Animal Welfare Act (7 USC 2131, et seq.) and the regulations promulgated thereunder by the Secretary of Agriculture (9 CFR 1 .1 -4.11) pertaining to the humane care, handling, and treatment of vertebrate animals held or used for research, teaching or other activities supported by Federal awards.

Not applicable; no vertebrate animals are part of the proposed research.  \\


%------------------------------
{\bf \Large Resubmission Change Description}
%------------------------------
%A declined proposal may be resubmitted, but only after it has undergone substantial revision.

Not applicable; this is the first proposal for this research topic.  \\
















\end{document}
















%%------------------------------------------------------------
%% Submission Summary (40 line max)
%%------------------------------------------------------------
%
%%Developing a unique unmanned aerial system (UAS), which is seeking a persistently enduring flight capability.  Such a vehicle collects enough solar energy during the day to remain aloft throughout the night, thereby eliminating the need to land for refueling.  Accomplishing this feat would enable a new breed of aircraft, called atmospheric satellites, which can provide the same services and functionality as our existing space network, but with some decided advantages.  Low signal latency can achieve broadband speed data transmission rates, station-keeping maintains continuous coverage over a target area, takeoff and landing permits scheduled maintenance and payload upgrades, and eliminating a launch platform vastly reduces program costs.  Conventional aircraft predominantly utilize a common tube-and-wing design approach, but this configuration has been optimized to its aerodynamic and structural limits.  This research considers a novel concept architecture which takes the best features of both glider and helicopter design methodologies, and minimizes their respective shortcomings.  The vehicle, named the Tethered Uni-Rotor Network (TURN), is a low-power system which utilizes centrifugal stiffening as a design element.  This permits high-camber thin-thickness airfoil profiles, that can attain lift-to-drag ratios three times greater than standard practice; and unlike traditional fixed-wing aircraft, the TURN system is capable of vertical takeoff and landing (VTOL).  Persistently enduring flight paired with a VTOL capability offers numerous advantages to both the DoD and the commercial sector.  Previous research, from an Air Force Phase I SBIR, built the first two prototypes within a spiral development program, which delivered custom avionics hardware and software, and helped validate early dynamic simulation models.  This proposed DARPA research will continue that effort with two more prototypes, which will validate the aerodynamic and structural models needed to design subsequent TURN embodiments with confidence.  The culmination of this project will conclude with a demonstrator seeking a world record flight endurance, and continue moving toward the aviation milestone of persistently enduring flight.
%
%
%%------------------------------------------------------------
%\newpage
%\section*{ADMINISTRATIVE}
%\addcontentsline{toc}{section}{ADMINISTRATIVE}
%%------------------------------------------------------------
%
%\pagenumbering{roman}
%\cfoot{\thepage}
%\newcommand{\forceindent}{\leavevmode{\parindent=1em\indent}}
%
%\begin{center}
%\textbf{\normalsize COVER SHEET}
%\end{center}
%\addcontentsline{toc}{subsection}{COVER SHEET}
%
%BAA Number: HR001117S0014  \\
%Title: Persistently Enduring Flight Utilizing A Tethered Uni-Rotor Network (TURN) System  \\
%Technical Area: Unmanned Aerial Systems  \\
%Lead Organization Submitting Proposal: Gradient Consulting, LLC  \\
%Type of Organization: Other Small Business  \\
%Other Team Members:  \\
%\forceindent  Saxon Remote Systems [Other Small Business]  \\
%\forceindent  Design, Analysis and Research Corporation (DAR Corp) [Other Small Business]  \\
%Technical Point of Contact:  \\
%\forceindent  Dr Justin M Selfridge  \\
%\forceindent  33 Shirley Rd  \\
%\forceindent  Newport News, VA 23601  \\
%\forceindent  Phone: (757) 416-8312  \\
%\forceindent  Email: jselfridge@gmail.com  \\
%Administrative Point of Contact: Same as TPOC  \\
%Total Funds Requested: \$999,855  \\
%Award Instrument Requested: Firm Fixed Price Grant  \\
%Place/Period of Performance: Newport News, VA; 07/01/2018-06/30/2019  \\
%Summary of Costs: \\
%\forceindent  Task 1: Multibody Dynamics Validation (\$47.5k)  \\
%\forceindent  Task 2: Aerodynamic Model Refinement (\$320.6k)  \\
%\forceindent  Task 3: Trade Studies and Optimizations (\$95.5k)  \\
%\forceindent  Task 4: Airfoil Structural Assessment (\$536.2k)  \\
%Defense Contract Management Agency (DCMA) Administration: Hampton, VA  \\
%Defense Contract Management Agency (DCMA) Audit Office: Hampton, VA  \\
%Date Prepared: June 2018  \\
%DUNS Number: 080481053  \\
%TIN Number: 47-3405111  \\
%Cage Code: 7SK65  \\
%Proposal Validity Period: 365 days  \\
%Affirmation of Existing SETA Support Contacts: None  \\
%Affirmation of Human Subject Research: None  \\
%Affirmation of Animal Research: None  \\
%Unique Capability by Government or Government-Funded
%Team Member: Not Applicable  \\
%Eligible Applicants: Not Applicable  \\
%
%\newpage
%\addcontentsline{toc}{subsection}{TABLE OF CONTENTS}
%\tableofcontents
%
%
%
%
%
%
%
%
%%------------------------------------------------------------
%\newpage
%\section{SUMMARY OF PROPOSAL}
%\pagenumbering{arabic}
%%------------------------------------------------------------
%
%Flight has always captured man's imagination, which is evidenced by the great variety of aerial vehicles that exist today.  Everything from fixed-wing to rotorcraft; satellites to spaceships; mono-wing to quadrotor.  However, despite the wide variety of existing flying vehicles, not one of them has attained persistently enduring flight.  Accomplishing this aviation milestone is one of the great challenges still facing the aerospace community.
%
%
%
%
%%------------------------------
%\subsection{Innovation}
%%------------------------------
%
%This project presents a novel persistently enduring unmanned aerial system (UAS).  Conventional aircraft predominantly utilize a common tube-and-wing design approach, but this configuration has been optimized to its aerodynamic and structural limits.  Conversely, this research considers a unique concept architecture which takes the best features of both glider and helicopter design methodologies, and minimizes their respective shortcomings.  The vehicle, named the Tethered Uni-Rotor Network (TURN), is a low-power system which utilizes centrifugal stiffening as a design element.  This permits high-camber thin-thickness profiles, that can attain lift-to-drag ratios three times greater than standard airfoils; and unlike traditional fixed-wing aircraft, the TURN system is capable of vertical takeoff and landing (VTOL).  Greatly extended flight endurance paired with a VTOL capability offers numerous advantages to both the DoD and the commercial industry.  Air Force sponsored Phase I SBIR research built the first two prototypes within a spiral development program, which delivered custom avionics hardware and software, and helped validate early simulation models.  Proposed DARPA research will fund the development of the next two prototype installments, which address aerodynamic model validation and structural considerations within a custom designed airfoil.  The project will culminate with a demonstration that is anticipated to set a new world record for flight endurance, and further progress toward persistently enduring flight.
%
%
%
%
%%------------------------------
%\subsection{Results}
%%------------------------------
%
%Centrifugal stiffening within the TURN concept helps minimize vehicle weight while maintaining a robust structural configuration, and allows for much more aerodynamically efficient airfoils to further reduce power consumption.  This vastly increases flight endurance over comparable fixed-wing aircraft, but still offers a VTOL capability.  Various scales of the TURN system would satisfy different mission requirements for both the DoD and the commercial sector.  First, consider a small-scale TURN system with electric propulsion.  While existing fixed-wing aircraft achieve a 90-minute flight endurance with a five pound payload, a TURN system of comparable size and weight could remain airborne for over six hours.  This would serve as a mobile remote ISR platform to aid ground troops, or could be used for precision agriculture and construction applications.  Second, a medium-scale vehicle utilizing a combustion engine, which is the focus of the Air Force SBIR research, supporting a 250 pound payload while drawing 2000 watts of power, would remain airborne for several weeks at a time.  This medium-altitude embodiment would reduce costs and improve operational logistics for ISR pattern-of-life missions, and could augment border patrol monitoring.  Third, the largest-scale TURN system, and the subject matter of this proposed DARPA research, is designed to attain a persistently enduring flight capability.  While operating within the stratosphere, the vehicle collects enough solar energy during the day to remain aloft throughout the night, thereby eliminating the need to land for refueling.  This capability would enable atmospheric satellites, which can augment/replace our existing space satellite network, but at a greatly reduced cost.  Furthermore, closer proximity to the earth allows for atmospheric weather monitoring, and reduces signal latency permitting broadband speed data transmission rates.  This could serve the DoD as an alternative position, navigation and timing (APNT) solution, or as a network relay station.  Meanwhile, Google and Facebook are both striving for this aviation milestone to bring the Internet to the most remote parts of the planet.
%
%
%
%
%%------------------------------
%\subsection{Technical Rationale}
%%------------------------------
%
%Many unique benefits arise from the TURN system.  Advantageous centrifugal stiffening is similar to traditional helicopters, except thin tether filaments replace wasted material with detrimental aerodynamic properties.  Outboard sections are free to pivot, so there is no adverse bending moment typically found at a wing root.  This permits much higher aspect ratio, increased lift-to-drag ratio, and lower thickness-to-chord ratio, than can be attained by a tube-and-wing aircraft.  This new approach dramatically lowers structural weight, greatly reduces drag, and provides a far more robust structure than existing designs.  Consider some of the system benefits in more detail.
%
%
%%------------------------------
%{\bf High Aspect Ratios:}
%High aspect ratio wings improve aerodynamic efficiency.  Most HALE aircraft increase the aspect ratio of a tube-and-wing design, but the wing needs extra material to reinforce a flimsy member.  Conversely, helicopter rotors utilize centrifugal stiffening within very thin and slender structures.  This strategy used within the TURN design permits high aspect ratios within very thin and lightweight airfoil structures.
%
%
%%------------------------------
%{\bf Centrifugal Stiffening:}
%Centrifugal stiffening is one of the most desirable attributes since very little material is needed within the spar.  The TURN design places large portions of the vehicle mass at the wingtip, which both reduces angular rate and increases structural stiffening.
%
%
%%------------------------------
%{\bf Tether:}
%Helicopter rotors are inefficient.  While outermost portions generate lift, the inner sections are ineffectual, but still subjected to induced drag.  With this concept, the tether is 80\% of the total radius, which eliminates an underutilized structure, and contributes less than 8\% to the total drag.
%
%
%%------------------------------
%{\bf High Lift-to-Drag Ratios:}
%Low bending moments permit very thin airfoil sections, which is especially important for aircraft operating at a low Reynolds number.  Each TURN satellite has a large radius of rotation, which produces an elliptic span loading distribution across the wing, and a large spatial separation, so each airfoil operates independently of other downwash fields.
%
%
%%------------------------------
%{\bf Rotating System:}
%Conventional aircraft designs experience substantial parasitic drag from bulky antenna or optic payloads.  Conversely, with the TURN concept, lift is generated by the satellites, which move independently from the stationary central hub.  Thus, large and bulky payload volumes are isolated from the cruise speed, and can be incorporated without similar drag penalties.
%
%
%
%
%%------------------------------
%\subsection{Technical Approach}
%%------------------------------
%
%The goal is to develop the set of tools and models to design a TURN system with confidence.  Spiral development is an iterative approach, where prototypes address the largest risk, simulation models are refined with flight data, and enhanced analytic tools aid with subsequent vehicle designs.
%
%
%%------------------------------
%{\bf Custom Avionics Platform [Mark 0]:}
%Before any type of TURN prototype system could be controlled, a custom designed flight control system (FCS) was needed.  This prototype was developed as a dynamic test bench which: field tested hardware, debugged software, characterized sensor noise, evaluated new state estimation algorithms, and assessed controller gain selection.
%
%
%%------------------------------
%{\bf Multibody Dynamics Validation [Mark 1]:}
%The TURN vehicle is a multibody system, with a complex set of intertwined dynamics.  This prototype established a testing platform to validate accuracy and identify limitations within the existing computer models and simulation routines.  Modularity was the emphasis of this prototype, to assess dynamics across several design variables.
%
%
%%------------------------------
%{\bf Aerodynamic Model Refinement [Mark 2]:}
%Aerodynamic efficiency has the strongest influence on aircraft dynamics and performance, so accurate models must be developed as early as possible.  The goal is to quickly build the next prototype system, develop a CFD model, run parameter identification on flight data, and then compare the analysis results against physical reality.
%
%
%%------------------------------
%{\bf Airfoil Structural Assessment [Mark 3]:}
%Centrifugal stiffening in the TURN system design abates flexibility within the very thin airfoil members, but structural attributes still need to be considered in greater detail.  The goal is to develop an FEA model, investigate bend and twist throughout the wingspan, and assess the influences of aeroelastic effects on the system dynamics.
%
%
%
%
%%------------------------------
%\subsection{Experience}
%%------------------------------
%
%Following describes preexisting work that directly pertains to the TURN research and development.
%
%
%%------------------------------
%{\bf PhD Research:}
%Modeled nonlinear dynamics within a numerical multibody simulation.  Presented the initial controls analysis, outlined the controller methodology, and introduced inner-loop and outer-loop control systems, which handle stabilization and navigation.  Multivariable adaptive control reduced cross-coupling between inputs and outputs and mitigated plant model uncertainties.
%
%
%%------------------------------
%{\bf Air Force Phase I SBIR:}
%Developed aerodynamic and structural models of the TURN system, through an Air Force Phase I SBIR.  Analysis shows persistently enduring flight is feasible for a wide range of latitudes and day-of-year.  Also, considered a combustion engine TURN embodiment, which operates at 15k feet and provides 2000W of power to a 250 lbf payload.  Additional SBIR funding will pursue a 30-day flight endurance utilizing a combustion engine TURN system.
%
%
%%------------------------------
%{\bf Biography:}
%Dr Justin M Selfridge completed his PhD in Electrical Engineering at the University of Virginia, developing the TURN concept UAS.  Research areas include: multivariable adaptive control, system identification, autonomous vehicles, custom avionics hardware development, and UAV design and prototype fabrication.  The past six years, Dr Selfridge worked within the Dynamic Systems and Controls Branch at NASA Langley, developing the controller for the \emph{GL-10}.  The past year he has worked full-time on the TURN system, under his company, Gradient Consulting, LLC.
%
%
%
%
%%------------------------------
%\subsection{Cost}
%%------------------------------
%
%The proposed research costs \$999,855, spans a 52-week period, and is broken out into four major activities.  Following is a brief summary of each, along with its associated timeline and expense.
%\begin{itemize}
%\item Multibody Dynamics Validation: Utilize VICON motion capture studio, collect flight data on the \emph{Mark 1} prototype, and validate the multibody dynamic simulation model [5wk\slash \$47.5k].
%\item Aerodynamic Model Refinement: Build the \emph{Mark 2} system, perform a CFD analysis, and refine the existing aerodynamic models with collected flight data [15wk\slash \$320.6k].
%\item Trade Studies and Optimizations: Conduct trade studies to evaluate cost/benefit scenarios, and perform engineering optimizations for various physical design attributes [8wk\slash \$95.5k].
%\item Airfoil Structural Assessment: Build the \emph{Mark 3} system, model it with FEA, and assess structural considerations within the slender custom designed airfoils [24wk\slash \$536.2k].
%\end{itemize}
%
%
%
%
%%------------------------------
%\subsection{Quad Chart}
%%------------------------------
%\vspace{-0.1cm}
%\begin{figure}[h!]
%\includegraphics
%[ width=\Hwhole, height=\Wwhole, angle=90, origin=c, clip, trim={0 0 0 0} ]
%{../Figures/Quad.png}
%\end{figure}
%
%
%
%
%
%
%
%
%%------------------------------------------------------------
%\newpage
%\section{DETAILED PROPOSAL INFORMATION}
%%------------------------------------------------------------
%
%Atmospheric satellites are a new breed of aircraft designed to operate at altitudes around 60,000 feet, which offers air density and winds aloft that require the lowest power consumption while minimizing weather concerns.  A single system can offer continuous coverage because it operates over relatively fixed locations; and it is fifty times closer than LEO satellites, so the signal latency is extremely low.  Cost of deployment and recovery is inexpensive because the aircraft is able to takeoff and land, so the total system cost is expected to be several orders of magnitude less than other satellite solutions, with the added benefit of upgrading and refurbishing satellite payloads as equipment ages.  At this time, existing HALE aircraft can only maintain flight for several weeks at a time, and lack the highly reliable and persistent capabilities that satellites offer.  To successfully compete with existing satellite technology, an operational atmospheric satellite product must demonstrate a minimum level of proficiency.  It must be capable of controlled flight, remain airborne indefinitely throughout the year, and only need to land for routine maintenance.  It needs a sufficiently large margin of error to operate at high latitudes, overcome seasonal wind, and accommodate solar flux variations.  It cannot utilize altitude energy storing, because weather effects hamper station-keeping, and line of sight is reduced at lower altitudes.  For these reasons, a new concept approach is needed to tackle the persistently enduring flight problem.
%
%
%
%
%%------------------------------------------------------------
%\subsection{Military Applications}
%%------------------------------------------------------------
%
%The DoD relies heavily on our existing network of space satellites, which provide surveillance, communications, and localization services that are critical to military operations.  Persistently enduring flight can provide that same functionality, but at reduced cost and with notable benefits.
%
%
%%------------------------------
%{\bf Persistent Surveillance and Find/Fix/Track Missions:}
%The DoD requires UAVs with increased endurance capabilities for ISR and FFT missions; which reduces the number of sorties and assets that must be launched, owned and maintained, and provides improved mission continuity within target areas.  Attaining persistently enduring flight will provide a revolutionary capability for DoD operations.  Such a vehicle could carry a variety of payloads providing Wide-Area Motion Imaging (WAMI), Signals Intelligence (SIGINT), Direction Finding (DF), Hyperspectral Imaging (HSI), and other ISR sensors, to feed theater commanders and ground teams critical intelligence data.  It could serve as a strategic node in a Layered Sensing construct, where a medium/high-altitude system delivers wide-area situational awareness to the commander, and provides cues for other sUAS platforms to provide closer proximity ISR or target tracking.
%
%
%%------------------------------
%{\bf GPS-Disabled Alternative:}
%Future conflicts with near-peer adversaries must be ready for jamming and possible destruction of communication and navigation systems, such as GPS, which the DoD relies on for networked warfare and precision targeting.  The new Space Enterprise Vision plans for disaggregated, resilient constellations to replenish our comm/nav capabilities.  Based on large numbers of SmallSats, this will require responsive space launch provided by both DoD and commercial capabilities, such as XS-1 and Falcon 9.  However, launching sufficient quantities of SmallSat systems to fully reconstitute existing capabilities may take many months.  Recent studies by AFRL show that UAS could augment space-based assets and help reconstitute comm/nav capabilities more quickly than relying on space launch alone.  These loitering UAS would provide regional communication relays and alternate position navigation and timing (APNT) solutions.  Persistent endurance platforms, which can hold significant payloads, will be extremely enabling.
%
%
%
%
%%------------------------------
%\subsection{Current Research Efforts and Their Limitations}
%%------------------------------
%
%Several commercial and government research groups are attempting to attain persistently enduring flight.  The problem reduces down to an energy balance relationship, where solar energy collected during the day must be sufficient to sustain the system throughout the night.  Some approaches are investigating balloon and airship embodiments, but lightweight tube-and-wing structures dominate the majority of current design paths.  Several variants of wing-tail configurations have been devised, but each concept maintains a similar structural approach, which places as much weight as possible across the span to decrease bending moments on the wing root.
%
%
%%------------------------------
%{\bf Google X, Project Loon:}
%This concept utilizes a large network of balloons placed in the stratosphere which drift with the prevailing winds. Changing the altitude, offers limited directional control, by utilizing different wind speeds and directivity, but the balloons can still drift aimlessly over different countries. It is a one-way operation, where balloons are active for 50 to 100 days before requiring descent and payload pickup, so this is not a true persistently enduring flight solution.
%
%
%%------------------------------
%{\bf DARPA Projects:}
%DARPA investigated several concepts, including ISIS and Vulture, with a goal of staying aloft for years at a time.  ISIS was a Lighter-Than-Air (LTA) concept using enormous gas filled vehicles, which would be an ideal solution if winds aloft could be ignored.  However, the extremely large surface areas containing the gas, produced such high drag that the power requirements exceed a feasible solution.  Vulture researched fixed-wing concepts, utilizing ultra-lightweight structure techniques, with very low wing loading; which provided large surface areas for solar cells, and minimized both the cruise velocity and required power.  However, the structure was extremely delicate, had a relatively small payload, was extremely sensitive to winds, and had huge wing root bending moments throughout its thin wings.  Both DARPA programs were canceled due to a lack of feasibility while closing the required day-night energy balance.
%
%
%%------------------------------
%{\bf Commercial Efforts:}
%Several aerospace companies have been developing their own solutions.  Lockheed placed a bottom strut across fuselage elements for greater wing spar depth, but this approach does not provide significant improvement.  Aurora utilized a combination Z-wing arrangement, incorporating a hinge element which does not transmit bending moments, and allows each wing element to fly at angles for solar energy capture.  Boeing utilized an Outboard Horizontal Surface (OHS) to minimize the induced drag and augment the pitch trimming capability, but showed questionable feasibility, and was descoped to focus on energy storage and capture technologies.
%
%
%%------------------------------
%{\bf QinetiQ Zephyr:}
%The QinetiQ Zephyr 7 aircraft set the world record for HALE flight, which remained airborne for two weeks.  Zephyr is a tube-and-wing structure design, has a 74-foot wingspan, weighs 110 pounds, and carries a payload of less than five pounds.  Energy collected during the day was not sufficient for night operation, so it relied on its altitude as potential energy, while consuming power from the batteries.  The Zephyr is highly reflective of all the other aircraft concept approaches currently pursued and provides many indicators on future concept feasibility.
%
%
%
%
%%------------------------------
%\subsection{Radical New Concept Approach}
%%------------------------------
%
%The next section presents a novel UAS, which combines the best features of both gliders and helicopters, while minimizing their deficiencies.  Before introducing the concept, pros and cons of both gliders and helicopters are presented, to illustrate the benefits of this radical new approach.
%
%\begin{figure}
%\begin{floatrow}
%\hspace{-0.1cm}
%\ffigbox[\FBwidth]{
%\includegraphics 
%[ width=\Whalf, height=2.0in, clip, trim={0 0 0 0} ]
%{../Figures/Glider}}
%{\caption{Glider with High Aspect Ratio Wing}
%\label{Fig:Glider}}
%\hspace{-0.4cm}
%\ffigbox[\FBwidth]{
%\includegraphics 
%[ width=\Whalf, height=2.0in, clip, trim={0 0 0 0} ]
%{../Figures/Helicopter}}
%{\caption{Helicopter Rotors with Deflection}
%\label{Fig:Helicopter}}
%\end{floatrow}
%\end{figure}
%
%
%%------------------------------
%{\bf Glider Attributes:}
%Gliders are aerodynamically efficient aircraft, because they strive to minimize drag.  They typically employ long slender wings, as seen in Figure \ref{Fig:Glider}, which increases the aspect ratio, and is desirable from an aerodynamic drag perspective, but there are limitations with this approach.  Slender wings introduce flexibility, which can exhibit bend and twist during flight.  These wings are subjected to greater bending moments at the wing root, so added structural material must reinforce this connection point.  Finally, like all fixed-wing aircraft, gliders need forward velocity to generate airflow over the airfoil which produces lift, so they cannot hover or takeoff vertically.
%
%
%%------------------------------
%{\bf Helicopter Attributes:}
%Unlike fixed-wing aircraft, helicopters have desirable VTOL and hovering capabilities.  Typical rotor blades are so thin, that they bend and deflect under their own weight, demonstrated in Figure \ref{Fig:Helicopter}.  Yet, these flimsy structures lift substantially heavy vehicles, because centrifugal forces provide stiffening throughout the rotor element.  Unfortunately, helicopter rotors are not aerodynamically efficient, because they suffer from triangular span loading.  Outboard sections produce a majority of the lift, while the inboard sections are quite ineffectual.  Finally, rotor mechanisms are extremely complex and must withstand massive internal structural forces.
%
%
%%------------------------------
%{\bf Best of Both Worlds:}
%A completely new aerial concept vehicle incorporates features from both glider and helicopter methodologies.  Gliders utilize extremely efficient wings, but must mitigate flexibility and bending moments between the wing and the fuselage.  Helicopters support large payloads on flimsy rotors with centrifugal stiffening, but suffer from inefficient aerodynamic drag characteristics.  A new concept combines the best attributes of both types of aircraft, and minimizes the deficiencies of each, where centrifugal stiffening mitigates flexibility common to gliders, while replacing inefficient rotors with extremely aerodynamic, low drag, high aspect ratio airfoils.
%
%
%
%
%%------------------------------
%\subsection{Introducing the Tethered Uni-Rotor Network}
%%------------------------------
%
%The design objective reduced structural mass, maintained robust structural configuration, and simultaneously increased the lift-to-drag ratio.  A tensegrity approach minimized compression by putting structural loads in tension, where advanced composites provide incredible strength to weight.  The concept combines the best features of high efficiency glider and helicopter designs, reduces deficiencies within each class of aircraft, and yields better attributes than either can provide on their own.  This novel aerial vehicle, called the Tethered Uni-Rotor Network (TURN), illustrated in Figures \ref{Fig:TURN_Retract} and \ref{Fig:TURN_Extend}, utilizes a unique alternative approach, which exceeds existing capabilities, outpaces current research efforts, and could be the answer to persistently enduring flight.
%
%\begin{figure}
%\includegraphics
%[ width=\Wwhole, height=3.2in, clip, trim={0 0 0 0} ]
%{../Figures/TURN_Retract}
%\caption{Tethers are Retracted for Takeoff and Landing}
%\label{Fig:TURN_Retract}
%\end{figure}
%
%
%%------------------------------
%{\bf Vehicle Layout:}
%The TURN system has a central hub containing the vehicle payload.  Four small diameter cable tethers radiate outwards from the central hub and attach to satellite bodies.  Each has typical aircraft components, including: airfoil wing, propeller for thrust, stabilizers and control surfaces, and a fuselage for batteries, hardware, and sensors.  Each satellite resembles a flying-wing aircraft, which provides all the lift, propulsion and control for the TURN system.  A propeller is mounted on the leading edge of the outer wingtip, and vertical stabilizers and control surfaces are located directly in the prop wash.  The concept is named \emph{Tethered Uni-Rotor Network} because a network of aircraft systems are tethered together to form a much larger singular rotor system.
%
%
%%------------------------------
%{\bf Hover Operation:}
%The vehicle operates in a perpetual state of rotation, where each satellite drives the system rotation with its respective propulsion system.  This is opposite of a conventional helicopter, where a torque acts on a central shaft; with a tip driven system, no torque is transmitted back to the central hub.  As the system spins, centrifugal forces keep the tethers taught, and mitigate moments common within thin wings.  Lift is generated on each of the winged airfoil sections, which counteracts the weight of the satellite, and indirectly supports the weight of the central hub.
%
%\begin{figure}
%\includegraphics
%[ width=\Wwhole, height=3.2in, clip, trim={0 0 0 0} ]
%{../Figures/TURN_Extend}
%\caption{Normal Flight Operation with Tethers Fully Extended}
%\label{Fig:TURN_Extend}
%\end{figure}
%
%
%%------------------------------
%{\bf Flight Operation:}
%Individual satellites are controlled through their propeller and control surfaces, and the central hub is controlled by coordinating the tether forces imparted from the satellite vehicles.  Two types of translation are considered, vertical and horizontal, which have a parallel in helicopter terminology as collective and cyclic commands.  Each type of translation has two associated control modes which can be implemented through different control inputs.
%
%
%%------------------------------
%{\bf Vertical Translation:}
%Collective commands alter each satellite control input in unison.  Adjusting throttle increases or decreases the velocity of the satellite, and thus the angular rate of the TURN system.  This changes the airflow across the wing, which increases or decreases the total lift generated, and causes the vehicle to ascend or descend.  Alternatively, adjusting the pitch of each satellite through the elevator control surface, causes each satellite to nose up or down, thus the entire system will climb or fall as each satellite moves through a spiral trajectory.
%
%
%%------------------------------
%{\bf Horizontal Translation:}
%Cyclic commands alter the control inputs in a sinusoidal fashion throughout the rotation.  The first cyclic command, applied to the rudder control surface, manipulates the radial tension on the tether.  At one point in the rotation there is maximum tension, and exactly opposite there is a minimum tension; this imbalance induces horizontal translation.  Another approach uses cyclic throttle commands, such that throughout each rotation cycle, each wing passes through advancing and retreating blade conditions, which moves the central hub laterally.
%
%
%%------------------------------
%{\bf Vertical Takeoff and Landing:}
%The TURN system begins at rest on the tarmac with retracted tethers, as depicted in Figure \ref{Fig:TURN_Retract}.  Each wing is initially secured to the central hub with a docking mechanism which permits roll and pitch articulation within each satellite wing.  As the propellers ramp up, the TURN system begins to spin, while rolling on small wheels embedded within the vertical stabilizers.  Once sufficient angular rate has been attained, the system induces liftoff and rises above any local obstructions.  At a suitable altitude (couple hundred feet), each of the satellites are released from their docking mechanisms, and the tethers are extended to their normal flight operation.  Landing follows an identical procedure in the reverse order.
%
%
%
%
%%------------------------------
%\subsection{System Benefits}
%%------------------------------
%
%Many unique benefits arise from the TURN system.  Advantageous centrifugal stiffening is similar to traditional helicopters, except thin tether filaments replace wasted material with detrimental aerodynamic properties.  Outboard sections are free to pivot, so there is no adverse bending moment typically found at a wing root.  This permits much higher aspect ratio, increased lift-to-drag ratio, and lower thickness-to-chord ratio, than can be attained by a tube-and-wing aircraft.  This new approach dramatically lowers structural weight, greatly reduces drag, and provides a far more robust structure than existing designs.  Consider some of the system benefits in more detail.
%
%
%%------------------------------
%{\bf Centrifugal Stiffening:}
%Thin airfoils are extremely susceptible to bending moments.  Centrifugal stiffening within the TURN system is a very desirable attribute, since very little material is needed within the spar.  The NASA/AeroVironment Helios, shown in Figure \ref{Fig:Helios}, is a HALE aircraft experiencing very high bending moments across a large wing span, which ultimately failed from a lack of structural stiffness.  Since the TURN design places a majority of the mass at the largest possible radius, even low wing speeds achieve beneficial structural stiffening.  While utilizing a large tip mass, the TURN system only experiences a few g's of radial acceleration.
%
%\begin{figure}
%\captionsetup{width=0.9\textwidth}
%\centering
%\begin{minipage}{0.60\textwidth}
%\captionsetup{width=0.9\textwidth}
%\centering
%\includegraphics
%[ width=1.01\linewidth, height=2.4in, clip, trim={0 0 0 0} ]
%{../Figures/Helios_Flight}
%\end{minipage}%
%\begin{minipage}{0.40\textwidth}
%\captionsetup{width=0.9\textwidth}
%\centering
%\includegraphics
%[ width=1.0\linewidth, height=2.4in, clip, trim={120 0 0 0} ]
%{../Figures/Helios_Break}
%\end{minipage}
%\caption{Helios Aircraft with Severe Bending Moment and Failure From a Lack of Stiffness}
%\label{Fig:Helios}
%\end{figure}
%
%
%%------------------------------
%{\bf High Aspect Ratios:}
%High aspect ratio wings improve aerodynamic efficiency, but the wing weight fraction is another closely related performance indicator.  Typical aircraft wings are commonly 10-20\% of the gross aircraft weight, and rarely exceed aspect ratios of ten.  Most HALE aircraft increase the aspect ratio of a tube-and-wing design, while pushing the limits of exotic structural materials.  While targeting aspect ratios of 20-25, they all experience similar wing weight fractions around 40-50\% of the total aircraft, indicating the wing receives extra material to reinforce a flimsy structure.  Conversely, most rotorcraft blades have aspect ratios of 40-50 and typically only represent 2-3\% of the gross helicopter weight.  Clearly, centrifugally stiffened thin structures can substantially reduce weight while using greater aspect ratios.  This approach used in the TURN concept permits high aspect ratios within very thin and lightweight airfoil structures.
%
%
%%------------------------------
%{\bf Tether:}
%Helicopter rotors are inefficient, because only the outermost portions generate lift.  The inner sections are quite ineffectual, but still subjected to induced drag.  With this concept, the tether is 80\% of the total radius, which eliminates an underutilized structure.  At first glance, the tether may appear to significantly increase the parasitic drag; however, it is made of advanced aerospace grade compressed Spectra, which provides incredible strength-to-weight, so even a small tether cable diameter is structurally efficient at carrying tensile loads.  Because the average tether speed is only 40\% of the tip speed, the tether drag only constitutes around 5-10\% of the total system drag.
%
%
%%------------------------------
%{\bf High Lift-to-Drag Ratios:}
%Persistently enduring flight must resolve the conflict between the forces needed to keep the system aloft versus the power required to do so.  Thus, a high lift-to-drag ratio is extremely important.  Low bending moments permit very thin airfoil sections with low thickness-to-chord ratios, which is especially important for aircraft operating at a low Reynolds number.  This maintains laminar flow, which achieves extremely low levels of induced and parasitic drag.  With this concept, the tether comprises the innermost 80\% of the arm, so the satellites have a large radius of rotation which enables the wings to achieve elliptic span loading, rather than an inefficient triangular load distribution found on helicopters.  Helicopter rotors perform in close proximity of the downwash of the blade rotating in front of it, which increases the required angle of attack, and leads to high induced drag.  Each TURN satellite has a large spatial separation, so they operate independently of other downwash fields.  Because total rotor disc area is directly related to the induced drag, the TURN approach offers a distinct advantage over conventional rotor designs.
%
%
%%------------------------------
%{\bf Rotating System:}
%Conventional aircraft designs require that payload volumes move at the same speed as the rest of the vehicle, which results in substantial parasitic drag from bulky antenna or optic payloads.  Conversely, all the lift from the TURN concept is generated within the moving satellites, which are independent from the stationary hub.  This means larger and aerodynamically inefficient payload volumes can be incorporated into the central hub without incurring similar drag penalties, because the payload is isolated from the cruise speed components which generate lift.
%
%
%
%
%%------------------------------
%\subsection{Milestones within Spiral Development Plan}
%%------------------------------
%
%Spiral development is a fast and economical approach to new aircraft design.  Traditional aircraft design, utilized by large tier-one aerospace companies, follows a linear progression.  But this requires large upfront costs in both time and money, and does not mitigate the largest program risks early within the new concept design process.  Rather than developing a single design, spiral development works iteratively.  Inexpensive prototypes, with fast development periods, are designed to address the most substantial risk.  Then, flight data is collected on the system, which helps refine the simulation models.  Finally, increased confidence in analytic tools aids with subsequent vehicle design.  This process repeats until the final product is complete.
%
%
%%------------------------------
%{\bf Current Status of the TURN Project:}
%Research and development for the TURN concept is well underway.  While working as an independent contractor for NASA Langley, Justin M Selfridge (PI for this proposal) completed a PhD which developed the simulation models and controls architecture for the TURN system.  The dissertation presented a unique multibody dynamic simulation program which models the nonlinear system dynamics within a modular matrix framework.  Then it developed several model reference adaptive control (MRAC) architectures for the various control loops, and utilized an LDS decomposition of the high-frequency gain matrix to relax \emph{a priori} knowledge of the plant structure.  Following graduation, the project was awarded a Phase I SBIR research grant sponsored by the Air Force; so the past year Dr Selfridge has been researching the TURN concept full-time while working under his company Gradient Consulting, LLC.  That research investigated a MALE UAS utilizing an internal combustion engine, capable of providing 2000 watts of power to a 250 pound payload, while accommodating 50 knot winds aloft.  Phase I research developed the aerodynamic and structural models to predict power requirements and flight endurance.  Results from the analysis indicate that an internal combustion engine TURN embodiment will attain a 30-day mission endurance.  Built and tested the first two prototype embodiments, described in greater detail within the next section, and developed a novel attitude heading and reference system (AHRS) algorithm to account for the rotating nature of the TURN system.
%
%
%%------------------------------
%{\bf Summary of Prototype Vehicles:}
%Several prototype vehicles incrementally proceed onto the next most risky endeavor, which needs to be addressed during the development process.  The spiral development plan describes the two prototypes already completed during the Air Force Phase I research effort, and the next two prototypes scheduled for this proposed DARPA funded project.
%
%\vspace{-0.4cm}
%\begin{center}
%\begin{tabular}{ cccccc }
%\hline \hline
%Prototype &
%Span (ft) &
%AR (-) &
%Sat Wt (lb) &
%Rad (ft) &
%Time (mo) \\
%\hline
%Mark 0 &  3 &  5 &  0.5 &  10 &  1 \\
%Mark 1 &  3 & 12 &    3 &  15 &  2 \\
%Mark 2 & 12 & 20 &   10 &  75 &  4 \\
%Mark 3 & 35 & 30 &   25 & 175 &  8 \\
%\hline
%\end{tabular}
%\end{center}
%
%
%%------------------------------
%{\bf Mark 0: Custom Avionics Platform [AF Phase I]}
%
%\emph{Greatest Risk:}
%Before any type of TURN prototype system could be controlled, a custom designed flight control system (FCS) needed to be developed.  The avionics system managed the local stabilization of each satellite, communicated with each of the entities and a ground control station (GCS), and articulated the coordinated maneuvers between each tether arm.
%
%\emph{Primary Objective:}
%This prototype system served as a dynamic test bench to: field test hardware, debug software, characterize sensor noise, dry run new algorithms, and assess controller gain settings.  It was not attempting to evaluate or study any specific attribute of the TURN performance; rather, it was the first necessary step before any other prototypes could be operated.
%
%\emph{Design Considerations:}
%A small commercial-off-the-shelf (COTS) acrobatic remote-controlled airplane, was the fastest and simplest option.  It provided a stable platform with docile dynamics, which were ideal while testing the new FCS.  Foam construction is robust and forgiving, which was needed for early flight testing development.  The COTS system, the \emph{Malibu III} from TechOne, took the guesswork out of the initial design, and its size was suitable for indoor flight testing.
%
%
%%------------------------------
%{\bf Mark 1: Multibody Dynamics Validation [AF Phase I]}
%
%\emph{Greatest Risk:}
%The TURN architecture is a multibody system, with a complex set of intertwined dynamics.  Simulations from the PhD research provide a very flexible framework for modeling such a system, and were used extensively during the controller development and analysis.  However, the simulation models have not yet been validated by testing them against a physical system.
%
%\emph{Primary Objective:}
%The goal of this prototype was not to seek out an optimized solution for a single TURN embodiment.  Rather, it was to establish a testing platform to validate the accuracy and to identify limitations within the existing computer models and simulation routines.  From the completion of this activity, the prototype can be tested within a VICON motion capture studio, to assist with the model validation process.
%
%\emph{Design Considerations:}
%Validating simulation models required a wide range of data across several design variables; thus, modularity was the emphasis of this prototype.  Rather than spend time on custom composite work, the aircraft was assembled from cutouts of carbon fiber sheets.  Sizing and placement of key aerodynamic surfaces can be quickly and easily modified to collect a wider range of data.  The aspect ratios vary between 10-14, and additional battery mass increases weight to 3 lbf/satellite.  This increases centrifugal forces and angular rate from the previous model, but the scale is still feasible for indoor flight testing.
%
%
%%------------------------------
%{\bf Mark 2: Aerodynamic Model Refinement [DARPA]}
%
%\emph{Greatest Risk:}
%Aerodynamic performance is the key feature for any low-power aircraft.  As one of the most difficult design considerations to perfect, it typically requires the longest lead time to see through to completion.  With a strong influence on aircraft dynamics and performance, accurate and validated models must be developed as early as possible to aid with future TURN designs.
%
%\emph{Primary Objective:}
%This prototype will help to identify the aerodynamic features that are unique to the TURN architecture.  The intention is still not to seek out a single optimized design, but to take a physical system with representative geometry, simulate it within a computational fluid dynamic (CFD) model, and assess the accuracy and discrepancies between the model and physical reality.
%
%\emph{Design Considerations:}
%Speed of execution is the driving force behind this research project.  Rather than spend time developing a custom single-use composite prototype, a commercially available 12-foot wing structure from a UAV manufacturer will be modified to incorporate the custom components that are representative of the TURN configuration, and will weigh about 10 lbf/satellite.
%
%
%%------------------------------
%{\bf Mark 3: Airfoil Structural Assessment [DARPA]}
%
%\emph{Greatest Risk:}
%With operational hardware and validated software models, the next largest risk addresses the unique considerations while designing and building a TURN aircraft.  The airfoil profile and structural features within the TURN layout allow geometries that are not typical within the aviation industry, so manufacturing techniques need to be investigated and tested.
%
%\emph{Primary Objective:}
%The goal of this prototype is to focus on the design of a custom airfoil profile and assess its structural performance.  Rather, than working with existing parts for rapid prototype assembly, this iteration will help uncover any potential pitfalls or difficulties while constructing a finished product.  This also designates the incorporation of solar cells into each new prototype.
%
%\emph{Design Considerations:}
%This TURN prototype will not push the limits of the design, but it will incorporate more aggressive dimensions than previous embodiments.  While it is unlikely to attain an eternal flight capability, as the first prototype with custom designed features, this system should seek to set a world record flight endurance while utilizing solar cells in a favorable testing location.  An aspect ratio of 30 is comparable to what common gliders attain without the aid of centrifugal stiffening.  It has an expected wingspan of around 35 feet and weighs about 25 lbf/satellite.
%
%
%
%
%%------------------------------
%\subsection{Research Partners}
%%------------------------------
%
%Developing the TURN aircraft requires a multidisciplinary effort, which will utilize three specialized research groups across several domains.  Gradient Consulting focuses on nonlinear dynamic simulations and controls development, Saxon Remote Systems is tasked with the prototype design and manufacturing, and DAR Corporation is responsible for developing the CFD and FEA models, conducting engineering trade studies, and performing aerodynamic optimizations.
%
%
%%------------------------------
%{\bf Gradient Consulting [Dynamic Simulations and Controls Development]:}
%The crux of the spiral development cycle resides within the model validation process.  Matching computer simulations to their physical counterparts, and identifying and assessing the discrepancies, increases confidence within the analytic tools utilized during subsequent prototype development.  Gradient Consulting specializes in parameter identification through multisine signal injection, modeling nonlinear dynamics of multibody systems, and implementing advanced MRAC architectures to mitigate modeling uncertainty and environmental disturbances.  The first task, assists with the \emph{Mark 2} development, by running multibody dynamic simulations to project proper sizing and layout.  The next task, post processes the the \emph{Mark 2} flight data, and obtains the steady-state equilibrium operating points and the transient dynamic system response characteristics, which are used to assign proper gains within the control law.  The third task, in conjunction with other research efforts, will deliver the controller architecture and contribute toward the overall design and flight testing program for the \emph{Mark 3} prototype system.  Aside from the specific contributions to the prototyping efforts, Gradient will also oversee the project management of the DARPA research effort.
%
%
%%------------------------------
%{\bf Saxon Remote Systems [Prototype Design and Manufacturing]:}
%High quality composite work, especially for new prototype aircraft, is a skill that takes time to master.  Utilizing a UAV manufacturer that is knowledgeable about the intricacies of layout design and the specifics of manufacturing, will expedite the spiral development process.  Saxon Remote Systems has a full line of UAV platforms, has a team of engineers specialized in UAV design, and engages in all aspects of prototyping, manufacturing, and flight testing experimental concepts.  Their first task will leverage one of their existing airfoils, slightly modify the design to form the \emph{Mark 2} system, and then flight test the vehicle.  Their second task will layout the custom design for the \emph{Mark 3} system, and offer their engineering experience to avoid common manufacturing pitfalls.  Their third task, late in the research timeline, will construct and flight test the \emph{Mark 3} prototype.  Each of these three tasks will interact with the other research activities over time.  Leveraging the experience and capabilities of Saxon, not only advances the current research effort, but they are also equipped to handle the production and delivery of completed commercial TURN aircraft.
%
%
%%------------------------------
%{\bf DAR Corporation [CFD/Optimizations/Trade Studies]:}
%For a low-power persistently enduring aircraft, aerodynamic efficiency and structural robustness are of paramount importance.  Unfortunately, they are also the most difficult tasks to model properly.  Realizing the full potential of the TURN methodology, requires slender wings with high aspect ratios, large tip masses to lower the system rotation rate, and unconventional control surface placement; thus, commonly used ``best-practices'' in aircraft design are not likely sufficient to accurately depict the flight performance.  As such, developing computational fluid dynamic models, optimizing aerodynamic features, and performing engineering trade studies, is the primary goal of the next prototype installments.  This will identify discrepancies and ultimately validate the models, which are then used as a design tool to optimize the aerodynamic performance of the next TURN system.  DAR Corporation is a leading industry expert in small aircraft analysis and design, has a proven track record for aeroelastic research, and their experience with propeller and wind turbine systems is extremely well suited for the rotating aspect of the TURN concept.  Their first activity, early within the research timeline, models \emph{Mark 2} prototype aerodynamics, and identifies forces and moments produced under various flight conditions.  Their second activity, midway through the research project, performs optimization improve the aerodynamic characteristics, and predict potential issues with flutter and vibrations.  After flight testing the \emph{Mark 3}, their final task compares analytic predictions to actual collected flight data from the system to aid with engineering trade studies.  Confidence within CFD models and an ability to optimize the layout, will provide a foundation for subsequent design.
%
%
%
%
%%------------------------------
%\subsection{Technical Objectives}
%%------------------------------
%
%While designing conventional tube-and-wing aircraft, there is an extensive number of existing tools and historical trends which aid the development process.  Conversely, the novel architecture of the TURN concept utilizes unique features that are not common within standard aircraft designs.  While fundamental aircraft design precepts lay the foundation for the TURN design process, those models and methods are currently untested pertaining to the specifics of the TURN approach.  Thus, the goal of the spiral development process is to establish the set of analytic tools and simulation models needed to design a TURN system with confidence.  Gradient will build upon the successes and lessons learned from the Air Force Phase I research program, and continue to expand upon and validate the models required to achieve persistently enduring flight.
%
%The technical objectives of the DARPA research program are to:
%\begin{enumerate}
%\item Use the \emph{Mark 1} prototype to validate the multibody dynamics simulation model.
%\item Use the \emph{Mark 2} prototype to refine the computational fluid dynamic model.
%\item Use the \emph{Mark 3} prototype to assess custom airfoil structural performance.
%\item Exceed a 14-day flight endurance with the \emph{Mark 3} prototype system.
%\end{enumerate}
%
%Specific questions to be addressed within the research program include:
%\begin{itemize}
%\item How accurate is parameter identification for obtaining steady state and transient response?
%\item What are the boundaries and limitations within the multibody dynamic model?
%\item Does a simulated control law match its physical embodiment?
%\item Does CFD provide an accurate representation of the forces/moments acting on the system?
%\item Can it predict the correct lift and drag at various operating points?
%\item Does it accurately depict prop wash flow interactions on other surfaces?
%\item Do the tether vibration predictions match physical phenomenon?
%\item Does the FEA model depict the bend, twist and flutter acting on the thin wing structures?
%\item Are there manufacturing limitations for the desired airfoil profile?
%\item Does the desired airfoil pressure distribution match the physical embodiment?
%\end{itemize}
%
%A detailed development program to achieve these objectives is described within the next section.
%
%
%
%
%%------------------------------
%\subsection{Proposed Work Plan}
%%------------------------------
%
%The research project spans a fifty-two (52) week duration, where key tasks and dates are provided in the table within the \emph{Project Management} section.  This section describes the VICON motion capture, introduces the next two prototype vehicles, discusses engineering trade studies, and outlines the aerodynamic optimizations.
%
%
%%------------------------------
%{\bf Primary Objective:}
%The overall objective is to develop the analytic tools and models needed to build a commercially viable TURN product.  The research continues the spiral development process, by constructing two more prototypes, which will provide all the model validation needed to design a full-scale TURN system.  The next prototype, named \emph{Mark 2}, will focus on validating CFD models which are needed to assess handling characteristics and predict flight endurance.  An emphasis is placed on fast implementation, so an existing airfoil from a UAV manufacturer is utilized to expedite the development process.  Flight data will be collected on the \emph{Mark 2}, which will be compared against and used to validate the CFD model.  The following prototype, the \emph{Mark 3}, will place an emphasis on the custom airfoil profile design and assess the structural features of this more aggressive geometry.  It will design and build the next largest TURN embodiment, investigate how bend and twist manifest within the wing, and demonstrate a 14-day flight endurance concluding this research project.  The outcome will provide: a validated multibody dynamic model, used for control algorithm simulations; an accurate CFD analysis, which serves as the primary design tool for future vehicle development; an understanding of how airfoil dimensions impact TURN structural characteristics; and all the tools and models to design a TURN system with confidence.
%
%
%%------------------------------
%{\bf Multibody Dynamics Validation:}
%A nonlinear multibody dynamic model and several adaptive control laws were devised from previous PhD research, and an avionics controller was developed during the Air Force Phase I effort.  The \emph{Mark 1} system has been built and performed initial flight testing, and is ready for more extensive VICON motion capture flight data.  This process spans the initial five (5) weeks of the project.
%
%\begin{figure}
%\centering
%\includegraphics
%[ width=5.8in, height=3.2in, clip, trim={0 0 0 0} ]
%{../Figures/Mark1.jpg}
%\caption{Modular \emph{Mark 1} Prototype System for Multibody Dynamic Model Validation}
%\label{Fig:Mark1}
%\end{figure}
%
%\emph{Objective:}
%The goal is to validate the nonlinear multibody dynamic simulation model, so it can be used as a design tool for subsequent vehicle iterations.  The process uses the \emph{Mark 1} prototype, displayed in Figure \ref{Fig:Mark1}, collects flight data across a wide range of operating points, iterates through different physical parameters, and then compares recorded flight data to the results produced by the simulation model.  To accomplish this task, VICON motion capture records system states, parameter identification yields the steady state and transient dynamics of the system, and a modular prototype framework is utilized to encompass a wide array of physical geometries.
%
%\emph{Motion Capture:}
%The VICON motion capture studio is an indoor testing facility equipped with 44 infrared cameras that detect reflective markers placed on an object.  Position and orientation states are collected at each sample step, which are used to calculate all the linear velocities and angular rates, which completely describes the motion of each rigid body element.  Data is sampled at 500 Hz, with sub-millimeter resolution, which offers higher fidelity than the on-board sensors, and provides a localization solution within a GPS-denied environment.  Obtaining this high quality flight data is of paramount importance for implementing the parameter identification process.
%
%\emph{Parameter Identification Process:}
%The simulation model must match physical reality in two different ways.  First, the trimmed steady state operating points are the set of states and control inputs which achieve a static equilibrium, such as hover or a constant rate of ascent.  Second, the transient dynamic response is described by the elements within the state space matrices, and are used for reference tracking and disturbance rejection.  Parameter identification algorithms run through the collected flight data and yield both the trimmed equilibrium operating points, and the state space dynamic model.  The state space form is populated with the partial derivatives of each state and control input with respect to the state derivatives, which mathematically describes the dynamic response of the system.  Similarly, the simulation model contains both trimming and linearization routines, which provide the state space representation that exactly matches the form provided by the parameter identification algorithms.  So, running multisine signal injection during the flight testing, and post processing the data via parameter identification, will directly convey any discrepancies between the physical flight characteristics and the simulated dynamics.  Validating the simulation across a wide range of operating conditions and physical parameters necessitates a modular prototype design.
%
%\emph{Elements of the Modular Design:}
%The \emph{Mark 1} system was designed as a modular testing platform.  Several attributes within the prototype are adjustable, and intended to collect key parameters which are needed to validate the multibody dynamic simulation model.  A total of six wing blades were constructed, each with a 40'' wingspan.  Four of the blades have aspect ratios of 12, and will be used together while demonstrating a complete flight capability.  The other two blades have aspect ratios of 10 and 14, which will be utilized for model validation purposes.  Modularity within the system includes: aspect ratio, prop diameter, elevator control surface area, and tail boom length.  Additionally, the battery location is adjustable, such that the CG of the satellite aligns with the quarter-chord position of the wing.
%\begin{itemize}
%\item Two prop diameters will be tested, which will assess throttle commands versus power consumption, and identify the control bandwidth available to maintain equal spacing and coordinate the satellites.
%\item There are three tail boom lengths to alter the moment arm distance between the elevator control surface and the CG of the satellite, which will evaluate the pitch dynamics within the simulation model.
%\item Testing two sizes of elevator control surface, which dictates the force generated from elevator deflections, and will evaluate the engineering tradeoff between drag and control bandwidth.
%\item Angle of attack will be evaluated at 3\degree, 5\degree, and 7\degree{} increments, which influences the required airspeed across the airfoil, and will help identify proper lift and drag coefficients across the operating region.
%\item Each blade has the same wingspan, but aspect ratio is evaluated at 10, 12, and 14, which is synonymous with adjusting the wing surface area, thus directly influencing the angular rate and centrifugal force within the TURN system, which will review the rotational aspects within the simulation model.
%\end{itemize}
%
%\emph{Additional Battery:}
%Each satellite can be flown with one or two battery packs.  The goal of adding a second battery is not to increase the flight endurance, rather it increases the weight of each satellite.  Since weight is directly counteracted with lift, this modification directly influences the angular rate and centrifugal force on each tether arm.  Time permitting, this option would provide another dimension to evaluate multiple physical parameters (weight/lift, drag/thrust, control bandwidth, rotational influences) within the simulation model.  Since all of these parameters can be obtained directly from the previous adjustments, this testing would serve as an optional redundancy check within the model validation process.
%
%\emph{Testing Process:}
%Altering propellers, tail boom lengths, and elevator surface areas are all easily adjustable, so testing each set of these parameters could be accomplished within a single day.  Then, proceeding through three different aspect ratios and three angle of attack settings, would require nine full days of VICON testing.  Repeating the full set of testing with a second battery configuration would double the testing duration to 18 days.  If needed, a smaller subset of testing with a second battery, may focus on operational areas that are most meaningful to the simulation validation.  To be safe, 20 days of VICON testing are allocated within the proposed budget.
%
%\emph{Anticipated Outcomes:}
%Any simulation model can be manipulated to fit a single operating condition, so evaluating a full range of physical parameters at different operating points will improve the level of confidence within this tool.  This activity yields elements within the state space matrices which represent the partial derivatives of the states and control inputs.  These values are needed to refine the accuracy of the dynamic models, and identify boundaries and limitations that need to be observed during the design process.  Since, the multibody simulation model is the primary means of sizing TURN components and evaluating controller gains, validating this tool is a crucial step prior to proceeding onto further prototype development.
%
%
%%------------------------------
%{\bf Aerodynamic Model Refinement Prototype:}
%Having validated the multibody dynamic model, the next most important design tool is the CFD analysis.  This effort quickly builds the next largest TURN system, the \emph{Mark 2} prototype, which is used to validate the aerodynamic models.  This research activity will target a fifteen (15) week duration, but allocates two weeks for flight testing delays associated with scheduling availability and inclement weather.
%
%\emph{Objective:}
%The goal is to build a scaled TURN system, as quickly and inexpensively as possible, which is conducive for validating CFD models.  Similar to the previous activity, the CFD analysis should provide insight into larger TURN designs.  However the accuracy of such an analysis, specifically addressing the rotating aspect of the system, has not yet been assessed.  During this prototyping period, the \emph{Mark 2} will be developed and flight tested.  To expedite the prototyping effort, it will utilize an existing airfoil, and incorporate custom modifications to represent the TURN layout.  This prototype is not intended to deliver an optimized design; rather it expedites the development process by incorporating a readily available component, modifies it to a TURN geometry, and then collects enough data to correlate the CFD models.
%
%\emph{Prototype Manufacturing:}
%Speed of development is of utmost importance for this research project.  As such, rather than spend time on a custom airfoil design for a single-use airframe, an expedited development process is employed.  Saxon Remote Systems has existing composite molds which are used for their \emph{Viper M-10} fixed-wing drone.  With a 12-foot wingspan, a modular construction, and suitable sizing for the prop/motor/battery components, this system is an ideal candidate for the next TURN prototype embodiment.  Furthermore, having existing molds, and knowledgeable personnel familiar with composite manufacturing techniques, will vastly expedite the research program.  Saxon will develop an outboard wingtip fuselage to accurately represent the TURN satellite, and manufacture a central hub with the TURN tether retracting mechanism.  For testing purposes, the central hub will accommodate variable payload weights, which will evaluate how additional mass influences the steady state operating points.  A small amount of custom design work resides within the elevator sizing and placement, which will be evaluated with the currently developed dynamic model, and further refined after VICON testing has been completed.
%
%\emph{CFD Model Development:}
%Parallel to the \emph{Mark 2} fabrication process, the prototype design specifications and physical parameters will be delivered to DAR Corporation to begin the CFD modeling effort.  The goal is to develop an analytic tool that depicts accurate force and moment values acting at the CG of each rigid body element, while operating at different flight conditions.  A complete TURN system and its rotation will be depicted within the simulation model.  An initial baseline configuration will obtain states and control inputs that achieve steady-state equilibrium, which is synonymous with the hover operating condition.  From there, a number of parameters will be independently adjusted away from the trim condition, which yield a new set of forces and moments.  Because these points are no longer trimmed values, the forces and moments will influence the dynamics of the system and alter its operating state.  Adjustable parameters under investigation include: throttle input, elevator input, airfoil angle of attack, and airflow velocity over the wing.  For each of the four parameters, two points above and two points below the trimmed operating point will be selected.  A complete set of forces and moments at various operating conditions provides an array of values that can be matched against the physical flight data.
%
%\emph{Taxi Tests:}
%Testing will proceed in an incremental manner, and begin with taxi tests prior to unconstrained flights.  The system begins at rest on the tarmac, and then the motors ramp up to induce the system rotation, just like a standard takeoff procedure.  However, rather than initiating takeoff with tethers still retracted, the prototype will let out the tethers while the system is still grounded and spinning on the runway.  An initial assessment will confirm equal spacing is maintained between each of the satellites.  Next, a small amount of collective input is commanded, such that each satellite slightly lifts off in unison.  Finally, small cyclic commands will sinusoidally adjust throttle inputs, which will translate the central hub across tarmac surface.  For both collective and cyclic testing, equal spacing between satellites must still be observed.  The objective is to ensure all control inputs function properly, and gain values invoke the desired system response.
%
%\emph{Flight Testing Procedure:}
%Following the taxi tests, the prototype will perform flights for data collection purposes.  The testing procedure is designed to yield data that will be directly evaluated against the CFD model results.  First, begin with a hover flight condition and run multisine signal injection.  This invokes all the control inputs simultaneously, and utilizes magnitudes encompassing the control input deviations used within the CFD analysis.  This process will be repeated for four or five different system angular rates.  Next consider the two CFD operating states: airspeed and angle of attack.  Adjusting one parameter, while leaving the other fixed, directly mirrors the two types of collective control modes.  Collective throttle adjusts the lift force, which maps directly to satellite airspeed.  Similarly, collective elevator induces a helix spiral vertical translation, which corresponds with angle of attack.  Thus, obtaining flight data for the two remaining CFD parameters is a simple matter of running multisine injection while ascending/descending at different rates, while utilizing the two different collective control modes.
%
%\emph{Model Validation:}
%Identical to the \emph{Mark 1} prototype, post processing \emph{Mark 2} flight data with parameter identification algorithms will yield the state space representation of the system.  As before, the matrix elements represent the partial derivatives of the system states, and prescribe the mathematical relationships between control inputs and system outputs.  However, unlike the multibody validation, where this relationship was directly obtained from the linearization routine within the simulation, the CFD validation requires one intermediate step: identifying the forces and moments acting on each of the TURN components.  The CFD analysis produces forces and moments for a given set of system states and control inputs.  This induces translation and rotation in a known manner because it must adhere to the laws of physics.  Therefore, with known mass and moment inertia parameters, the time-varying force and moment values can be derived directly from the system states and control inputs via the recorded flight data.  The set of forces and moments that arise during flight testing will be compared to the CFD results, which will identify and assess any major discrepancies, fine tune the parameter settings, and validate the CFD modeling process.
%
%\emph{Anticipated Outcomes:}
%The outcome of this effort will produce the next largest scale TURN system, and yield the flight data needed to validate the CFD models.  Similar to the multibody simulation validation, a CFD analysis can always be adjusted to accurately represent a single operating point, but this would not provide confidence for the entire flight envelope.  Assessing the flight dynamics at several hover conditions, and at multiple operating points, provides a more rigorous investigation for the accuracy of the CFD analysis model.  Having confidence in this tool will help predict endurance and flight characteristics, and provides the foundation for performing design trade studies and engineering optimizations.
%
%
%%------------------------------
%{\bf Design Trade Studies and Engineering Optimizations:}
%Following the \emph{Mark 2} flight testing, eight (8) weeks are allocated for engineering design and optimization.
%
%\emph{Objective:}
%Having validated the primary set of tools needed to design a TURN aircraft, the next task utilizes those tools and investigates how to maximize the performance of subsequent TURN prototypes.  Some design parameters are in conflict with one another, so trade studies will showcase the relative cost/benefit of each embodiment.  Similarly, there are many potential design optimizations within the TURN system, which can now be more formally investigated via the CFD analysis.
%
%\emph{Design Trade Studies:}
%Following analyses will evaluate cost/benefit associated with each feature.
%\begin{itemize}
%\item Aspect Ratio:
%Lift induced drag is heavily influenced by the aspect ratio of the wing.  While, a high aspect ratio offers a definitive improvement in aerodynamic performance, slender wings also introduce flexibility within the structure.  Centrifugal stiffening within the TURN system offers a decided advantage, but a formal investigation is still needed to identify the diminishing returns on aerodynamic performance.
%\item Winglets:
%These improve aerodynamic performance by decreasing wingtip vortices, but they add weight to the system and have a more pronounced influence on low aspect ratio wings.  This analysis will assess whether any reduction in drag is worth the added weight penalty.
%\item Washout:
%This feature varies the incidence of the airfoil across the wingspan.  In general, it can decrease aerodynamic efficiency because the entirety of the wing no longer operates at an ideal angle-of-attack.  However, it offers two potential benefits.  First, it can be used to attain an elliptic load distribution, which increases the Oswald efficiency factor, and reduces induced drag from wingtip vortices.  Second, different portions of the wing stall at different times, which adds robustness into the flight envelope.  These benefits will be weighed against any degradation in aerodynamic efficiency.
%\end{itemize}
%
%\emph{Engineering Optimizations:}
%Following optimizations identify ideal values for each design feature.
%\begin{itemize}
%\item Outboard Horizontal Stabilizer (OHS):
%Research shows that a horizontal stabilizer properly placed within the wingtip vortex can reduce drag and contribute some amount of lift.  Appropriate sizing and placement is highly dependent on the system, so this task will find an ideal layout to capture the maximum benefit.
%\item Dihedral:
%Centrifugal stiffening will greatly mitigate bending moments within the wing, but some may still be present.  Introducing a curvature with some dihedral can offset any remaining bending, which will help keep the wing horizontal during operation.
%\item Asymmetric Span Profile:
%An asymmetric nonuniform profile distribution is a unique feature available to a TURN system.  From the system rotation, the outboard wingtip travels faster than the inboard section.  An asymmetric wingspan can exploit this feature, such that each segment is optimized for its intended airspeed.
%\end{itemize}
%
%\emph{Anticipated Outcomes:}
%These analyses will provide valuable insight for subsequent TURN prototypes, help increase aerodynamic performance, improve flight handling characteristics, and save time and money by seeking optimized TURN designs from an early stage of development.
%
%
%%------------------------------
%{\bf Airfoil Structural Assessment Prototype:}
%After validating aerodynamic flight characteristics, the next most pressing research activity considers the custom airfoil design and its structural attributes.  This endeavor builds and flies the custom \emph{Mark 3} prototype system.  The process spans a twenty-four (24) week period, but allocates three weeks for prudent flight testing delays associated with test site scheduling availability and inclement weather.
%
%\emph{Objectives:}
%While previous prototypes focused on validating multibody dynamic and CFD models, the primary purpose of the \emph{Mark 3} system is evaluate structural models while utilizing a custom designed airfoil that fully leverages centrifugal stiffening within the TURN concept design.  This vehicle will also incorporate all the previously developed components: custom avionics, revised AHRS algorithm, validated dynamic system response model, CFD aerodynamic analysis tool, deign trade studies, and system optimizations.
%
%\emph{Finite Element Analysis:}
%Building upon the \emph{Mark 2} CFD analysis, DAR will develop a Finite Element Analysis (FEA) model of the \emph{Mark 3} system, to assess static and dynamic structural loads.  The culmination of these two models will provide a full aeroelastic model analysis.  Since this is the largest scale vehicle with the greatest aspect ratio wing, understanding the structural and aeroelastic properties is of utmost importance.  DAR will review: flutter, which is a type of dynamic instability undergoing simple harmonic motion; divergence, where a deflected load increases the magnitude of the deflection; and control surface reversal, where inputs have the opposite expected response.  Finally, a refined propeller model utilizing blade element theory, will better assess prop wash interactions on the fuselage, stabilizers and control surfaces.
%
%\emph{Custom Airfoil Design:}
%Up to this point, each prototype utilized an existing airfoil profile with known characteristics.  This accelerated the development process, and reduced program risk by utilizing components with familiar capabilities.  While these serve their primary purpose, they do not approach the full potential that centrifugal stiffening offers from the TURN architecture.  DAR Corp will be tasked with developing a custom airfoil profile for the TURN system.  Their CFD and FEA models will provide a baseline for the dimensions and characteristics desired within the custom airfoil.  Given these inputs, they will develop a tailored pressure distribution to maximize lift and minimize drag, while exploring geometries (thin thickness, high camber) that are not typically appropriate for standard traditional fixed-wing aircraft.
%
%\emph{Prototype Manufacturing:}
%Saxon Remote Systems is responsible for the design and construction of the \emph{Mark 3} prototype.  While the \emph{Mark 2} central hub will be reused for this embodiment, each satellite rotor will be a custom fabrication job.  The wing construction will utilize a low density foam core cut to specifications with a CNC machine, an embedded carbon fiber stringer will add stiffness, and a carbon fiber outer skin encompasses the wing structure using a vacuum bag sandwich construction technique.
%
%\emph{Structural Testing:}
%In addition to the components needed for flight testing, an extra satellite wing will be manufactured for the sole purpose of structural testing.  The satellite fuselage will be secured in place, and incremental weight will be applied on the opposite end.  Wingtip deflections will be recorded, which will assess the bending moment throughout the wingspan.  Similarly, a torsion test will apply incremental moments on the free end, and angular deflections will characterize the twist present within the structure.
%
%\emph{Flight Testing Procedure:}
%As before, flight testing will proceed incrementally with a taxi test, to monitor equal spacing, collective commands, and cyclic commands.  The same flight testing procedure utilized on the \emph{Mark 2} prototype will be repeated for the \emph{Mark 3} vehicle.  Thus, the same parameter identification and model validation processes will be used for this larger system.  An additional round of flight testing will assess flexibility within the wingspan.  Sinusoidal signals will be superimposed on the collective input commands to induce a target oscillation within the TURN prototype, such that the dynamic response can be compared to the developed FEA models.
%
%\emph{Anticipated Outcomes:}
%The \emph{Mark 3} system will attempt to exceed a 14-day flight endurance, while utilizing solar cells to regenerate power.  After initial testing, assessment, evaluation, and minor adjustments or modifications, the concluding flights will showcase this capability.  Saxon will provide manpower support to help continuously monitor the TURN aircraft during this operation.  Flight testing this vehicle will help further refine the available design tools, assess the performance and endurance capabilities of the system, and illuminate required revisions within subsequent designs.
%
%
%
%
%%------------------------------
%\subsection{Statement of Work (SOW)}
%%------------------------------
%Since the Statement of Work does not count towards the page limit, a stand-alone section has been Appended following \emph{Volume I: Technical and Management Proposal}.  It closely mirrors the Work Plan content, but includes: key tasks, critical path elements, and milestone completion dates.
%
%
%
%
%%------------------------------
%\subsection{Technical Rationale}
%%------------------------------
%
%The goal of the TURN concept vehicle is to attain persistently enduring flight, utilizing solar cells with an electric propulsion system.  Phase I SBIR research for the Air Force produced models for flight endurance, and investigated the feasibility of attaining the stated goal.  Two designs were developed; the first meets the specified mission with a 250 pound payload and 2000 watt power requirements, while the second attains a wider range of coverage a with smaller 100 pound payload and 800 watt power draw.
%
%
%%------------------------------
%{\bf Airfoil Selection:}
%An analysis investigated the influence of two primary airfoil parameters: thickness and camber.  For each group, five sample profiles were compiled from an airfoil database.  Camber has a diminished influence on thick airfoils, and the $L/D_\text{max}$ ratio falls within a 70-110 range.  Camber has a much more pronounced effect on thin airfoils, where the $L/D_\text{max}$ averaged over 200, with a maximum of 237.  Furthermore, they demonstrate greater separation between cruise and stall, which helps improve wind robustness at slow airspeeds.  Thick airfoils are commonly used to reinforce flimsy wing structures, but with with centrifugal stiffening within the TURN concept, slender airfoils with high camber are feasible, and can achieve extraordinarily high $L/D$ ratios, as seen in Figure \ref{Fig:Eppler}.  Finally, the proposed research budgets for a custom designed airfoil, which may attain values beyond what is readily available in the existing airfoil database.
%
%\begin{figure}
%\begin{floatrow}
%\includegraphics
%[ width=\Wthird, height=1.7in, clip, trim={0 0 0 0} ]
%{../Figures/Eppler_LDAlpha.jpg}
%\hspace{0.1cm}
%\includegraphics
%[ width=\Wthird, height=1.7in, clip, trim={0 0 0 0} ]
%{../Figures/Eppler_ClAlpha.jpg}
%\hspace{0.1cm}
%\includegraphics
%[ width=\Wthird, height=1.7in, clip, trim={0 0 0 0} ]
%{../Figures/Eppler_CdAlpha.jpg}
%\end{floatrow}
%\caption{Eppler 58 Airfoil Used for TURN Sizing Analysis}
%\label{Fig:Eppler}
%\end{figure}
%
%
%%------------------------------
%{\bf Operational Environment:}
%A unique attribute of the stratosphere is that temperature rises with increased altitude, so the air is dynamically stable offering mild wind conditions and little turbulence.  Lower air density also reduces parasitic drag and power consumption, and larger wings are advantageous for a solar collection.  An atmospheric-satellite at 65k feet has line of sight for 4.5\degree{} of latitude, which is equivalent to a coverage area with a 311 mile radius.
%
%
%%------------------------------
%{\bf Analysis Setup:}
%An optimization program allows key parameters to vary across a specified range, and the objective function minimizes daily energy consumed per area of solar cell.  The program includes: airfoil, wing, coefficients, fuselage, prop, battery, and aerodynamic surfaces.  It establishes the force/moment balance relationships within the multibody element system, and obtains the total power consumed by the vehicle.  Finally, the daily energy requirement is normalized by diving by the amount of solar area.
%
%
%%------------------------------
%{\bf Solar Collection Map:}
%After developing the analysis for the persistently enduring TURN model, a metric to evaluate its ability to remain airborne indefinitely was devised.
%
%\emph{Solar Cells:}
%Solar cell efficiency is a percentage of the total amount of solar power captured.  Commercial solar cells attain 20-24\% ratings, new research shows gallium arsenide with 30-34\% efficiency, Boeing is developing a product with targeting 39.2\%, and the world record was recorded at 46\%.  Rather than design the TURN system based on unproven solar cell technology, the analysis utilized a 24\% efficiency factor within the calculations.
%
%\emph{Angle of Incidence:}
%Solar cell orientation is rarely orthogonal to the direction of sunlight.  Angle of incidence measures this deviation, where larger angles reduce the amount of energy capture.  Angular elevation measures the angle of the sun with respect to the equator, and is determined by latitude and declination angle.  The analysis calculated the energy captured by a solar cell array while accounting for both latitude and day-of-year.
%
%\emph{Daily Collected Energy:}
%The next task found daily energy collected at various latitudes and day-of-year.  Maximum solar elevation, and maximum energy collection, were determined at each state.  Power per unit area for the day was modeled as the top half of a sine wave, where the function was bounded by the duration of daylight, and the integral of the curve yielded the daily energy collected.  Table \ref{Tbl:SolarMap} summarizes the results and provides daily energy collected per area of solar cell.
%
%\newcommand{\cp}[1]{$\;\;$#1$\;\;$}
%\begin{table}
%\small
%\renewcommand{\arraystretch}{0.93}
%\begin{tabular}{[c!cccccccccc]}
%\thickhline
%\bf Lat/Day&
%\bf \cp{00}  & 
%\bf \cp{40}  & 
%\bf \cp{80}  & 
%\bf \cp{120} & 
%\bf \cp{160} &
%\bf \cp{200} & 
%\bf \cp{240} & 
%\bf \cp{280} & 
%\bf \cp{320} & 
%\bf \cp{360} \\
%\thickhline
%\bf    90 &
%\p   0.00 & \p  0.00 & \p  0.00 & \y  4.50 & \g  6.99 & 
%\g   6.40 & \y  2.91 & \p  0.00 & \p  0.00 & \p  0.00 \\
%\bf    85 &
%\p   0.00 & \p  0.00 & \p  0.77 & \g  6.00 & \g  8.41 &
%\g   7.84 & \y  4.44 & \p  0.00 & \p  0.00 & \p  0.00 \\
%\bf    80 &
%\p   0.00 & \p  0.00 & \y  1.55 & \g  7.45 & \g  9.76 &
%\g   9.22 & \g  5.94 & \p  0.40 & \p  0.00 & \p  0.00 \\
%\bf    75 &
%\p   0.00 & \p  0.00 & \y  2.32 & \g  8.85 & \g 11.04 &
%\g  10.53 & \g  5.58 & \y  1.09 & \p  0.00 & \p  0.00 \\
%\bf    70 &
%\p   0.00 & \p  0.38 & \y  3.07 & \g  7.87 & \g 12.23 &
%\g  11.76 & \g  5.96 & \y  1.82 & \p  0.03 & \p  0.00 \\
%\bf    65 &
%\p   0.09 & \p  0.97 & \y  3.80 & \g  8.02 & \g 11.88 &
%\g  10.70 & \g  6.44 & \y  2.56 & \p  0.46 & \p  0.08 \\
%\bf    60 &
%\p   0.54 & \y  1.63 & \y  4.50 & \g  8.30 & \g 11.12 &
%\g  10.39 & \g  6.93 & \y  3.29 & \y  1.05 & \p  0.51 \\
%\bf    55 &
%\y   1.12 & \y  2.33 & \g  5.16 & \g  8.60 & \g 10.91 &
%\g  10.34 & \g  7.40 & \y  4.01 & \y  1.70 & \y  1.08 \\
%\bf    50 &
%\y   1.77 & \y  3.04 & \g  5.79 & \g  8.89 & \g 10.83 &
%\g  10.36 & \g  7.83 & \y  4.69 & \y  2.39 & \y  1.73 \\
%\bf    45 &
%\y   2.46 & \y  3.75 & \g  6.37 & \g  9.14 & \g 10.76 &
%\g  10.38 & \g  8.22 & \g  5.35 & \y  3.10 & \y  2.42 \\
%\bf    40 &
%\y   3.17 & \y  4.45 & \g  6.91 & \g  9.34 & \g 10.69 &
%\g  10.38 & \g  8.55 & \g  5.96 & \y  3.81 & \y  3.13 \\
%\bf    35 &
%\y   3.89 & \g  5.12 & \g  7.39 & \g  9.49 & \g 10.57 &
%\g  10.33 & \g  8.83 & \g  6.53 & \y  4.51 & \y  3.85 \\
%\bf    30 &
%\y   4.60 & \g  5.77 & \g  7.82 & \g  9.58 & \g 10.42 &
%\g  10.24 & \g  9.04 & \g  7.06 & \g  5.20 & \y  4.56 \\
%\bf    25 &
%\g   5.30 & \g  6.38 & \g  8.19 & \g  9.61 & \g 10.21 &
%\g  10.09 & \g  9.19 & \g  7.53 & \g  5.85 & \g  5.26 \\
%\bf    20 &
%\g   5.98 & \g  6.95 & \g  8.49 & \g  9.57 & \g  9.95 &
%\g   9.88 & \g  9.28 & \g  7.95 & \g  6.48 & \g  5.94 \\
%\bf    15 &
%\g   6.63 & \g  7.48 & \g  8.74 & \g  9.47 & \g  9.63 &
%\g   9.61 & \g  9.30 & \g  8.31 & \g  7.07 & \g  6.60 \\
%\bf    10 &
%\g   7.24 & \g  7.96 & \g  8.91 & \g  9.30 & \g  9.26 &
%\g   9.29 & \g  9.25 & \g  8.61 & \g  7.62 & \g  7.22 \\
%\bf     5 &
%\g   7.82 & \g  8.38 & \g  9.02 & \g  9.07 & \g  8.84 &
%\g   8.91 & \g  9.13 & \g  8.84 & \g  8.11 & \g  7.79 \\
%\bf     0 &
%\g   8.35 & \g  8.75 & \g  9.07 & \g  8.78 & \g  8.36 &
%\g   8.48 & \g  8.95 & \g  9.01 & \g  8.56 & \g  8.33 \\
%\thickhline
%\end{tabular}
%{\caption{Solar Collection Map: Daily Energy per Solar Array Area (MJ m${}^{\text{-2}}$ day${}^{\text{-1}}$)}
%\label{Tbl:SolarMap}}
%\end{table}
%
%
%%------------------------------
%{\bf Results:}
%Table \ref{Tbl:SolarMap} presents the amount of energy collected per day at various latitudes throughout the year.  Red indicates there is not enough available energy, green indicates persistently enduring flight is possible with 250 pound payload drawing 2000 watts of power, and yellow regions are attainable if some requirements were relaxed.  This conveys that persistently enduring flight is feasible with 77\% coverage of the solar map, where a TURN system can service the equator and polar regions during summer months.  However, the real challenge targets higher latitudes during winter months, with limited daylight and low azimuth angles.  Even on the worst winter day, the TURN aircraft can achieve eternal flight at 30\degree{} latitude, while a small compromise on payload or power, could service the entire desired geographic region.
%
%
%
%
%%------------------------------
%\subsection{Risk and Risk Reduction}
%%------------------------------
%
%The spiral development process is designed to address the greatest project risk in incremental stages throughout the research program.  Some of those elements have already been abated through previous efforts, while the next most pressing endeavors are tackled within this proposed research.
%
%As a summary, the following items outline the major risk elements and how they are mitigated:
%\begin{itemize}
%\item \emph{PhD Research}: Previously developed a multibody dynamic model to capture the nonlinear system dynamics; derived several MRAC architectures to implement various control loops; and successfully demonstrated reference tracking and disturbance rejection performance.
%\item \emph{System Models}: Phase I SBIR research for the Air Force developed the preliminary aerodynamic and structural models needed to size the TURN system and predict its flight endurance capability; and showed that thin airfoils with high camber can attain L/D ratios three times greater than profiles utilized on traditional tube-and-wing aircraft.
%\item \emph{Avionics Development}: Assembled custom hardware, software, and algorithms needed for prototype development; implemented and tested a custom AHRS algorithm on the \emph{Mark 0} system to accommodate the system rotation; and demonstrated multisine signal injection routines to facilitate parameter identification within subsequent prototypes.
%\item \emph{Multibody Validation}: The modular \emph{Mark 1} prototype has already been built and performed early flight tests; proposed work will perform parameter identification across a wide range of physical parameters and flight conditions; resulting in a validated simulation model describing the nonlinear system dynamics among the multiple rigid body elements.
%\item \emph{CFD Model Refinement}: Preliminary aerodynamic models have been developed to assess flight endurance capabilities; the \emph{Mark 2} system will be modeled within a more formal CFD investigation; collected flight data will correlate the analytic models; resulting in a design tool to predict aerodynamic performance of a TURN system.
%\item \emph{Optimizations and Trade Studies}: Optimizations will identify ideal physical parameters to minimize power consumption; engineering trade studies will evaluate the cost/benefits of competing design considerations; outcomes will be incorporated into the subsequent prototype; resulting in an analysis techniques which expedites future TURN designs.
%\item \emph{Airfoil Structural Assessment}: Preliminary structural models assessed vehicle sizing; the \emph{Mark 3} system will be modeled with a more formal FEA investigation; collected flight data will correlate bend and twist within the custom airfoil profile; resulting in a design tool to fully leverage centrifugal stiffening while utilizing more aggressive airfoil profiles.
%\end{itemize}
%
%
%
%
%%------------------------------
%\subsection{Results}
%%------------------------------
%
%Prototype hardware will be produced during the proposed research program.  While Gradient retains ownership of the physical prototype systems, DARPA will be provided with documentation and demonstration of all the research results.  Specific deliverables are as follows:
%\begin{enumerate}
%\item Documentation and demonstration of the \emph{Mark 1} prototype, with validated dynamic model.
%\item Documentation and demonstration of the \emph{Mark 2} prototype, with validated CFD model.
%\item Documentation and demonstration of the \emph{Mark 3} prototype, with validated FEA model.
%\item Trade study analysis.
%\item Optimization analysis.
%\item Custom airfoil profile analysis.
%\item Monthly status reports.
%\item Draft and final technical reports.
%\end{enumerate}
%
%Patent protection for the general TURN architecture has already been submitted with the USPTO.  Furthermore, it is anticipated that the novel attitude heading and reference system algorithm, developed during the Phase I Air Force research, will likely be a candidate for additional intellectual property protection.  Gradient will retain ownership of any patentable materials developed during the proposed DARPA research effort, but will extend rights which mirror SBIR Data Rights.
%
%
%
%
%%------------------------------
%\subsection{Experience}
%%------------------------------
%
%Several previous research projects conducted by Dr Justin M Selfridge, both directly and indirectly contribute towards the current TURN development effort.
%
%
%%------------------------------
%{\bf Air Force SBIR Phase I, Tethered Uni-Rotor Network:}
%Investigated an internal combustion engine TURN embodiment.  The specified mission calls for 2000 watts of power supplied to a 250 pound payload, while operating at 15,000 feet within 50 knot winds aloft.  Developed aerodynamic and structural models of the TURN concept, and adapted an optimization routine to account for changing force/moment balance relationships resulting from fuel burn.  Developed custom avionics hardware which was implemented on the first two spiral development prototypes.  A novel attitude and heading reference system (AHRS) algorithm was developed and tested to accommodate the rotating nature of the TURN system.  Results from the analysis and initial flight testing indicate that a 30-day flight endurance is attainable at a scale five times lighter than the MQ-9 Reaper.
%
%
%%------------------------------
%{\bf PhD Research, Tethered Uni-Rotor Network:}
%The first of two publications modeled the TURN system, which described the nonlinear dynamics, numerical multibody simulation structure, and unique requirements for trimming and linearizing a nonlinear multibody problem.  A second publication presented the initial controls analysis, which outlined the controller methodology, and introduced the unique requirements for the inner-loop and outer-loop control systems, which handle stabilization and navigation, respectively.  Controls development began with a single-input single-output solution and expanded to a more involved multivariable controller; which mitigates cross-coupling between inputs and outputs.  Although the TURN system layout and geometry has evolved since these publications, they still represents a substantial portion of the modeling and controller content seen in the current TURN embodiment.
%
%
%%------------------------------
%{\bf PhD Research, Adaptive Control:}
%Adaptive control techniques initially developed for quadrotors were subsequently implemented on the TURN system.  Multivariable model reference adaptive control (MRAC) architecture for a quadrotor, subsequently published in a journal, employed a state-feedback output-tracking design, with a less restrictive matching condition than a comparable state-tracking design.  A similar topic reviewed an output-feedback output-tracking controller design, which was applied to a quadrotor system.  The primary benefit of output-feedback rather than state-feedback, is that not all of the system states need to be directly available; rather the controller works exclusively with a smaller number of measured states.  Both adaptive control laws were based on a parameterization of an LDS matrix decomposition of the high-frequency gain matrix, which reduces the amount of \emph{a priori} knowledge needed for the plant.
%
%
%%------------------------------
%{\bf NASA Research, Greased Lightning (GL-10) Autonomous Flight:}
%While working for NASA Langley, this novel tiltwing UAV has been featured on CNN and Popular Science.  It uses ten electric brushless motors, and transitions from a VTOL configuration into a fixed-wing forward flight system.  Implemented a remote, autonomous takeoff and landing capability for the unmanned aircraft, and developed theory for a robust adaptive controller which interacts with the localization solution to enhance the takeoff and landing tasks.  Developed autonomous waypoint navigation and path following routines, implemented with custom developed avionics hardware.
%
%
%
%
%%------------------------------
%\subsection{Facilities}
%%------------------------------
%
%Two more prototypes will be developed during the proposed research.  First, an existing airfoil wing will be modified to represent a TURN configuration, which will be used for CFD model validation.  Second, a larger prototype will utilize a custom designed airfoil, and develop and test the manufacturing processes required to build these thin profile structures.  Facilities and equipment from several groups will be utilized to carry out this research program.
%
%
%%------------------------------
%{\bf Hardware for Avionics Development:}
%All the tools and equipment needed to build custom avionic hardware for concept aircraft are available to Gradient Consulting.  This includes a prototyping workshop that has previously constructed numerous unconventional multirotor and fixed-wing platforms, and developed multiple custom fabricated avionics boards.  Equipment commonly utilized includes: soldering station, power supply, oscilloscopes, prototype boards, motors, props, servos, electronic components, and model aircraft building supplies.  More specialized equipment includes: a 3D printer, RF transceivers, dedicated GCS laptop computers, and RC radios.  Additionally, the PI is registered with the FAA and holds a valid Part 107 Remote Pilot Certification.
%
%
%%------------------------------
%{\bf Software for Analysis and Design:}
%DAR Corp routinely employs Star-CCM+ and FlightStream software for CFD, FEA, and aeroelastic modeling an analysis.  Gradient relies heavily on the Matlab software package, for vehicle modeling and simulation, controls development, post processing flight data, and running parameter identification.  Finally, Gradient will purchase SolidWorks CAD modeling software to coordinate design activities with Saxon Remote Systems.
%
%
%%------------------------------
%{\bf Prototype Manufacturing:}
%Saxon is already fully equipped for the prototype manufacturing task.  Their facility includes all the tooling and equipment for custom composite work, and is readily used to build their commercial products in-house.  This encompasses: CNC machines, hot-wire foam cutter, vacuum bagging equipment, composite mold preparation, 3D printers, and a wide assortment of tools commonly employed within manufacturing workshops.
%
%
%%------------------------------
%{\bf Flight Testing Facilities:}
%Established contacts within NASA Langley and its Autonomy Incubator, permit access to the ``back forty,'' which is forty acres of sanctioned FAA airspace, dedicated to research and development for experimental UAV prototypes.  For larger facilities, the PI has relationships with Fort AP Hill, from flight testing the NASA \emph{GL-10} prototype.  Located less than two hours away, it includes a dedicated FAA sanctioned UAV flight testing range, with a larger operational area, and a dedicated hangar to accommodate multiple day testing.  Finally, Saxon Remote Systems routinely utilizes Ryan Airfield, located outside Tuscon, AZ, to conduct flight testing and customer training for their commercial UAS system offerings.
%
%
%
%
%%------------------------------
%\subsection{Organization}
%%------------------------------
%
%This proposed DARPA research project is a highly multidisciplinary effort, which specifically leverages the engineering PhD research experience of three extremely qualified individuals, as well as the organizations they work within.
%
%
%%------------------------------
%{\bf Justin M Selfridge, PhD}
%
%\vspace{-0.2cm}
%United States Citizen \\
%PhD in Electrical and Computer 
%Engineering - University of Virginia (UVA) \\
%Gradient Consulting, LLC - Owner, Managing Member \\
%NASA Langley - Independent Contractor
%
%\emph{Biography} \\
%Dr Justin M Selfridge completed his PhD in Electrical Engineering at the University of Virginia.  Research areas include: multivariable adaptive control, system identification, autonomous vehicles, custom avionics hardware development, and UAV design and prototype fabrication.  The past six years, Dr Selfridge has worked within the Dynamic Systems and Controls Branch at NASA Langley.  His most recent NASA project developed the controller for the \emph{GL-10} concept vehicle; a ten rotor, tilt wing/tail, transformational flight vehicle, with VTOL capabilities and an extended flight endurance.  His PhD developed the nonlinear dynamic model, adaptive controller methodology, and numerical multibody simulation tools, for the TURN concept system; a novel aircraft seeking persistently enduring flight.  Most recently he has been pursuing the TURN vehicle development full-time, working under his consulting company, Gradient Consulting, LLC.
%
%\emph{Education} \\
%\begin{tabular}{llll}
%\vspace{-0.15in}
%\hspace{2.2in} &
%\hspace{1.3in} &
%\hspace{1.3in} &
%\hspace{0.0in} \\
%University of Virginia &
%Charlottesville, VA &
%Electrical Engr &
%Ph.D., 2017 \\
%Old Dominion University &
%Norfolk, VA &
%Mechanical Engr &
%MS, 2012 \\
%University of Virginia &
%Charlottesville, VA &
%Mechanical Engr &
%BS, 2004
%\end{tabular}
%
%\emph{Research Experience} \\
%\begin{tabular}{llll}
%\vspace{-0.15in}
%\hspace{2.2in} &
%\hspace{1.3in} &
%\hspace{1.3in} &
%\hspace{0.0in} \\
%Air Force Phase I SBIR &
%Newport News, VA &
%Principle Investigator &
%2017-present \\
%NASA Langley Research Center &
%Hampton, VA &
%Contractor &
%2013-2017 \\
%NASA Langley Research Center &
%Hampton, VA &
%Work Study &
%2011-2013 \\
%University of Virginia &
%Charlottesville, VA &
%Teach/Research Asst &
%2012-2014 \\
%Old Dominion University &
%Norfolk, VA &
%Teach/Research Asst &
%2010-2012
%\end{tabular}
%
%\emph{Authored Publications} \\
%\vspace{-0.2in}
%
%\PubSpace
%Selfridge, Justin M.
%\emph{Achieving Eternal Flight With a Tethered Uni-Rotor Network (TURN) Aircraft: A Complete Development of the Nonlinear Dynamic Model and Controller Architecture}.
%Doctorate of Philosophy dissertation.
%University of Virginia; Charlottesville, VA;
%May 2017.
%
%\PubSpace
%Selfridge, Justin M., and Gang Tao.
%``Multivariable Output Feedback MRAC for a Quadrotor UAV.''
%\emph{IEEE American Control Conference (ACC).}
%Boston, MA.
%Jul 2016.
%
%\PubSpace
%Selfridge, Justin M., and Gang Tao.
%``Centrifugally Stiffened Rotor: A Complete Derivation of the Plant Model with Nonlinear Dynamics.''
%\emph{AIAA Aviation Technology, Integration, and Operations Conference.}
%Dallas, TX.
%Jun 2015.
%
%\PubSpace
%Selfridge, Justin M., and Gang Tao.
%``Centrifugally Stiffened Rotor: A Complete Derivation and Simulation of the Inner Loop Controller.''
%\emph{AIAA Guidance Navigation and Control (GNC) Conference.}
%Kissimmee, FL.
%Jan 2015.
%
%\PubSpace
%Selfridge, Justin M., and Gang Tao.
%``A Multivariable Adaptive Controller for A Quadrotor with Guaranteed Matching Conditions.''
%\emph{IEEE American Control Conf (ACC).}
%Portland, OR.
%Jun 2014.
%
%\PubSpace
%Selfridge, Justin M.
%``A Multivariable Adaptive Controller for a Quadrotor with Guaranteed Matching Conditions.''
%\emph{System Science and Control Engineering.}
%Vol 2, No 1, pp 24-33; 2014.
%
%
%%------------------------------
%{\bf Willem AJ Anemaat, PhD}
%
%\vspace{-0.2cm}
%United States Citizen \\
%PhD in Aerospace Engineering - University of Kansas (KU) \\
%DAR Corporation - Founder, Owner
%
%\emph{Biography} \\
%Dr Willem AJ Anemaat is frequently a session chair for SAE, AIAA and ICAS conferences, and is a current member of the SAE WATC (Wichita Aviation Technology Conference and Exposition) Committee, where he served as the program technical chair in 2008.  Dr Anemaat has been a regular reviewer of airplane design related articles for the AIAA Journal of Aircraft and is a judge for AIAA Design competitions.  He was member of the AIAA Journal of Aircraft Editorial Advisory Board from 2004 until 2009, and then graduated to became an Associate Editor for the AIAA Journal of Aircraft dealing with Aircraft Design topics.  He is the Chair of the AIAA Aircraft Design Technical Committee, and is an AIAA Associate Fellow.  Dr Anemaat became member of the Kansas UAV Consortium Executive Board in 2005, is the recipient of the SAE 2010 Forest R. McFarland Award, was inducted into the University of Kansas Aerospace Engineering Honor Roll in 2017, and is a member of the KUAE Advisory Board, having served as Chair from 2016-2017.
%
%\emph{Authored Publications} \\
%\vspace{-0.2in}
%
%\PubSpace
%Anemaat, Willem, et. al.
%``Comparison of Aerodynamic Analysis Tools for Rotorcraft in Hover.''
%\emph{AIAA Aerospace Sciences Meeting.}
%Kissimmee, FL.
%Jan, 2018.
%
%\PubSpace
%Anemaat, Willem, M. Schuurman, and W. Liu.
%``Aerodynamic Design, Analysis and Testing of Propellers for Small UAVs.''
%\emph{AIAA Aerospace Sciences Meeting.}
%Grapevine, TX.
%Jan, 2017.
%
%\PubSpace
%Anemaat, Willem.
%``Smart Aircraft Modeler for Aircraft Conceptual Design.''
%\emph{Collaboration in Aircraft Design (SCAD), Council of European Aerospace Societies (CEAS), Technical Committee Aircraft Design (TCAD), Research Section.}
%Naples, Italy.
%Oct, 2015.
%
%\PubSpace
%Anemaat, Willem, B. Kaushik, J. Carroll, and J. Jeffery.
%``Software Tool Development to Improve the Airplane Preliminary Design Process.''
%\emph{ISPE International Conference on Concurrent Engineering.}
%Melbourne, Australia.
%Sept, 2013.
%
%\PubSpace
%Anemaat, Willem, W. Liu, B. Kaushik, M. Yang, M. Brown, and R. Hale.
%``Design, Build and Fly: NASA Lockheed P-3 Orion with External Antenna Fairings.''
%\emph{AIAA Aerospace Sciences Meeting.}
%Grapevine, TX.
%Jan 2013.
%
%
%%------------------------------
%{\bf James E Hubbard, Jr, PhD}
%
%\vspace{-0.2cm}
%United States Citizen \\
%PhD in Mechanical Engineering - Massachusetts Institute of Technology (MIT)  \\
%Langley Distinguished Professor - University of Maryland (UM)  \\
%Samuel P Langley Professor - National Institute of Aerospace (NIA) \\
%StarLab VICON Motion Capture Studio - Founder
%
%\emph{Biography} \\
%Dr James E Hubbard's career has spanned some 20 years and includes work in aero-acoustics for noise-control, adaptive structures, spatially distributed transducers and the extension of modern time domain control methodologies into the spatial domain for the real-time control of distributed systems.  His work has resulted in a dozen patents which have demonstrated the efficacy and practicality of the techniques that he and his students have developed over these years.  These techniques have been viewed as innovative and revolutionary by his colleagues, and to recognize of these accomplishments, the University of Maryland named Dr Hubbard the University of Maryland Langley Professor at the National Institute of Aerospace.
%
%\emph{Honors and Awards} \\
%\vspace{-0.2in}
%
%\PubSpace
%2016 - Inducted into the National Academy of Engineering for advances in the modeling, design, analyses, and application of adaptive structures.
%
%\PubSpace
%2016 - Inducted into the Virginia Academy of Engineering, Science and Medicine.
%
%\PubSpace
%2016 - Society of Photonics and Instrumentation Engineers (SPIE) Smart Structures and Materials (SSM) Lifetime Achievement Award for those viewed as luminaries in the fields of SSM.
%
%\PubSpace
%2015 - Best Paper in Structures Award from the Adaptive Structures and Material Systems (ASMS) branch of the Aerospace Division of the American Society of Mechanical Engineers (ASME).
%
%\PubSpace
%2015 - Elected American Society of Mechanical Engineers (ASME) Fellow having made noteworthy invention, discovery or advancement in the state of the art of mechanical engineering research.
%
%\PubSpace
%2012 - Elected American Institute of Aeronautics and Astronautics (AIAA) Fellow, where this distinction is conferred upon those members of the Institute who have made notable and valuable contributions to the arts, sciences, or technology of aeronautics and astronautics.
%
%
%
%
%%------------------------------
%\newpage
%\subsection{Project Management}
%%------------------------------
%\vspace{-0.1cm}
%\begin{figure}[h!]
%\includegraphics
%[ width=\Hwhole, height=\Wwhole, angle=90, origin=c, clip, trim={0 0 0 0} ]
%{../Figures/ProjMgmt.png}
%\end{figure}
%
%
%
%
%
%
%
%
%%------------------------------------------------------------
%\newpage
%\section{STATEMENT OF WORK (SOW)}
%%------------------------------------------------------------
%
%
%
%
%%------------------------------
%{\bf 1.0 \hspace{0.2cm} Research Overview}
%
%The overall objective of this research is to progress the development of a high-altitude unmanned aerial system (UAS), capable of persistently enduring flight.  Such an aircraft collects enough solar energy during the day to remain aloft throughout the night, thereby eliminating the need to land for refueling.  This research is developing a new concept aerial architecture called the Tethered Uni-Rotor Network (TURN), which combines the best features of both fixed-wing and rotorcraft designs, and provides aerodynamic capabilities beyond what existing aerial systems can attain.  The target altitude is between 60k-70k feet, where air density and winds aloft offer ideal flight conditions.  It seeks to operate at a wide range of latitudes for any given day-of-year, while supporting payloads greater than 200 pounds.  The research project will: collect data on existing prototypes, build and fly two additional prototypes, refine a CFD aerodynamic analysis, perform design trade studies and engineering optimizations, develop and assess an FEA structural model, and further progress the project toward accomplishing the aviation milestone of persistently enduring flight.
%
%
%
%
%%------------------------------
%{\bf 2.0 \hspace{0.2cm} Spiral Development Summary}
%
%Because the TURN concept vehicle is a radical departure from conventional aircraft design, the research employs a spiral development methodology.  Scaled prototypes address the largest project risk, then simulation models are validated with collected flight data, which yields new analytic tools for subsequent prototype design.  Following this methodology, an Air Force Phase I SBIR successfully delivered the \emph{Mark 0} and \emph{Mark 1} prototypes.  These systems developed and tested the custom avionics hardware, software, and algorithms, and established a modular flight testing framework for multibody dynamic model validation, respectively.  Proposed DARPA research continues this process with two more prototypes, \emph{Mark 2} and \emph{Mark 3}.  These aircraft will refine aerodynamic models to perform trade studies and optimizations, and assess the structural attributes of the slender airfoil which dictates TURN vehicle sizing.  While potential SBIR Phase II funding will investigate a medium-altitude combustion engine TURN embodiment, the exclusive focus of the proposed DARPA project pursues the persistently enduring flight capability.
%
%
%
%
%%------------------------------
%{\bf 3.0 \hspace{0.2cm} Technical Objectives}
%
%The specific technical objectives are compartmentalized into four distinct tasks:
%\begin{itemize}
%\item Collect flight data on the existing \emph{Mark 1} system within a VICON motion capture studio and validate the multibody dynamic models developed from previous PhD research.
%\item Design, model, build, and flight test the \emph{Mark 2} prototype, where collected flight data will refine an aerodynamic model and characterize flight performance.
%\item Utilizing the revised CFD model as an analytic tool, perform engineering optimizations and design trade studies to aid the design of the \emph{Mark 3} system.
%\item Design, model, build, and flight test the \emph{Mark 3} prototype, and assess structural considerations within the custom designed thin and slender airfoil structures.
%\end{itemize}
%
%
%
%
%%------------------------------
%{\bf 4.0 \hspace{0.2cm} Listing of Itemized Tasks}
%\newcounter{Task}
%\newcounter{SubTask}[Task]
%
%Four distinct tasks are associated with the research effort.  First, use VICON motion capture studio to collect flight data on existing modular prototype, to aid with multibody dynamic model validation.  Second, develop CFD models of the aircraft, build the next prototype with an existing airfoil from a UAV manufacturer, and collect flight data for CFD validation.  Third, expand CFD models for the next prototype, perform optimizations for parameter settings and investigate cost/benefit trade studies.  Fourth, use analysis results to design and build the next prototype with a custom airfoil profile, and assess structural considerations through an FEA model.
%
%
%
%
%%------------------------------
%\stepcounter{Task}
%{\bf Task \theTask:}
%\emph{Multibody Dynamics Validation (5 weeks)}
%
%
%\stepcounter{SubTask}
%\hypertarget{Sim}
%{Task \theTask.\theSubTask}:
%Setup and Run Matlab Simulations
%\TaskSpace
%\begin{itemize}
%\item Objectives: Simulate modular \emph{Mark 1} prototype within the multibody dynamic framework, while capturing all adjustable parameters utilized within the flight testing.
%\item Technical Approach:
%\begin{itemize}
%\item Run each simulation at its prescribed setting.
%\item Trim and linearize each of the models.
%\item Repeat for all adjustable parameters (prop, tail boom length, elevator control surface size, angle of attack, and aspect ratio).
%\end{itemize}
%\item Timeline: Initiates Task 1; spans two (2) weeks; leads into \task{Flight1}.
%\item Exit Criteria: All simulations have been run and post processed.
%\item Deliverables: State space models describing a multidimensional array of design parameters.
%\end{itemize}
%
%
%\stepcounter{SubTask}
%\hypertarget{Flight1}
%{Task \theTask.\theSubTask}:
%Mark 1 VICON Flight Testing
%\TaskSpace
%\begin{itemize}
%\item Objectives: Collect high fidelity localization information to be used for parameter identification and multibody dynamic model validation.
%\item Technical Approach:
%\begin{itemize}
%\item Perform camera calibration and assign reflective markers to prototype system.
%\item Flight test \emph{Mark 1} system across all adjustable parameters.
%\item Collect external `ground truth' flight data (position and attitude) at 500 Hz sample rate.
%\end{itemize}
%\item Timeline: Begins during \task{Sim}; spans two (2) weeks; leads into \task{Param1}.
%\item Exit Criteria: All prescribed flight conditions have been recorded.
%\item Deliverables: Sets of flight data for parameter identification.
%\end{itemize}
%
%
%\stepcounter{SubTask}
%\hypertarget{Param1}
%{Task \theTask.\theSubTask}:
%Parameter Identification
%\TaskSpace
%\begin{itemize}
%\item Objectives: Process collected flight so it can facilitate multibody dynamic model validation.
%\item Technical Approach:
%\begin{itemize}
%\item Run parameter identification on all collected data sets.
%\item Identify steady state operating points from flight data.
%\item Identify transient response characteristics from flight data.
%\end{itemize}
%\item Timeline: Begins during \task{Flight1}; spans two (2) weeks; leads into \task{MBD_Val}.
%\item Exit Criteria: All sets of \emph{Mark 1} flight data have been post processed.
%\item Deliverables: Collection of state space models from flight data to compare with simulations.
%\end{itemize}
%
%
%\stepcounter{SubTask}
%\hypertarget{MBD_Val}
%{Task \theTask.\theSubTask}:
%Multibody Validation Process
%\TaskSpace
%\begin{itemize}
%\item Objectives: Identify the parameter settings which minimize the total modeling error within the multibody dynamic simulation.
%\item Technical Approach:
%\begin{itemize}
%\item Compare the trimmed equilibrium values from the simulation to the steady state operating points from collected flight data.
%\item Compare the linearized state space model from the simulation to the transient response characteristics from collected flight data.
%\item Run linear regression and optimization techniques to identify the minimum error across all the adjustable physical parameters.
%\end{itemize}
%\item Timeline: Begins during \task{Param1}; spans two (2) weeks; completes the Multibody Dynamics Validation process.
%\item Exit Criteria: Simulation results have been correlated with collected flight data.
%\item Deliverables: A validated multibody dynamic model, that accurately depicts the flight characteristics of the TURN system, which can be used for subsequent prototype design.
%\end{itemize}
%
%
%
%
%%------------------------------
%\stepcounter{Task}
%{\bf Task \theTask:}
%\emph{Aerodynamic Model Refinement (15 weeks)}
%
%
%\stepcounter{SubTask}
%\hypertarget{Layout2}
%{Task \theTask.\theSubTask}:
%Mark 2 Layout and Sizing
%\TaskSpace
%\begin{itemize}
%\item Objectives: Define suitable physical parameters and dimensions for next the prototype.
%\item Technical Approach:
%\begin{itemize}
%\item Size the \emph{Mark 2} prototype with existing simulation models.
%\item Deliver prototype layout and sizing to UAV manufacturer and CFD developer.
%\item Create multisine signal injection algorithm for parameter identification.
%\end{itemize}
%\item Timeline: Initiates Task 2; spans two (2) weeks; critical path for both \task{CAD_New} and \task{CFD_New}.
%\item Exit Criteria: Physical layout and sizing parameters have been finalized.
%\item Deliverables: Vehicle sizing and dimensions.
%\end{itemize}
%
%
%\stepcounter{SubTask}
%\hypertarget{CAD_New}
%{Task \theTask.\theSubTask}:
%Develop CAD Model
%\TaskSpace
%\begin{itemize}
%\item Objectives: Create a 3D solid model of the physical \emph{Mark 2} prototype to facilitate the construction and manufacturing process.
%\item Technical Approach:
%\begin{itemize}
%\item Input dimensions into CAD software.
%\item Ensure all internal components fit within envelope.
%\item Confirm construction and manufacturing logistics are feasible.
%\end{itemize}
%\item Timeline: Begins after \task{Layout2}; spans two (2) weeks; leads into both \task{Hub} and \task{Fuse}.
%\item Exit Criteria: CAD model sufficiently represents the physical prototype.
%\item Deliverables: \emph{Mark 2} CAD model.
%\end{itemize}
%
%
%\stepcounter{SubTask}
%\hypertarget{Hub}
%{Task \theTask.\theSubTask}:
%Central Hub Construction
%\TaskSpace
%\begin{itemize}
%\item Objectives: Build a central hub with functional tether retracting mechanism, which can be used for both the \emph{Mark 2} and \emph{Mark 3} prototype embodiments.
%\item Technical Approach:
%\begin{itemize}
%\item Size retracting mechanism based on torque requirements.
%\item Design storage bay for supporting payload volumes.
%\item Construct central hub housing.
%\item Install retracting mechanism and avionics components.
%\end{itemize}
%\item Timeline: Begins during \task{CAD_New}; spans three (3) weeks; critical path for \task{Build2}.
%\item Exit Criteria: Central hub is assembled with a functional retracting mechanism.
%\item Deliverables: Completed central hub ready for vehicle assembly.
%\end{itemize}
%
%
%\stepcounter{SubTask}
%\hypertarget{Fuse}
%{Task \theTask.\theSubTask}:
%Fuselage Construction
%\TaskSpace
%\begin{itemize}
%\item Objectives: Build custom fuselage components that integrate with existing airfoil wing from the UAV manufacturer, to form a system that is representative of the TURN concept.
%\item Technical Approach:
%\begin{itemize}
%\item Assemble inner structural framework.
%\item Build outer fuselage shell.
%\item Attach each fuselage to its airfoil wing.
%\item Install batteries, motor, prop, avionics, etc.
%\end{itemize}
%\item Timeline: Begins after \task{CAD_New}; spans four (4) weeks; leads into for \task{Build2}.
%\item Exit Criteria: All four fuselages have been manufactured.
%\item Deliverables: Completed satellites ready for vehicle assembly.
%\end{itemize}
%
%
%\stepcounter{SubTask}
%\hypertarget{Build2}
%{Task \theTask.\theSubTask}:
%Vehicle Assembly
%\TaskSpace
%\begin{itemize}
%\item Objectives: Assemble TURN components into a working \emph{Mark 2} prototype.
%\item Technical Approach:
%\begin{itemize}
%\item Secure each satellite to its respective retracting mechanism.
%\item Test reel-out and reel-in capabilities.
%\item Range test wireless transmission between rigid bodies.
%\item Bench test all signal connections and software parameter settings.
%\end{itemize}
%\item Timeline: Begins during \task{Fuse}; spans two (2) weeks; critical path for \task{Flight2}.
%\item Exit Criteria: \emph{Mark 2} prototype is ready for flight testing.
%\item Deliverables: Completed \emph{Mark 2} prototype system.
%\end{itemize}
%
%
%\stepcounter{SubTask}
%\hypertarget{CFD_New}
%{Task \theTask.\theSubTask}:
%Develop CFD Analysis
%\TaskSpace
%\begin{itemize}
%\item Objectives: Develop an aerodynamic model utilizing CFD techniques, to predict forces and moments acting on the system at various flight conditions.
%\item Technical Approach:
%\begin{itemize}
%\item Develop mesh for CFD analysis.
%\item Utilize STAR-CCM+ and FlightStream CFD software suites.
%\item Setup baseline simulation to model trimmed equilibrium point.
%\item Run alternate scenarios, each adjusting a single variable from the trim condition. 
%\item Obtain the three-axis force and moment values for each iteration.
%\end{itemize}
%\item Timeline: Begins after \task{Layout2}; spans eight (8) weeks; critical path for \task{CFD_Val}.
%\item Exit Criteria: Predicted force and moment values have been generated for each condition.
%\item Deliverables: CFD outputs that can be compared against physical flight testing data.
%\end{itemize}
%
%
%\stepcounter{SubTask}
%\hypertarget{Flight2}
%{Task \theTask.\theSubTask}:
%Mark 2 Flight Testing
%\TaskSpace
%\begin{itemize}
%\item Objectives: Collect flight data that can be used to validate the multibody dynamic model.
%\item Technical Approach:
%\begin{itemize}
%\item Run taxi tests to demonstrate equal spacing between satellites, and confirm proper controls for collective and cyclic commands.
%\item Run multisine signal injection during hover to assess static equlibrium operating points.
%\item Run multisine signal injection ascending/descending with adjusted throttle to assess deviation from nominal airspeed.
%\item Run multisine signal injection ascending/descending with adjusted pitch attitude to assess deviation from nominal angle of attack.
%\end{itemize}
%\item Timeline: Begins after \task{Build2}; spans three (3) weeks; critical path for \task{CFD_Val}.
%\item Exit Criteria: All prescribed flight data sets have been recorded.
%\item Deliverables: \emph{Mark 2} flight data ready for parameter identification.
%\end{itemize}
%
%
%\stepcounter{SubTask}
%\hypertarget{Param2}
%{Task \theTask.\theSubTask}:
%Parameter Identification
%\TaskSpace
%\begin{itemize}
%\item Objectives: Obtain trimmed steady state operating points and transient response characteristics which can be evaluated against the CFD analysis results. 
%\item Technical Approach:
%\begin{itemize}
%\item Obtain equilibrium points for hover.
%\item Obtain equilibrium points for vertical translation.
%\item Obtain equilibrium points for horizontal translation.
%\item Obtain state space model for hover.
%\item Obtain state space model for vertical translation.
%\item Obtain state space model for horizontal translation.
%\item For each flight condition, derive the forces and moments that are present which match the recorded control inputs and system states.
%\end{itemize}
%\item Timeline: Begins during \task{Flight2}; spans two (2) weeks; leads into \task{CFD_Val}.
%\item Exit Criteria: Flight data has been processed into trimmed and state space parameters.
%\item Deliverables: Trimmed and linearized parameters which represent the \emph{Mark 2} prototype static and dynamic response characteristics.
%\end{itemize}
%
%
%\stepcounter{SubTask}
%\hypertarget{CFD_Val}
%{Task \theTask.\theSubTask}:
%CFD Refinement Process
%\TaskSpace
%\begin{itemize}
%\item Objectives: Use collected flight data to refine the CFD aerodynamic model.
%\item Technical Approach:
%\begin{itemize}
%\item Compare forces and moments from CFD to those generated during flight.
%\item Run linear regression and optimization techniques to identify the minimum error across the collected flight conditions.
%\item Revise settings within the CFD model to better represent the physical system.
%\end{itemize}
%\item Timeline: Begins after \task{Flight2} and during \task{Param2}; spans two (2) weeks; completes Aerodynamic Model Refinement.
%\item Exit Criteria: Force and moment outputs from CFD match physical flight characteristics.
%\item Deliverables: Validated CFD model to be used for subsequent TURN vehicle design.
%\end{itemize}
%
%
%
%
%%------------------------------
%\stepcounter{Task}
%{\bf Task \theTask:}
%\emph{Trade Studies and Optimizations (8 weeks)}
%
%
%\stepcounter{SubTask}
%\hypertarget{CFD_Scale}
%{Task \theTask.\theSubTask}:
%Scale Existing CFD Model
%\TaskSpace
%\begin{itemize}
%\item Objectives: Use refined CFD model and further improve the TURN aircraft performance.
%\item Technical Approach:
%\begin{itemize}
%\item Increase sizing to represent the \emph{Mark 3} system, and incorporate alternative elements for trade studies (winglets, dihedral, etc).
%\item Revise mesh for CFD analysis.
%\item Utilize STAR-CCM+ and FlightStream CFD software suites.
%\end{itemize}
%\item Timeline: Initiates Task 3; spans two (2) weeks; leads into \task{Trade}.
%\item Exit Criteria: Revised CFD model depicts the next prototype size and layout.
%\item Deliverables: Updated CFD model.
%\end{itemize}
%
%
%\stepcounter{SubTask}
%\hypertarget{Trade}
%{Task \theTask.\theSubTask}:
%Engineering Trade Studies
%\TaskSpace
%\begin{itemize}
%\item Objectives: Evaluate the cost/benefit of competing design elements and determine which features offer the greatest utility.
%\item Technical Approach:
%\begin{itemize}
%\item Evaluate diminishing returns on aspect ratio.
%\item Review weight versus efficiency of winglets.
%\item Assess outboard horizontal stabilizer effectiveness.
%\item Investigate asymmetric washout across wingspan.
%\end{itemize}
%\item Timeline: Begins during \task{CFD_Scale}; spans four (4) weeks; leads into \task{Opt}.
%\item Exit Criteria: All engineering analysis has been completed.
%\item Deliverables: Potentially revised TURN layout.
%\end{itemize}
%
%
%\stepcounter{SubTask}
%\hypertarget{Opt}
%{Task \theTask.\theSubTask}:
%Design Optimizations
%\TaskSpace
%\begin{itemize}
%\item Objectives: Utilize the refined CFD analysis tool and seek out physical dimensions that maximize aerodynamic efficiency and minimize power consumption.
%\item Technical Approach:
%\begin{itemize}
%\item Design an airfoil to the custom specifications required for the \emph{Mark 3} TURN system.
%\item Develop a specific pressure distribution for an ideal 2D airfoil.
%\item Tailor analysis to low Mach number, low Reynolds number, thin airfoil embodiment.
%\item Optimize control surface sizing and placement.
%\end{itemize}
%\item Timeline: Begins during \task{Trade}; spans four (4) weeks; completes the Trade Studies and Optimizations process.
%\item Exit Criteria: All optimization analysis has been completed.
%\item Deliverables: Potentially revised TURN sizing.
%\end{itemize}
%
%
%
%
%%------------------------------
%\stepcounter{Task}
%{\bf Task \theTask:}
%\emph{Airfoil Structural Assessment (24 weeks)}
%
%
%\stepcounter{SubTask}
%\hypertarget{Layout3}
%{Task \theTask.\theSubTask}:
%Mark 3 Layout and Sizing
%\TaskSpace
%\begin{itemize}
%\item Objectives: Define suitable physical parameters and dimensions for the next prototype.
%\item Technical Approach:
%\begin{itemize}
%\item Size the \emph{Mark 3} prototype with validated multibody model and refined CFD analysis.
%\item Deliver prototype layout and sizing to UAV manufacturer and FEA developer.
%\item Incorporate design revisions from engineering trade studies and system optimizations.
%\end{itemize}
%\item Timeline: Initiates Task 4; spans two (2) weeks; critical path for both \task{CAD_Scale} and \task{FEA_New}.
%\item Exit Criteria: Physical layout and sizing parameters have been finalized.
%\item Deliverables: Vehicle sizing and dimensions.
%\end{itemize}
%
%
%\stepcounter{SubTask}
%\hypertarget{CAD_Scale}
%{Task \theTask.\theSubTask}:
%Scale Existing CAD Model
%\TaskSpace
%\begin{itemize}
%\item Objectives: Revise previously developed CAD model to reflect larger prototype dimensions.
%\item Technical Approach:
%\begin{itemize}
%\item Increase sizing to represent the \emph{Mark 3} system, and incorporate revisions from trade studies and optimizations.
%\item Format output to be suitable for CNC machine and FEA investigation.
%\item Confirm packaging requirements to accommodate new hardware components.
%\end{itemize}
%\item Timeline: Begins after \task{Layout3}; spans two (2) weeks; leads into \task{Mold}.
%\item Exit Criteria: Revised CAD model depicts the next prototype size and layout.
%\item Deliverables: Updated CAD model.
%\end{itemize}
%
%
%\stepcounter{SubTask}
%\hypertarget{Mold}
%{Task \theTask.\theSubTask}:
%Composite Mold Construction
%\TaskSpace
%\begin{itemize}
%\item Objectives: Build the molds that will be used during the composite layup process.
%\item Technical Approach:
%\begin{itemize}
%\item Use CNC machine to precisely route desired mold surface contours.
%\item Prep mold surface for composite layup process.
%\end{itemize}
%\item Timeline: Begins during \task{CAD_Scale}; spans four (4) weeks; leads into \task{Vacuum}.
%\item Exit Criteria: Molds are ready for composite layup process.
%\item Deliverables: Composite molds.
%\end{itemize}
%
%
%\stepcounter{SubTask}
%\hypertarget{Vacuum}
%{Task \theTask.\theSubTask}:
%Vacuum Bag Layup
%\TaskSpace
%\begin{itemize}
%\item Objectives: Build custom airfoil wings utilizing the composite molds.
%\item Technical Approach:
%\begin{itemize}
%\item Use CNC machine to route low density foam core.
%\item Add stringer for increased stiffness.
%\item Wrap foam core with pre-preg carbon fiber fabric.
%\item Let resin cure within vacuum bagged system.
%\end{itemize}
%\item Timeline: Begins during \task{Mold}; spans six (6) weeks; leads into \task{Build3}.
%\item Exit Criteria: All composite pieces have completed the layup process.
%\item Deliverables: Composite parts ready for assembly.
%\end{itemize}
%
%
%\stepcounter{SubTask}
%\hypertarget{Build3}
%{Task \theTask.\theSubTask}:
%Vehicle Assembly
%\TaskSpace
%\begin{itemize}
%\item Objectives: Assemble TURN components into a working \emph{Mark 3} prototype.
%\item Technical Approach:
%\begin{itemize}
%\item Reuse previous central hub for \emph{Mark 3} system.
%\item Join the custom composite airfoil wings to their larger fuselages.
%\item Install internal hardware (motors, servos, avionics, etc).
%\item Test retracting mechanism docking apparatus.
%\end{itemize}
%\item Timeline: Begins after \task{Mold} and during \task{Vacuum}; spans four (4) weeks; critical path for \task{Flight3}.
%\item Exit Criteria: The \emph{Mark 3} prototype is ready for flight testing.
%\item Deliverables: Completed \emph{Mark 3} prototype system.
%\end{itemize}
%
%
%\stepcounter{SubTask}
%\hypertarget{FEA_New}
%{Task \theTask.\theSubTask}:
%Develop FEA Model
%\TaskSpace
%\begin{itemize}
%\item Objectives: Develop a structural model utilizing FEA techniques, to predict bend and twist magnitudes when subjected to various applied moments.
%\item Technical Approach:
%\begin{itemize}
%\item Develop mesh for FEA program.
%\item Utilize Nastran solver to execute the routine.
%\item Run bend and twist scenarios for several load cases.
%\end{itemize}
%\item Timeline: Begins after \task{Layout3}; spans twelve (12) weeks; critical path for both \task{FEA_Stat} and \task{FEA_Dyn}.
%\item Exit Criteria: Predicted deflection values have been generated for each load case.
%\item Deliverables: FEA outputs that can be compared against static and dynamic tests.
%\end{itemize}
%
%
%\stepcounter{SubTask}
%\hypertarget{Bend}
%{Task \theTask.\theSubTask}:
%Bend Test
%\TaskSpace
%\begin{itemize}
%\item Objectives: Measure bend deflection on a wing section to compare with static FEA model.
%\item Technical Approach:
%\begin{itemize}
%\item Secure fuselage end of wingspan.
%\item Apply dead weight load to opposite end of wingspan.
%\item Measure amount of deflection for several load values.
%\end{itemize}
%\item Timeline: Begins during \task{Vacuum}; spans two (2) weeks; leads into \task{FEA_Stat}.
%\item Exit Criteria: Airfoil bending deflection has been recorded for several applied loads.
%\item Deliverables: Bend test results.
%\end{itemize}
%
%
%\stepcounter{SubTask}
%\hypertarget{Twist}
%{Task \theTask.\theSubTask}:
%Torsion Test
%\TaskSpace
%\begin{itemize}
%\item Objectives: Measure twist deflection on a wing section to compare with static FEA model.
%\item Technical Approach:
%\begin{itemize}
%\item Secure fuselage end of wingspan.
%\item Constrain opposite end to single degree of rotation.
%\item Apply a torque to induce twist along the wingspan.
%\item Measure angle of rotation for several applied moments.
%\end{itemize}
%\item Timeline: Begins during \task{Vacuum}; spans two (2) weeks; leads into \task{FEA_Stat}.
%\item Exit Criteria: Airfoil twist rotation has been recorded for several applied moments.
%\item Deliverables: Torsion test results.
%\end{itemize}
%
%
%\stepcounter{SubTask}
%\hypertarget{FEA_Stat}
%{Task \theTask.\theSubTask}:
%FEA Static Assessment
%\TaskSpace
%\begin{itemize}
%\item Objectives: Compare static bend and torsion test results to values predicted from the FEA.
%\item Technical Approach:
%\begin{itemize}
%\item Compare bend deflections to FEA results.
%\item Compare twist rotations to FEA results.
%\item Adjust CAD and FEA parameters to accurately reflect physical component.
%\end{itemize}
%\item Timeline: Begins after \task{FEA_New} and during \task{Bend} and \task{Twist}; spans two (2) weeks; critical path for \task{Flight3}.
%\item Exit Criteria: FEA static predictions match physical test results.
%\item Deliverables: Validated static structural model.
%\end{itemize}
%
%
%\stepcounter{SubTask}
%\hypertarget{Flight3}
%{Task \theTask.\theSubTask}:
%Mark 3 Flight Testing
%\TaskSpace
%\begin{itemize}
%\item Objectives: Collect flight data that can be used to validate the dynamic structural model.
%\item Technical Approach:
%\begin{itemize}
%\item Repeat the multisine signal injection process that was used on the \emph{Mark 2} system.
%\item Apply sinusoidal collective elevator control inputs, to induce a vertical oscillation within the TURN system, which will dynamically deflect the wings during flight.
%\item Repeat the process across several different frequencies and magnitudes.
%\end{itemize}
%\item Timeline: Begins after \task{Build3} and \task{FEA_Stat}; spans six (6) weeks; leads into both \task{Param3} and \task{FEA_Dyn}.
%\item Exit Criteria: All prescribed flight data sets have been recorded.
%\item Deliverables: \emph{Mark 3} flight data ready for parameter identification.
%\end{itemize}
%
%
%\stepcounter{SubTask}
%\hypertarget{Param3}
%{Task \theTask.\theSubTask}:
%Parameter Identification
%\TaskSpace
%\begin{itemize}
%\item Objectives: Obtain trimmed steady state operating points and transient response characteristics which can be evaluated against the dynamic structural analysis.
%\item Technical Approach:
%\begin{itemize}
%\item Repeat the parameter identification process that was used on the \emph{Mark 2} system.
%\item Review the flight data and look for deviations within the system states that can be accounted for from flexibility within the slender custom airfoil structures.
%\end{itemize}
%\item Timeline: Begins during \task{Flight3}; spans three (3) weeks; leads into \task{FEA_Dyn}.
%\item Exit Criteria: Flight data has been processed into trimmed and state space parameters.
%\item Deliverables: Trimmed and linearized parameters which represent the \emph{Mark 3} prototype static and dynamic response characteristics.
%\end{itemize}
%
%
%\stepcounter{SubTask}
%\hypertarget{FEA_Dyn}
%{Task \theTask.\theSubTask}:
%FEA Dynamic Assessment
%\TaskSpace
%\begin{itemize}
%\item Objectives: Compare structural dynamic response with collected flight data.
%\item Technical Approach:
%\begin{itemize}
%\item Obtain control induced oscillation frequency and magnitude from recorded flight data.
%\item Input appropriate parameters into FEA model and obtain anticipated dynamic response.
%\item Compare predicted results to actual data, and refine model as needed.
%\end{itemize}
%\item Timeline: Begins during \task{Flight3} and \task{Param3}; spans (3) weeks; completes Airfoil Structural Assessment.
%\item Exit Criteria: FEA dynamic predictions match physical test results.
%\item Deliverables: Validated dynamic structural model.
%\end{itemize}
%
%
%
%
%
%
%
%
%%------------------------------------------------------------
%\newpage
%\begin{thebibliography}{99}
%%------------------------------------------------------------
%
%
%\PubSpace
%\bibitem{Keshmiri}
%Keshmiri, M., A. K. Misra, and V. J. Modi. ``General Formulation for N-body Tethered Satellite System Dynamics.'' \emph{Journal of Guidance, Control, and Dynamics}; Vol. 19, No. 1, pp. 75-83; 1996. \url{http://arc.aiaa.org/doi/10.2514/3.21582}
%
%
%\PubSpace
%\bibitem{Mattos}
%Mattos, Bento Silva de, Ney R. Secco, Eduardo F. Salles. ``Optimal Design of a High-Altitude Solar-Powered Unmanned Airplane.'' \emph{Journal of Aerospace Technology Management}; Vol. 5, No. 3, pp. 349-361; 2013. \url{http://www.jatm.com.br/ojs/index.php/jatm/article/view/223}
%
%
%\PubSpace
%\bibitem{Nickol}
%Nickol, Craig L., Mark D. Guynn, Lisa L. Kohout, and Thomas A. Ozoroski. ``High Altitude Long Endurance Air Vehicle Analysis of Alternatives and Technology Requirements Development.'' \emph{Aerospace Sciences Meeting}; Reno, NV; Jan 2007. \url{http://arc.aiaa.org/doi/abs/10.2514/6.2007-1050}
%
%
%\PubSpace
%\bibitem{Pizarro-Chong}
%Pizarro-Chong, Ary, and Arun K. Misra.  ``Dynamics of Multi-Tethered Satellite Formations Containing a Parent Body.'' \emph{Acta Astronautica}; Vol. 63, pp. 1188-1202; 2008. \url{http://www.sciencedirect.com/science/article/pii/S0094576508002452}
%
%
%\PubSpace
%\bibitem{Selfridge_TURN}
%Selfridge, Justin M. \emph{Achieving Eternal Flight With a Tethered Uni-Rotor Network (TURN) Aircraft: A Complete Development of the Nonlinear Dynamic Model and Controller Architecture}. Doctorate of Philosophy dissertation. University of Virginia; Charlottesville, VA; May 2017 [pending].
%
%
%\PubSpace
%\bibitem{Selfridge_OF_IEEE}
%Selfridge, Justin M., and Gang Tao. ``Multivariable Output Feedback MRAC for a Quadrotor UAV.'' \emph{IEEE American Control Conference (ACC)}; Boston, MA; Jul 2016.  \url{http://ieeexplore.ieee.org/document/7524962/}
%
%
%\PubSpace
%\bibitem{Selfridge_Plant}
%Selfridge, Justin M., and Gang Tao. ``Centrifugally Stiffened Rotor: A Complete Derivation of the Plant Model with Nonlinear Dynamics.'' \emph{AIAA Aviation Technology, Integration, and Operations Conference}; Dallas, TX; Jun 2015.  \url{http://arc.aiaa.org/doi/abs/10.2514/6.2015-2735}
%
%
%\PubSpace
%\bibitem{Selfridge_Inner}
%Selfridge, Justin M., and Gang Tao. ``Centrifugally Stiffened Rotor: A Complete Derivation and Simulation of the Inner Loop Controller.'' \emph{AIAA Guidance Navigation and Control (GNC) Conference}; Kissimmee, FL; Jan 2015.  \url{http://arc.aiaa.org/doi/abs/10.2514/6.2015-0073}
%
%
%\PubSpace
%\bibitem{Selfridge_SF_SSCE}
%Selfridge, Justin M., and Gang Tao. ``A Multivariable Adaptive Controller for a Quadrotor with Guaranteed Matching Conditions.'' \emph{System Science and Control Engineering}; Vol. 2, No. 1, pp. 24-33; 2014. \url{http://www.tandfonline.com/doi/pdf/10.1080/21642583.2013.879050}
%
%
%\PubSpace
%\bibitem{Selfridge_SF_IEEE}
%Selfridge, Justin M., and Gang Tao. ``A Multivariable Adaptive Controller for A Quadrotor with Guaranteed Matching Conditions.'' \emph{IEEE American Control Conference (ACC)}; Portland, OR; Jun 2014.  \url{http://ieeexplore.ieee.org/document/6859355/}
%
%
%\PubSpace
%\bibitem{Selfridge_Local_GNC}
%Selfridge, Justin M. ``An On-Board Camera-Based Localization Solution.'' \emph{AIAA Guidance Navigation and Control (GNC) Conference}; Minneapolis, MN; Aug 2012. \url{http://arc.aiaa.org/doi/abs/10.2514/6.2012-4847}
%
%
%\PubSpace
%\bibitem{Selfridge_Local_AUVSI}
%Selfridge, Justin M. ``An On-Board Camera-Based Localization Solution.'' \emph{AUVSI Unmanned Systems Conference}; Las Vegas, NV; Aug 2012.  \url{https://www.researchgate.net/publication/268558033_An_On-Board_Camera-Based_Localization_Solution}
%
%
%\PubSpace
%\bibitem{Selfridge_Master}
%Selfridge, Justin M. \emph{A Complete Autonomous Vehicle Solution: An ASV Case Study}. Master's thesis. Old Dominion University; Norfolk, VA; May 2012.  \url{https://books.google.com/books/about/A_Complete_Autonomous_Vehicle_Solution.html?id=31qEMwEACAAJ}
%
%
%\PubSpace
%\bibitem{Stoneking}
%Stoneking, Eric. ``Newton-Euler Dynamic Equations of Motion for a Multi-Body Spacecraft.'' \emph{AIAA Guidance, Navigation and Control Conference and Exhibit}; Hilton Head, SC; Aug 2007. \url{http://arc.aiaa.org/doi/abs/10.2514/6.2007-6441}
%
%
%\PubSpace
%\bibitem{Tealgroup2015}
%Teal Group. ``Teal Group Predicts Worldwide UAV Production Will Total \$93 Billion in Its 2015 UAV Market Profile and Forecast.''  Website; Aug 14, 2015. \url{http://www.prnewswire.com/news-releases/teal-group-predicts-worldwide-uav-production-will-total-93-billion-in-its-2015-uav-market-profile-and-forecast-300128745.html}
%
%
%\PubSpace
%\bibitem{Tealgroup2016}
%Teal Group. ``Teal Group Predicts Worldwide Civil UAS Production Will Total \$65 Billion in Its 2016 UAS Market Profile and Forecast.'' Website; Jul 7, 2016. \url{http://www.prnewswire.com/news-releases/teal-group-predicts-worldwide-civil-uas-production-will-total-65-billion-in-its-2016-uas-market-profile-and-forecast-300295255.html}
%
%
%\PubSpace
%\bibitem{Xian-Zhong}
%Xian-Zhong, Gao, Hou Zhong-Xi, Guo Zheng, Zhu Xiong-Feng, Liu Jian-Xia, and Chen Xiao-Qian. ``Parameter Determination for Concept Design of Solar-Powered, High-Altitude Long-Endurance UAV.'' \emph{Aircraft Engineering and Aerospace Technology}; Vol. 85, No. 4, pp. 293-303; 2013. \url{http://www.emeraldinsight.com/doi/abs/10.1108/AEAT-Jan-2012-0011}
%
%
%\PubSpace
%\bibitem{Zhao}
%Zhao, Zhenjun, and Gexue Ren. ``Multibody Dynamic Approach of Flight Dynamics and Nonlinear Aeroelasticity of Flexible Aircraft.'' \emph{AIAA Journal}; Vol. 49, No. 1, pp. 41-54; 2011. \url{http://arc.aiaa.org/doi/abs/10.2514/1.45334?journalCode=aiaaj}
%
%
%\PubSpace
%\bibitem{Xiongfeng}
%Zhu Xiongfeng, Guo Zheng, Fan Rongfei, Hou Zhongxi, and Gao Xianzhong. ``How High Can Solar-Powered Airplanes Fly.'' \emph{Journal of Aircraft}; Vol. 51, No. 5, pp. 1653-1659; 2014. \url{http://arc.aiaa.org/doi/abs/10.2514/1.C032333}
%
%
%\end{thebibliography}
%
%
%
%
%
%
%
%
%
%
%
%
%
%
%
%
